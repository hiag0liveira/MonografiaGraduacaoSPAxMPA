\chapter{Metodologia}
\label{cap:metodologia}

\section{Levantamento Teórico}
\label{sec:levantamento-teorico}

O levantamento teórico consistiu na revisão e sistematização dos principais conceitos, tecnologias e práticas relacionadas à renderização de conteúdo em aplicações web, com foco nas abordagens de \english{\acrfull{csr}} e \english{\acrfull{ssr}}. Essa etapa teve como objetivo fornecer o embasamento necessário para o desenvolvimento do estudo de caso e da análise comparativa proposta neste trabalho.

A fundamentação iniciou-se com a exploração dos princípios do desenvolvimento web moderno, incluindo a arquitetura cliente-servidor, o funcionamento do protocolo \acrshort{http} e as tecnologias essenciais do \textit{frontend}: \acrshort{html}, \acrshort{css} e JavasSript. Estes elementos formam a base para compreender como as estratégias de renderização operam, tanto no lado do cliente quanto no lado do servidor.

Em seguida, foram estudadas em detalhes as abordagens \acrshort{csr} e \acrshort{ssr}. A renderização no lado do cliente (\acrshort{csr}) foi analisada quanto ao seu funcionamento típico em aplicações \english{\acrfull{spa}}, caracterizadas por uma única página carregada inicialmente, com atualizações dinâmicas de conteúdo via JavaScript. Essa abordagem oferece vantagens como maior interatividade e fluidez na navegação, além de transições rápidas entre páginas internas. No entanto, apresenta desvantagens como maior tempo de carregamento inicial e limitações de indexação por mecanismos de busca.

Por outro lado, a renderização no lado do servidor (\acrshort{ssr}) foi abordada sob a ótica de desempenho inicial otimizado e maior compatibilidade com \acrshort{seo}, pois o conteúdo é entregue já renderizado ao navegador. Essa abordagem é comumente utilizada em aplicações do tipo \english{\acrfull{mpa}}, que possuem múltiplas páginas distintas e se beneficiam da pré-renderização para melhorar a performance inicial, a acessibilidade e a visibilidade em mecanismos de busca. Como contraponto, o SSR demanda maior processamento no servidor e pode aumentar a complexidade da infraestrutura.

Além dessas duas abordagens principais, o estudo também incluiu métodos híbridos como o \english{\acrfull{ssg}}, \english{\acrfull{isr}} e {\acrfull{dsg}}, que visam equilibrar performance, escalabilidade e atualizações de conteúdo dinâmico, especialmente em aplicações que exigem alta eficiência e atualizações frequentes.

O levantamento teórico foi complementado por uma análise dos principais \textit{frameworks} e bibliotecas utilizados no desenvolvimento web atual, como \textit{React}, \textit{Vue.js}, \textit{Angular}, \textit{Next.js}, \textit{Nuxt} e \textit{SvelteKit}, e por uma discussão sobre os impactos de cada abordagem na experiência do usuário (\acrshort{ux}), incluindo aspectos como \acrshort{seo}, acessibilidade, tempo de carregamento e interatividade.

Esse base teórica serviu para as próximas etapas da pesquisa, em especial o mapeamento sistemático da literatura e a condução do estudo de caso prático.

\section{Mapeamento Sistemático da Literatura}
\label{sec:mapemento-sistematico-da-literatura}

O mapeamento sistemático da literatura teve como objetivo identificar, selecionar e analisar estudos acadêmicos relevantes que abordassem comparações entre as abordagens de renderização \english{\acrshort{csr}} e \english{\acrshort{ssr}} no contexto do desenvolvimento de aplicações web. Essa etapa foi essencial para compreender o estado da arte, bem como identificar lacunas e oportunidades para a realização do estudo de caso proposto neste trabalho.

A estratégia de busca foi estruturada com o apoio da metodologia \textit{PICOC}, que define os elementos População, Intervenção, Comparação, Resultado e Contexto, com o intuito de guiar a construção das expressões de busca e garantir abrangência e precisão nos resultados. As principais bases de dados utilizadas incluíram Periódicos Capes e Scopus, por oferecerem amplo acervo e suporte a pesquisas refinadas.

Foram utilizadas expressões booleanas combinando termos como \textit{Client-Side Rendering}, \textit{Server-Side Rendering}, \textit{Web Performance}, \textit{SEO}, \textit{UX} e \textit{Frontend Architecture}. Após a aplicação dos critérios de inclusão e exclusão, os artigos resultantes foram classificados e analisados de acordo com sua relevância, tipo de abordagem estudada, metodologias utilizadas e principais conclusões.

A seleção final contemplou trabalhos que abordavam métricas de desempenho, tempo de carregamento, interatividade, \acrshort{seo} e experiência do usuário. Além disso, foram considerados estudos que analisavam o uso de frameworks modernos como \textit{React}, \textit{Next.js}, \textit{Nuxt.js} e \textit{Angular Universal}, além de pesquisas aplicadas em contextos reais de produção.

Como resultado, foi possível consolidar uma visão abrangente sobre os desafios, vantagens e limitações de cada abordagem, fornecendo subsídios importantes para a execução do estudo de caso prático apresentado nos capítulos seguintes. O mapeamento sistemático também evidenciou a escassez de estudos nacionais aplicados ao tema, reforçando a relevância deste trabalho no cenário acadêmico e profissional brasileiro.


\section{Estudo de Caso Prático}
\label{sec:estudo-de-caso-pratico}

O escopo da aplicação a ser desenvolvida visa avaliar os impactos das abordagens de renderização \english{\acrshort{csr}} e \english{\acrshort{ssr}} no desenvolvimento de aplicações web. Para isso, será considerado o contexto de uma empresa fictícia cujo time de gestores precisa decidir qual modelo de renderização adotar para a criação de sua nova aplicação web. Serão desenvolvidas duas versões de uma mesma aplicação: uma utilizando \acrshort{csr} e outra baseada em \acrshort{ssr}. Ambas as implementações terão as mesmas funcionalidades, aparência visual e estrutura de dados, de modo a permitir uma análise equitativa quanto ao desempenho, à experiência do usuário e à otimização para mecanismos de busca.

A aplicação a ser desenvolvida simulará um catálogo de produtos, com navegação por páginas, carregamento de dados via \textit{API} e exibição de informações detalhadas. Esse modelo foi escolhido por representar um cenário comum na web moderna, abrangendo interações usuais como carregamento dinâmico de conteúdo, roteamento entre páginas e exibição de listas e detalhes.

Cada versão será implementada conforme os princípios da respectiva abordagem de renderização: a versão \acrshort{csr} terá a interface renderizada predominantemente no navegador do usuário, enquanto a versão \acrshort{ssr} contará com o conteúdo renderizado no servidor e enviado ao cliente já montado.

Durante o desenvolvimento, serão seguidas boas práticas de acessibilidade, responsividade e otimização para \acrshort{seo}, garantindo que ambas as versões possam ser avaliadas com base em critérios equivalentes. As métricas a serem analisadas incluirão tempo de carregamento, tempo até a interatividade, consumo de recursos, desempenho percebido, qualidade do código e compatibilidade com ferramentas de análise de \acrshort{seo}.

A coleta de dados será realizada por meio de ferramentas como Google Lighthouse, PageSpeed Insights e WebPageTest, além de testes manuais com usuários, a fim de observar qualitativamente a experiência de uso. Esses dados fundamentarão a análise comparativa e a discussão dos resultados, que serão apresentados nas seções seguintes.

\section{Coleta de Dados e Testes}
\label{sec:coleta-de-dados-e-testes}

Esta seção descreve o procedimento de coleta e análise dos dados obtidos a partir da comparação entre as aplicações desenvolvidas com \english{\acrshort{csr}} e \english{\acrshort{ssr}}. O objetivo é mensurar o desempenho, a eficiência e a experiência do usuário proporcionada por cada aplicação.

\subsection{Definição das Métricas}

As métricas utilizadas para a avaliação foram selecionadas com base nas recomendações do Google e nos indicadores mais relevantes para mensurar a performance e a experiência do usuário em aplicações web modernas. São elas:

\begin{itemize}
    \item \textbf{Time to First Byte (TTFB)}: Tempo até o primeiro byte da resposta ser recebido;
    \item \textbf{First Contentful Paint (FCP)}: Tempo até a exibição do primeiro conteúdo visível;
    \item \textbf{Largest Contentful Paint (LCP)}: Tempo até o carregamento do maior bloco de conteúdo;
    \item \textbf{Time to Interactive (TTI)}: Tempo até a página estar completamente interativa;
    \item \textbf{Cumulative Layout Shift (CLS)}: Medida de estabilidade visual da interface;
    \item \textbf{Número de Requisições HTTP} e \textbf{Uso de Cache};
    \item \textbf{Consumo de recursos do navegador}: Memória e CPU;
\end{itemize}

\subsection{Ferramentas de Teste}

Para a análise do desempenho das aplicações, foram utilizadas ferramentas amplamente reconhecidas no diagnóstico de aplicações web. Essas ferramentas permitiram a coleta de métricas técnicas relevantes, tanto do ponto de vista do usuário quanto do comportamento interno das aplicações:

\begin{itemize}
    \item \textbf{Google Lighthouse}: Ferramenta automatizada de código aberto para auditoria de desempenho, acessibilidade, SEO e melhores práticas em aplicações web;
    \item \textbf{PageSpeed Insights}: Serviço do Google que avalia o desempenho da página em dispositivos móveis e desktops, destacando métricas como FCP, LCP e TTI;
    \item \textbf{WebPageTest}: Utilizada para testes avançados de tempo de carregamento, número de requisições e análise de cache em diferentes condições de rede;
    \item \textbf{Core Web Vitals}: Conjunto de métricas essenciais definido pelo Google para mensurar a qualidade da experiência do usuário, incluindo LCP, FID e CLS;
    \item \textbf{Chrome DevTools}: Ferramenta integrada ao navegador Google Chrome que possibilita a análise detalhada de rede, execução de scripts, uso de memória e inspeção do processo de hidratação no SSR.
\end{itemize}

\subsection{Ambiente de Testes}

Os testes foram executados localmente, com as aplicações rodando diretamente na máquina do desenvolvedor. A máquina utilizada foi um notebook \textbf{Asus Vivobook M1502IA}, com as seguintes configurações:

\begin{itemize}
    \item Processador \textbf{AMD Ryzen 5 4600H}, 6 núcleos e 12 threads;
    \item \textbf{20GB de memória RAM DDR4, 3200 MHz};
    \item \textbf{SSD de 256GB};
    \item Sistema operacional \textbf{Ubuntu 24.04.2 LTS}, ambiente GNOME;
    \item Kernel \textbf{Linux 6.14.0-27-generic}.
\end{itemize}


\subsection{Execução dos Testes}

Os testes foram realizados em diferentes horários do dia, em rede Wi-Fi com largura de banda média de 15 Mbps. Para cada versão (\acrshort{csr} e \acrshort{ssr}), os testes foram repetidos \textbf{cinco vezes} por métrica, com o objetivo de reduzir variações nos resultados e garantir maior confiabilidade estatística.

\subsection{Registro e Organização dos Dados}

Todos os dados coletados foram registrados em planilhas estruturadas, organizadas por data, horário, métrica e tipo de abordagem. As informações foram categorizadas para facilitar a posterior análise comparativa.

\subsection{Método de Análise}

Os valores obtidos foram analisados por meio do \textbf{cálculo de médias}, \textbf{desvios padrão} e \textbf{comparações diretas} entre \acrshort{csr} e \acrshort{ssr}. As principais conclusões foram representadas graficamente e sintetizadas em tabelas, conforme apresentado na Seção de Resultados e Discussões.

\section{Síntese e Discussão}
\label{sec:sintese-e-discussao}

De maneira geral, será feita uma análise comparativa entre as abordagens \english{\acrshort{csr}} e \english{\acrshort{ssr}}, com base nas informações que serão obtidas ao longo da implementação descrita no Capítulo \ref{cap:estudo_caso1}. Essa análise terá como foco os aspectos técnicos e funcionais relacionados ao desempenho, tempo de carregamento, interatividade, experiência do usuário e compatibilidade com mecanismos de busca.

A discussão será conduzida de forma imparcial, considerando os dados coletados por meio de ferramentas especializadas, como Google Lighthouse, PageSpeed Insights e WebPageTest, bem como possíveis observações manuais obtidas em testes de usabilidade. A intenção será compreender como cada abordagem impacta diferentes requisitos de projeto e quais implicações essas escolhas podem trazer no contexto do desenvolvimento de aplicações web.

Além da comparação direta entre {\acrshort{csr}} e {\acrshort{ssr}}, poderá ser explorada, conforme a relevância no decorrer da análise, a viabilidade de estratégias híbridas, como o uso de \english{\acrshort{ssg}} e \english{\acrshort{isr}}, no sentido de identificar soluções que conciliem desempenho, escalabilidade e experiência do usuário.

Essa etapa terá caráter investigativo e descritivo, sem antecipar conclusões, e servirá como base para reflexões futuras e para a elaboração de recomendações alinhadas aos objetivos definidos neste trabalho.
