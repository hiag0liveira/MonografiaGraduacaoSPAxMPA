\chapter{Fundamentação Teórica}
\label{cap:fundamentacao}

Este capítulo apresenta os conceitos de \english{\acrfull{csr}} e \english{\acrfull{ssr}}, abordando os princípios fundamentais do desenvolvimento web relacionados à renderização de conteúdo. Também são discutidos aspectos como \acrshort{seo}, desempenho, infraestrutura de serviços e impacto na experiência do usuário, estabelecendo a base teórica para o estudo de caso desenvolvido neste trabalho.

\section{Fundamentos de Desenvolvimento Web}
\label{sec:fundamentos-devweb}
Para entender como as abordagens \acrshort{ssr} e \acrshort{csr} se inserem no cenário de desenvolvimento web, é fundamental revisar protocolos, modelos de arquitetura e ferramentas.
Os fundamentos de desenvolvimento web englobam os princípios, tecnologias e práticas essenciais para a criação e manutenção de aplicações acessíveis via internet. 

O desenvolvimento web baseia-se na arquitetura cliente-servidor, onde o cliente (geralmente um navegador) solicita recursos ao servidor, que processa essas requisições e retorna os dados necessários. Essa interação é mediada por protocolos como o 
\english{\acrfull{http}} , que define as regras de comunicação entre cliente e servidor.

As tecnologias fundamentais incluem \english{\acrfull{html}} para estruturação do conteúdo, \english{\acrfull{css}} para estilização e JavaScript para interatividade. Essas linguagens permitem a construção de interfaces dinâmicas e responsivas. Além disso, o desenvolvimento web envolve práticas como controle de versão, testes automatizados e integração contínua, que garantem a qualidade e a escalabilidade das aplicações \cite{fundamentosDevWeb}. 

\subsection{Arquitetura Cliente-Servidor}
\label{subsec:Arquitetura Cliente-Servidor}

A arquitetura cliente-servidor é um modelo amplamente adotado no desenvolvimento de aplicações web, caracterizado pela separação entre dois componentes principais: o \textbf{cliente}, responsável pela interface com o usuário, e o \textbf{servidor}, que processa solicitações e fornece os recursos necessários \cite{clienteServidorControlNet}.

Nesse modelo, os clientes — como navegadores em diferentes dispositivos — enviam requisições através da internet, enquanto os servidores respondem disponibilizando dados, arquivos e serviços. Essa divisão de responsabilidades favorece a escalabilidade, facilita a manutenção e permite que cliente e servidor operem em plataformas distintas~\cite{fundamentosDevWeb}.

A Figura~\ref{fig:cliente-servidor} ilustra, de forma simplificada, esse fluxo de comunicação entre cliente e servidor.

\begin{figure}[H]
  \centering
  \caption{Comunicação entre cliente e servidor.}
  \includegraphics[width=0.8\textwidth]{media/cliente_servidor.jpeg}
  \legend{Fonte: \cite{fundamentosDevWeb} }
  \label{fig:cliente-servidor}
\end{figure}


A arquitetura cliente-servidor apresenta características que contribuem para sua ampla adoção em aplicações web. Entre elas, destaca-se a \textbf{distribuição de responsabilidades}, onde o servidor gerencia dados e processos mais complexos, enquanto o cliente lida com a interface e a interação com o usuário. 

Outro ponto relevante é a \textbf{independência entre plataformas}, possibilitada pelo uso de protocolos padronizados, o que permite a comunicação entre diferentes dispositivos e sistemas operacionais. Além disso, esse modelo favorece a \textbf{facilidade de manutenção}, já que atualizações podem ser feitas no servidor sem necessidade de intervenção nos dispositivos dos usuários.

Na web, essa arquitetura é implementada por padrão: navegadores atuam como clientes, enviando requisições \acrshort{http} que são processadas por servidores, os quais respondem com páginas e recursos solicitados~\cite{fundamentosDevWeb}.


\subsection{Protocolo \acrshort{http}}
\label{subsec:http}
O \textbf{Protocolo de Transferência de Hipertexto} (\acrshort{http}) é a base da comunicação na World Wide Web, definindo como clientes (navegadores) e servidores trocam informações. Ele especifica a estrutura das requisições e respostas, permitindo a recuperação de recursos como documentos \acrshort(html), imagens e vídeos \cite{mdn_http}.


 O \textbf{Funcionamento do \acrshort{http}}opera no modelo cliente-servidor, onde o cliente inicia uma requisição e o servidor responde com os recursos solicitados ou mensagens de erro, se aplicável. Cada interação consiste em uma mensagem de requisição do cliente e uma mensagem de resposta do servidor. As mensagens \acrshort{http} são compostas por:

\begin{itemize}
    \item \textbf{Linha de início:} Indica o método \acrshort{http} (como \texttt{GET} ou \texttt{POST}) e o caminho do recurso.
    \item \textbf{Cabeçalhos:} Fornecem informações adicionais sobre a requisição ou resposta, como tipo de conteúdo e codificação.
    \item \textbf{Corpo:} Contém os dados enviados ou recebidos, sendo opcional dependendo do método utilizado.
\end{itemize}

\begin{flushright}
    \cite{mdn_http}
\end{flushright}


\textbf{Métodos \acrshort{http}} são operações definidas pelo protocolo que especificam a ação a ser realizada em um recurso. Os métodos mais comuns incluem:

\begin{itemize}
    \item \textbf{GET:} Solicita a representação de um recurso específico. Requisições GET devem ser utilizadas apenas para recuperar dados.
    \item \textbf{POST:} Envia dados ao servidor para processamento, como o envio de formulários.
    \item \textbf{PUT:} Atualiza um recurso existente ou cria um novo se não existir.
    \item \textbf{DELETE:} Remove um recurso específico.
    \item \textbf{HEAD:} Similar ao GET, mas solicita apenas os cabeçalhos da resposta, sem o corpo.
\end{itemize}
Cada método possui uma finalidade específica e deve ser utilizado conforme a necessidade da aplicação \cite{wikipedia_http}.


\textbf{Códigos de Status \acrshort{http}} são códigos de três dígitos que indicam o resultado de uma requisição feita pelo cliente ao servidor. Eles são agrupados em cinco classes principais:

\begin{itemize}
    \item \textbf{1xx (Informativo):} Indica que a requisição foi recebida e o processo continua.
    \item \textbf{2xx (Sucesso):} Indica que a requisição foi bem-sucedida. Exemplo: 200 OK.
    \item \textbf{3xx (Redirecionamento):} Indica que é necessário tomar medidas adicionais para completar a requisição. Exemplo: 301 Moved Permanently.
    \item \textbf{4xx (Erro do Cliente):} Indica que houve um erro na requisição do cliente. Exemplo: 404 Not Found.
    \item \textbf{5xx (Erro do Servidor):} Indica que o servidor falhou ao processar uma requisição válida. Exemplo: 500 Internal Server Error.
\end{itemize}

Esses códigos auxiliam na identificação e resolução de problemas durante a comunicação \acrshort{http} \cite{mdn_http}.


\textbf{Evolução do \acrshort{http}} refere-se às revisões progressivas do protocolo com o objetivo de aprimorar sua eficiência, segurança e desempenho ao longo do tempo. As principais versões são:

\begin{itemize}
    \item \textbf{HTTP/1.0:} Primeira versão oficial do protocolo, em que cada requisição exigia uma nova conexão com o servidor.
    \item \textbf{HTTP/1.1:} Introduziu conexões persistentes, permitindo múltiplas requisições por conexão. Trouxe também melhorias no controle de cache e suporte a novos métodos.
    \item \textbf{HTTP/2:} Implementou multiplexação, compressão de cabeçalhos e priorização de fluxos, resultando em uma transferência de dados mais rápida e eficiente.
    \item \textbf{HTTP/3:} Baseado no protocolo \acrshort{quic}, substitui o \acrshort{tcp} pelo \acrshort{udp}, oferecendo conexões mais rápidas e seguras, com menor latência e melhor desempenho em redes instáveis.
\end{itemize}

Essas atualizações refletem a evolução das necessidades da web e a busca por protocolos mais robustos e otimizados \cite{cloudflare_http}.

\textbf{\acrshort{http} e \acrshort{https}} representam protocolos utilizados para comunicação na web, com a principal distinção centrada na segurança da transmissão dos dados.

O \textbf{\english{\acrfull{https}}} é uma extensão do \acrshort{http} que adiciona uma camada de proteção por meio do protocolo \english{\acrfull{tls}} ou, anteriormente, \english{\acrfull{ssl}}. Essa camada de segurança garante a confidencialidade, integridade e autenticidade dos dados trafegados entre cliente e servidor. 

A criptografia utilizada impede que terceiros acessem ou modifiquem as informações transmitidas, o que é fundamental em transações sensíveis, como cadastros, pagamentos e autenticações. Além disso, o uso de \textit{certificados digitais} garante que o site visitado é realmente aquele que afirma ser, protegendo os usuários contra ataques como o \textit{man-in-the-middle}\footnote{Um ataque \textit{man-in-the-middle} ocorre quando um invasor intercepta e possivelmente altera a comunicação entre duas partes que acreditam estar se comunicando diretamente. Isso pode permitir que o invasor capture informações sensíveis ou injete dados maliciosos na comunicação.\cite{wikipedia_man_in_the_middle}}.

Enquanto o \acrshort{http} tradicional opera normalmente na porta \acrshort{tcp} 80, o \acrshort{https} utiliza, por convenção, a porta 443. Atualmente, o uso do \acrshort{https} é fortemente recomendado — e até exigido por navegadores modernos — como padrão de segurança para qualquer aplicação web, contribuindo para a privacidade e confiança dos usuários \cite{wikipedia_http}.

\subsection{\acrshort{html}, \acrshort{css} e JavaScript}
\label{subsec:html-css-js}

O desenvolvimento \textit{frontend}\footnote{%
\textit{Frontend} refere-se à camada de apresentação de uma aplicação web, ou seja, a interface gráfica com a qual os usuários interagem diretamente. Envolve a implementação de elementos visuais e interativos, como menus, botões e formulários, utilizando tecnologias como \acrshort{html}, \acrshort{css} e JavaScript. Essa camada é fundamental para proporcionar uma experiência de usuário intuitiva e responsiva~\cite{aws_frontend_backend}.
} de aplicações web baseia-se em um conjunto de tecnologias fundamentais que operam em conjunto para fornecer estrutura, estilo e interatividade às páginas: \acrshort{html}, \acrshort{css} e JavaScript. Cada uma dessas linguagens desempenha um papel específico e complementar, sendo essenciais tanto em abordagens tradicionais quanto em técnicas modernas como o \acrshort{csr}.

\textbf{\acrfull{html}} é a linguagem de marcação padrão para a criação da estrutura de páginas web. Através de um conjunto de elementos (ou \textit{tags}), o \acrshort{html} organiza e define o conteúdo exibido ao usuário, como textos, imagens, links, formulários e tabelas. Além de estruturar visualmente o documento, o \acrshort{html} também confere semântica aos elementos, facilitando a indexação por motores de busca e promovendo acessibilidade para leitores de tela. Elementos como \texttt{<header>}, \texttt{<main>}, \texttt{<article>} e \texttt{<footer>} exemplificam essa função semântica~\cite{alura_htmlcssjs}.

\textbf{\acrfull{css}} é a linguagem responsável pela estilização das páginas web. Com o \acrshort{css}, define-se a aparência dos elementos estruturados no \acrshort{html}, controlando propriedades visuais como cores, fontes, espaçamentos, tamanhos e posicionamentos. O \acrshort{css} permite ainda a construção de layouts complexos e responsivos, adaptando o conteúdo para diferentes tamanhos de tela e dispositivos. A separação entre estrutura (\acrshort{html}) e estilo (\acrshort{css}) é um dos pilares das boas práticas em desenvolvimento web, promovendo manutenibilidade, reutilização e modularidade do código.

Entre os recursos modernos do \acrshort{css}, destacam-se os seletores avançados, variáveis \acrshort{css}, pseudo-classes, animações e as funcionalidades de \textit{Flexbox} e \textit{Grid}, que facilitam a criação de interfaces ricas e adaptáveis~\cite{herocode_diferencas}.

\textbf{JavaScript} é uma linguagem de programação interpretada, orientada a objetos e baseada em eventos, que permite adicionar comportamento e interatividade às páginas. Ao manipular a \textit{\acrfull{dom}}, o JavaScript possibilita respostas dinâmicas às ações do usuário, como cliques, envio de formulários, movimentações do mouse ou digitação. Essa linguagem é essencial para o carregamento assíncrono de dados por meio da técnica \textit{AJAX} (\textit{Asynchronous JavaScript and XML}), evitando recarregamentos completos da página e melhorando a experiência do usuário~\cite{alura_htmlcssjs}.

Além disso, JavaScript desempenha papel central na construção de aplicações baseadas em \acrshort{csr}, onde a renderização das páginas ocorre no lado do cliente. Com o avanço do ecossistema JavaScript, surgiram bibliotecas e frameworks robustos como React, Vue.js e Angular, que permitem o desenvolvimento de aplicações complexas com componentes reutilizáveis e gerenciamento eficiente de estado.

A interação entre essas três tecnologias pode ser compreendida por meio de uma analogia: o \acrshort{html} representa a estrutura de um corpo (esqueleto), o \acrshort{css} corresponde à sua aparência externa (pele, roupas, estilo), enquanto o JavaScript age como os músculos e o sistema nervoso, controlando os movimentos e respostas interativas da aplicação. A Figura~\ref{fig:html-css-js} ilustra essa relação.

\begin{figure}[H]
  \centering
  \caption{Analogia entre HTML, CSS e JavaScript e os componentes de um corpo humano.}
  \includegraphics[width=0.6\textwidth]{media/html_css_js_analogia.png}
  \legend{Fonte: \cite{herocode_diferencas}.}
  \label{fig:html-css-js}
\end{figure}

Portanto, o domínio dessas três tecnologias é indispensável para qualquer desenvolvedor web. Elas formam o alicerce sobre o qual se constroem interfaces acessíveis, performáticas e envolventes, sendo empregadas tanto em aplicações renderizadas no servidor (\acrshort{ssr}) quanto no cliente (\acrshort{csr}), com adaptações específicas conforme a abordagem escolhida.


\section{\english{Client-Side Rendering} (\acrshort{csr})}
\label{subsec:csr}

A \textbf{\english{\acrfull{csr}}} é uma técnica em que a geração da interface e do conteúdo final ocorre diretamente no navegador do usuário, utilizando JavaScript\footnote{JavaScript é uma linguagem de programação interpretada, amplamente utilizada para adicionar interatividade e dinamismo às páginas web. Ela permite implementar funcionalidades como atualizações em tempo real, animações, mapas interativos e validações de formulários, enriquecendo a experiência do usuário. Juntamente com HTML e CSS, JavaScript compõe as três principais tecnologias da World Wide Web. Além de ser executada no navegador (\acrshort{csr}), a linguagem também pode ser utilizada no \english{\acrfull{ssr}} por meio de ambientes como Node.js(um ambiente de execução JavaScript de código aberto e multiplataforma que permite executar código JavaScript no lado do servidor, utilizando uma arquitetura orientada a eventos e não bloqueante, ideal para aplicações escaláveis e em tempo real.\cite{nodejs2025} ), possibilitando o desenvolvimento de aplicações completas com uma única linguagem.\cite{js2025}}. Nessa abordagem, o servidor envia um arquivo \english{\acrfull{html}} mínimo, contendo apenas a estrutura básica da página e referências a arquivos de estilo e scripts.{\cite{atori2024}}

Segundo \citeonline{atori2024}, o processo de renderização no cliente segue as seguintes etapas:

\begin{enumerate}
    \item O servidor envia uma página \acrshort{html} em branco contendo apenas links para os arquivos \english{\acrfull{css}} e JavaScript.
    \item O navegador interpreta o \acrshort{html} e constrói a árvore do \english{\acrfull{dom}}
    \item Os arquivos de estilo (\acrshort{css}) e script (JavaScript) são baixados pelo navegador.
    \item A aplicação é renderizada dinamicamente pelo JavaScript, incluindo elementos visuais como texto, imagens e botões.
    \item O conteúdo da página é atualizado de forma interativa conforme o usuário interage com a aplicação.
\end{enumerate}

Esse modelo é comumente utilizado em aplicações \english{\acrfull{spa}}, nas quais o carregamento inicial é seguido por atualizações dinâmicas sem recarregamento da página. Frameworks como React\footnote{React é uma biblioteca JavaScript para construção de interfaces de usuário, desenvolvida pelo Facebook. \cite{react2025} }, Vue.js\footnote{Vue.js é um framework JavaScript progressivo utilizado para a criação de interfaces web interativas, focado na camada de visualização. \cite{vue2025} }, Angular\footnote{Angular é um framework para desenvolvimento de aplicações web, mantido pelo Google, que utiliza TypeScript como linguagem principal. \cite{angular2025} } e Svelte\footnote{Svelte é um framework JavaScript que realiza a compilação de componentes no momento do build, gerando código otimizado sem a necessidade de um virtual DOM. \cite{svelte2025} } são amplamente utilizados para implementar \acrshort{csr}.

\begin{figure}[h!]
    \centering
    \caption{Etapas do método de renderização no lado do cliente}
    \includegraphics[width=0.9\textwidth]{media/client_side_rendering.png}
    \legend{Fonte: \cite{atori2024} (adaptado)}
    \label{fig:client_side_rendering}
\end{figure}

A \autoref{fig:client_side_rendering} ilustra visualmente o fluxo completo da renderização no lado do cliente (\acrshort{csr}). O processo é iniciado quando o usuário acessa o site em questão. Em resposta, o servidor envia o arquivo \acrshort{html} básico, contendo apenas links para os arquivos de estilo \acrshort{css} e scripts JavaScript responsáveis por carregar e renderizar o conteúdo da aplicação.

Na sequência, o navegador interpreta esse \acrshort{html} e constrói a estrutura da página por meio da árvore \acrshort{dom}. No entanto, o conteúdo principal ainda não está visível. O navegador então precisa baixar os arquivos de estilo (\acrshort{css}) e os scripts JavaScript referenciados no documento inicial.

Com os scripts carregados, o navegador executa o código JavaScript, que normalmente utiliza bibliotecas ou frameworks como React ou Vue para gerar dinamicamente o conteúdo da aplicação. Somente após essa etapa o conteúdo completo do site é finalmente exibido ao usuário, quando o navegador conclui o processo de renderização e o site é carregado completamente.

\begin{lstlisting}[language=html, capt.\ion={Exemplo de HTML mínimo em aplicação Angular com CSR}, label={lst:angular_html}]
<!DOCTYPE html>
<html lang="en">
<head>
  <meta charset="utf-8">
  <title>CryptoWebsite</title>
  <base href="/">
  <meta name="viewport" content="width=device-width, initial-scale=1">
  <link rel="icon" type="image/x-icon" href="favicon.ico">
  <style>*,*:before,*:after{margin:0;padding:0;box-sizing:border-box;
    font-family:Inter,sans-serif}html{font-size:62.5%}</style>
  <link rel="stylesheet" href="styles.9d4c7581c7242.css">
</head>
<body>
  <app-root></app-root>
  <script src="runtime.6170988ad52a05db.js" type="module"></script>
  <script src="polyfills.574970d5ec4bdb97.js" type="module"></script>
  <script src="main.202d37bb6740400e.js" type="module"></script>
</body>
</html>
\end{lstlisting}

Esse padrão é típico de aplicações \acrshort{spa}, onde todo o conteúdo é inserido dinamicamente a partir da execução dos arquivos JavaScript. O elemento \texttt{<app-root>} funciona como ponto de entrada da aplicação, sendo substituído no navegador pelos componentes definidos no framework Angular. {\cite{atori2024}}


\section{\english{Server-Side Rendering} (\acrshort{ssr})}
\label{subsec:ssr}

A \textbf{\english{\acrfull{ssr}}} é uma abordagem em que a geração do conteúdo e da interface ocorre integralmente no servidor antes de ser enviada ao navegador do cliente. Ou seja, o servidor processa a lógica da aplicação, obtém dados necessários (por exemplo, em bancos de dados ou \emph{APIs}) e retorna ao cliente um arquivo \english{\acrshort{html}} já renderizado. Dessa forma, o navegador exibe imediatamente a página completa, sem precisar executar \emph{scripts} para montar o conteúdo inicial \cite{atori2024}. 

Segundo \citeonline{atori2024}, o processo típico de renderização no lado do servidor pode ser descrito em quatro etapas principais:

\begin{enumerate}
    \item O servidor recebe uma requisição para uma página e recupera os dados necessários para compor seu conteúdo (por exemplo, produtos de uma base de dados ou artigos de um blog).
    \item O servidor insere esses dados em um \emph{template} \acrshort{html}, gerando a estrutura final da página.
    \item Em seguida, o servidor aplica estilos e finaliza a renderização, resultando em um documento \acrshort{html} completamente montado.
    \item Por fim, esse documento \acrshort{html} é enviado ao navegador do usuário, exibindo a página prontamente, sem a necessidade de executar \emph{JavaScript} durante o carregamento inicial.
\end{enumerate}

Nesse modelo, a fase de hydration\footnote{Hydration é uma etapa essencial no \acrshort{ssr}, em que o JavaScript torna interativo o conteúdo HTML previamente renderizado no servidor.} ocorre após o carregamento inicial da página. costuma ocorrer após a entrega do conteúdo estático. Significa que, assim que o arquivo \acrshort{html} é carregado e mostrado ao usuário, o \emph{JavaScript} do lado do cliente assume o controle para tratar as interações e atualizações dinâmicas subsequentes. Dessa forma, o \acrshort{ssr} beneficia tanto o primeiro acesso (tornando o conteúdo visível rapidamente) quanto o \acrshort{seo}, por exibir ao rastreador dos mecanismos de busca um código \acrshort{html} completo. \cite{atori2024}.

\begin{figure}[H]
  \centering
  \caption{Etapas do método de renderização no lado do servidor}
  \includegraphics[width=0.8\textwidth]{media/server_side_rendering.jpeg}
  \legend{Fonte: \cite{atori2024} (adaptado)}
  \label{fig:server_side_rendering}
\end{figure}

A \autoref{fig:server_side_rendering} ilustra o fluxo de uma aplicação \acrshort{ssr}. Ao receber a requisição, o servidor gera a página completa em \acrshort{html} e a envia ao cliente. Essa estratégia costuma ser vantajosa em cenários onde o carregamento inicial rápido e a indexação por motores de busca são prioridades, como em sites de e-commerce e páginas de \emph{landing}, permitindo que o usuário visualize o conteúdo de forma imediata. 


\emph{Meta-\-frameworks}\footnote{
  Os meta-frameworks são sistemas de desenvolvimento web que operam em um nível superior a frameworks tradicionais, como React, Vue ou Svelte. Seu principal objetivo é agregar e organizar múltiplas funcionalidades comuns no desenvolvimento de aplicações, oferecendo uma estrutura mais completa e opinativa. \cite{benHolmes}
} como \emph{Next.js}\footnote{Next.js é um framework de código aberto para React que facilita a criação de aplicações web com renderização no lado do servidor (\acrshort{ssr}) e geração de sites estáticos. Fornece recursos como roteamento baseado em arquivos, pré-renderização e suporte a APIs. \cite{nextjs2024}} , \emph{Nuxt.js}\footnote{Nuxt.js é um framework baseado em Vue.js que simplifica a criação de aplicações universais, oferecendo renderização no lado do servidor, geração de sites estáticos e uma arquitetura modular que facilita a escalabilidade. \cite{nuxtjs2024}} , \emph{SvelteKit}\footnote{SvelteKit é um framework para Svelte que oferece uma experiência de desenvolvimento simplificada, permitindo a criação de aplicações com renderização no lado do servidor e no cliente, além de otimizações de desempenho. \cite{sveltekit2024}} , \emph{Angular Universal}\footnote{Angular Universal é a solução oficial de renderização no lado do servidor para aplicações Angular, permitindo a renderização de páginas no servidor para melhorar o desempenho e a indexação por mecanismos de busca. \cite{angularuniversal2024}} , \emph{Remix}\footnote{Remix é um framework full-stack para React que foca na experiência do desenvolvedor e no desempenho, oferecendo renderização no lado do servidor e um modelo de dados baseado em carregadores e ações. \cite{remix2024}} , \emph{Astro}\footnote{Astro é um framework moderno que permite a construção de sites rápidos, carregando apenas o JavaScript necessário e permitindo o uso de componentes de diferentes frameworks como React, Svelte e Vue. \cite{astro2024}} e \emph{Qwik}\footnote{Qwik é um framework JavaScript que introduz o conceito de aplicações web "resumíveis", visando tempos de carregamento instantâneos e desempenho aprimorado, utilizando renderização no lado do servidor e aprimoramentos progressivos. \cite{qwik2024}} são amplamente utilizados para construir aplicações com suporte a \acrshort{ssr}, simplificando a configuração e fornecendo recursos prontos para lidar com roteamento, recuperação de dados e \english{hidration}.

No \autoref{lst:nextjs_html}, pode-se observar que o arquivo \acrshort{html} já contém todo o \emph{markup} necessário para exibir o conteúdo da página. Assim que o navegador recebe esse arquivo, o usuário já visualiza o cabeçalho, o texto e o layout definidos. Posteriormente, o \emph{JavaScript} baixado (por exemplo, \texttt{main.js}) pode entrar em ação para lidar com eventos, rotas adicionais e atualizações dinâmicas, caso o desenvolvedor deseje funcionalidades mais interativas.

Por fim, aplicações \acrshort{ssr} tendem a apresentar melhor performance em termos de \emph{time-to-first-byte}\footnote{O \emph{time-to-first-byte} (TTFB) é uma métrica que mede o tempo decorrido entre o envio de uma solicitação HTTP pelo cliente e o recebimento do primeiro byte da resposta do servidor. Um TTFB menor indica maior rapidez na resposta do servidor, impactando diretamente na velocidade de carregamento da página e na experiência do usuário. \cite{ttfb-craig}} e de \emph{indexabilidade}\footnote{A \emph{indexabilidade} refere-se à capacidade dos motores de busca de rastrear e indexar o conteúdo de uma página web. Aplicações SSR, ao fornecerem conteúdo totalmente renderizado no servidor, facilitam a indexação eficiente pelos motores de busca, melhorando a visibilidade nos resultados de pesquisa. \cite{ttfb-oskay}} por motores de busca, ao mesmo tempo em que podem demandar maior carga de processamento no servidor. A escolha por \acrshort{ssr} ou não, portanto, depende do perfil da aplicação e das prioridades do projeto, considerando fatores como volume de tráfego, necessidade de interatividade e requisitos de otimização de conteúdo.

\begin{lstlisting}[language=html, caption={Exemplo de HTML mínimo em aplicação Next.js com SSR}, label={lst:nextjs_html}]
<!DOCTYPE html>
<html lang="en">
<head>
  <meta charset="utf-8">
  <title>My SSR App</title>
  <meta name="viewport" content="width=device-width, initial-scale=1">
  <style>
    /* Exemplo simples de estilo inline */
    body {
      margin: 0;
      font-family: Arial, sans-serif;
      background: #f6f6f6;
    }
    h1 { color: #333; }
  </style>
</head>
<body>
  <!-- Conteúdo já processado e inserido no servidor -->
  <div id="__next">
    <header>
      <h1>Olá, mundo!</h1>
    </header>
    <main>
      <p>Este conteúdo foi renderizado no servidor usando Next.js.</p>
    </main>
  </div>
  <!-- Scripts do Next.js para interação no cliente -->
  <script src="/_next/static/chunks/main.js" defer></script>
</body>
</html>
\end{lstlisting}


% \subsection{Modelos de Arquitetura Web}
% \label{subsec:modelos-arq-web}
% Os modelos arquiteturais variam conforme os requisitos de escalabilidade, manutenção e desempenho:

% \begin{itemize}
%     \item \textbf{Arquitetura Monolítica}: Um único projeto concentra \textit{frontend} e \textit{backend}, frequentemente usando \acrshort{ssr}. Possui inicialização simples, mas pode tornar-se complexo de manter e escalar.
%     \item \textbf{Microserviços}: Divide a aplicação em múltiplos serviços independentes. Cada serviço pode escolher a melhor abordagem de renderização (SSR ou CSR), facilitando a escalabilidade seletiva.
%     \item \textbf{Serverless}: As funções são executadas sob demanda em plataformas de nuvem, onde a renderização pode ocorrer tanto no servidor (funções que retornam HTML) quanto no cliente (ao entregar apenas APIs).
% \end{itemize}

% \subsection{Ferramentas e \textit{Frameworks}}
% \label{subsec:ferramentas-frameworks}
% O ecossistema de desenvolvimento web oferece diversas ferramentas que simplificam \acrshort{ssr} e \acrshort{csr}:

% \begin{itemize}
%     \item \textbf{\acrshort{ssr}}: \textit{Next.js} (React), \textit{Nuxt.js} (Vue), \textit{SvelteKit} (Svelte), entre outros.
%     \item \textbf{\acrshort{csr}}: React, Vue.js, Angular e muitas bibliotecas voltadas para \textit{Single Page Applications} (SPA).
% \end{itemize}


% \section{Infraestrutura de Serviço Web}
% \label{sec:infraestrutura-web}

% A decisão por \acrshort{ssr} ou \acrshort{csr} influencia diretamente a infraestrutura necessária:

% \begin{itemize}
%     \item \textbf{Servidores e Processamento}: Em \acrshort{ssr}, o servidor gera páginas dinamicamente, aumentando a carga de CPU. Já em \acrshort{csr}, o servidor atua mais como um provedor de arquivos estáticos e APIs.
%     \item \textbf{\english{Content Delivery Networks} (CDNs)}: Tanto para SSR quanto para CSR, uma CDN pode melhorar a distribuição de arquivos estáticos (HTML, CSS, JavaScript, imagens) e reduzir a latência.
%     \item \textbf{Escalabilidade}: Aplicações com alto número de requisições precisam de estratégias adequadas para lidar com picos de acesso. Em \acrshort{ssr}, muitas requisições simultâneas podem sobrecarregar o servidor; em \acrshort{csr}, o foco está em serviços de dados e na entrega eficiente de arquivos iniciais.
% \end{itemize}

% \textbf{Segurança} também se faz presente em ambas as abordagens. Boas práticas incluem:
% \begin{itemize}
%     \item Uso de \textbf{HTTPS} para proteger a comunicação.
%     \item Implementação de \textbf{CORS} (Cross-Origin Resource Sharing) quando necessário.
%     \item Tratamento de \textbf{tokens de sessão/autenticação} com cuidado para evitar vazamento de dados.
% \end{itemize}

% ---

\section{Experiência do Usuário}
\label{sec:ux}
A \acrfull{ux} é um aspecto crítico no desenvolvimento de aplicações web, influenciando diretamente a satisfação e a eficácia da interação do usuário com o sistema \cite{atori2023}. Para alcançar uma \acrshort{ux} satisfatória, a escolha entre \acrshort{csr} e \acrshort{ssr} deve considerar fatores como: {\acrshort{seo}}, velocidade de carregamento, interatividade e acessibilidade.
Conforme \citeonline{atori2024}, a \acrshort{ux} vai além da interface gráfica, englobando toda a jornada do usuário desde a navegação até a conclusão de tarefas. 

\subsection{\english{Search Engine Optimization} (SEO)}
\label{sec:seo}

O \acrfull{seo} consiste em um conjunto integrado de práticas de otimização, tanto no aspecto técnico quanto no de conteúdo, com três objetivos principais: maximizar a visibilidade orgânica nos mecanismos de busca, posicionar estrategicamente páginas-chave e garantir uma experiência de usuário qualificada durante o processo de busca. Essas práticas são essenciais para garantir que o conteúdo de um site seja facilmente encontrado e indexado pelos motores de busca, aumentando a probabilidade de atrair visitantes qualificados. Entre os fatores mais conhecidos, destaca-se a velocidade de carregamento da página, que impacta diretamente a experiência do usuário e a classificação nos resultados de busca \cite{conor2022}.
\begin{figure}[H]
    \centering
    \caption{Tempo de Rastreamento e Posicionamento da Página}
    \includegraphics[width=0.6\textwidth]{media/rank_crawl_and_page_rank.png}
    \legend{Fonte: \cite{webPerformance}(adaptado)}
    \label{fig:rank_crawl_and_page_rank}
\end{figure}

A \autoref{fig:rank_crawl_and_page_rank} ilustra a relação entre o tempo de rastreamento e o posicionamento da página. O tempo de rastreamento refere-se ao tempo que os mecanismos de busca levam para acessar e indexar uma página. Quanto mais rápido o tempo de rastreamento, maior a probabilidade de a página ser indexada rapidamente e, consequentemente, melhor seu posicionamento nos resultados de busca. Isso destaca a importância de otimizar o desempenho do site para garantir uma boa classificação nos motores de busca.



\subsection{\english{Velocidade de carregamento} }
\label{sec:velocidade da página}
A velocidade de carregamento da página refere-se ao tempo que uma página da web leva para carregar completamente, desde a solicitação do usuário até que o conteúdo esteja totalmente visível e interativo no navegador. Esse tempo pode ser impactado por diversos fatores, como o tamanho dos arquivos da página, a complexidade do conteúdo, a qualidade da conexão com a internet e o desempenho do servidor.

A velocidade de carregamento é um dos principais fatores que afetam a \acrshort{ux} e o \acrshort{seo}. Páginas que carregam rapidamente tendem a ter taxas de rejeição mais baixas e melhor desempenho em termos de conversão. O \acrshort{seo} também considera a velocidade de carregamento como um fator importante para determinar a classificação nos resultados de busca \cite{conor2022}.

Segundo o \citeonline{google}, aumentar a velocidade de carregamento é fundamental: quanto mais rápido o site carregar, melhor será a experiência do usuário. Caso contrário, o usuário passa menos tempo na página, o que pode aumentar a taxa de rejeição.

\subsection{Interatividade}
\label{subsec:interatividade}

A interatividade é um fator decisivo na experiência do usuário em aplicações web modernas, pois determina a forma como os usuários percebem a continuidade e a capacidade de resposta durante a navegação. Nas aplicações que utilizam \acrshort{csr}, o código JavaScript é executado diretamente no navegador, permitindo respostas imediatas a interações como cliques, preenchimento de formulários ou navegação entre páginas internas. Essa abordagem possibilita transições de página mais suaves e experiências semelhantes às de aplicativos nativos, sem a necessidade de recarregamentos completos \cite{pixelfree2023}.

Segundo \citeonline{atori2024}, a renderização no lado do cliente favorece experiências altamente dinâmicas, oferecendo um nível elevado de controle sobre os elementos da interface. Em contrapartida, o \acrshort{ssr}, embora proporcione carregamento inicial mais rápido e visibilidade imediata do conteúdo, apresenta limitações em termos de interatividade. Alterações na interface em aplicações \acrshort{ssr} geralmente demandam comunicações adicionais com o servidor, o que pode comprometer a continuidade da experiência do usuário \cite{atori2024, splunk2023}.

Para mitigar essas limitações, abordagens híbridas têm sido amplamente adotadas. Nelas, o conteúdo é inicialmente renderizado no servidor e, posteriormente, reativado no cliente com JavaScript, em uma estratégia conhecida como \emph{hydration} \cite{splunk2023}. Essa técnica busca aliar os benefícios de desempenho e \acrshort{seo} do \acrshort{ssr} com a interatividade aprimorada do \acrshort{csr}.


\subsection{Acessibilidade}
\label{subsec:acessibilidade}

A acessibilidade em aplicações web refere-se à capacidade de tornar conteúdos e funcionalidades utilizáveis por pessoas com deficiência, como visual, auditiva, motora ou cognitiva. É um princípio essencial para garantir a equidade no acesso à informação e à interação digital. De acordo com \cite{pixelfree2023access}, acessibilidade diz respeito a assegurar que todos, independentemente de suas habilidades, possam acessar e interagir com o conteúdo da web. Para pessoas com deficiência, isso pode significar o uso de leitores de tela, navegação por teclado ou a dependência de outras tecnologias assistivas.

As abordagens de renderização, como \acrshort{csr} e \acrshort{ssr}, impactam diretamente a acessibilidade, especialmente na compatibilidade com essas tecnologias. Em aplicações que utilizam \acrshort{csr}, o conteúdo geralmente é carregado de forma assíncrona após a execução do JavaScript, o que pode dificultar a leitura imediata por leitores de tela que dependem de uma estrutura HTML previamente carregada para interpretar a página corretamente \cite{pixelfree2023access}. Já no \acrshort{ssr}, o conteúdo é entregue completamente no carregamento inicial, facilitando a interpretação por essas ferramentas e proporcionando uma experiência mais estável para usuários com deficiência visual \cite{atori2024}.

Além disso, em contextos com atualizações dinâmicas de conteúdo — como ocorre em SPAs com \acrshort{csr} — é necessário adotar práticas específicas para garantir a acessibilidade, como gerenciamento de foco, uso de alertas ARIA e atualização de leitores de tela após mudanças no DOM. Essas medidas são fundamentais para que as mudanças de visualização sejam percebidas corretamente por tecnologias assistivas, uma vez que alterações no DOM nem sempre são reconhecidas automaticamente por leitores de tela. O envio de foco a elementos interativos ou o uso de regiões ARIA ao vivo são técnicas recomendadas para anunciar mudanças de estado ao usuário \cite{sutton2018}.

Assim, embora o \acrshort{ssr} ofereça uma base naturalmente mais acessível, ambas as abordagens podem ser igualmente inclusivas quando aplicadas com atenção às diretrizes e boas práticas de acessibilidade.


