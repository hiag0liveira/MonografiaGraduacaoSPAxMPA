\chapter{Fundamentação Teórica}
\label{cap:fundamentacao}

Este capítulo apresenta os conceitos de \english{\acrfull{csr}} e \english{\acrfull{ssr}}, abordando os princípios fundamentais do desenvolvimento web relacionados à renderização de conteúdo. Também são discutidos aspectos como \acrshort{seo}, desempenho, infraestrutura de serviços e impacto na experiência do usuário, estabelecendo a base teórica para o estudo de caso desenvolvido neste trabalho.

\section{Fundamentos de Desenvolvimento Web}
\label{sec:fundamentos-devweb}
Para entender como as abordagens \acrshort{ssr} e \acrshort{csr} se inserem no cenário de desenvolvimento web, é fundamental revisar protocolos, modelos de arquitetura e ferramentas.
Os fundamentos de desenvolvimento web englobam os princípios, tecnologias e práticas essenciais para a criação e manutenção de aplicações acessíveis via internet. 

O desenvolvimento web baseia-se na arquitetura cliente-servidor, onde o cliente (geralmente um navegador) solicita recursos ao servidor, que processa essas requisições e retorna os dados necessários. Essa interação é mediada por protocolos como o 
\english{\acrfull{http}} , que define as regras de comunicação entre cliente e servidor.

As tecnologias fundamentais incluem \english{\acrfull{html}} para estruturação do conteúdo, \english{\acrfull{css}} para estilização e JavaScript para interatividade. Essas linguagens permitem a construção de interfaces dinâmicas e responsivas. Além disso, o desenvolvimento web envolve práticas como controle de versão, testes automatizados e integração contínua, que garantem a qualidade e a escalabilidade das aplicações \cite{fundamentosDevWeb}. 

\subsection{Arquitetura Cliente-Servidor}
\label{subsec:Arquitetura Cliente-Servidor}

A arquitetura cliente-servidor é um modelo amplamente adotado no desenvolvimento de aplicações web, caracterizado pela separação entre dois componentes principais: o \textbf{cliente}, responsável pela interface com o usuário, e o \textbf{servidor}, que processa solicitações e fornece os recursos necessários \cite{clienteServidorControlNet}.

Nesse modelo, os clientes — como navegadores em diferentes dispositivos — enviam requisições através da internet, enquanto os servidores respondem disponibilizando dados, arquivos e serviços. Essa divisão de responsabilidades favorece a escalabilidade, facilita a manutenção e permite que cliente e servidor operem em plataformas distintas~\cite{fundamentosDevWeb}.

A Figura~\ref{fig:cliente-servidor} ilustra, de forma simplificada, esse fluxo de comunicação entre cliente e servidor.

\begin{figure}[H]
  \centering
  \caption{Comunicação entre cliente e servidor.}
  \includegraphics[width=0.8\textwidth]{media/cliente_servidor.jpeg}
  \legend{Fonte: \cite{fundamentosDevWeb} }
  \label{fig:cliente-servidor}
\end{figure}


A arquitetura cliente-servidor apresenta características que contribuem para sua ampla adoção em aplicações web. Entre elas, destaca-se a \textbf{distribuição de responsabilidades}, onde o servidor gerencia dados e processos mais complexos, enquanto o cliente lida com a interface e a interação com o usuário. 

Outro ponto relevante é a \textbf{independência entre plataformas}, possibilitada pelo uso de protocolos padronizados, o que permite a comunicação entre diferentes dispositivos e sistemas operacionais. Além disso, esse modelo favorece a \textbf{facilidade de manutenção}, já que atualizações podem ser feitas no servidor sem necessidade de intervenção nos dispositivos dos usuários.

Na web, essa arquitetura é implementada por padrão: navegadores atuam como clientes, enviando requisições \acrshort{http} que são processadas por servidores, os quais respondem com páginas e recursos solicitados~\cite{fundamentosDevWeb}.


\subsection{Protocolo \acrshort{http}}
\label{subsec:http}
O \textbf{Protocolo de Transferência de Hipertexto} (\acrshort{http}) é a base da comunicação na World Wide Web, definindo como clientes (navegadores) e servidores trocam informações. Ele especifica a estrutura das requisições e respostas, permitindo a recuperação de recursos como documentos \acrshort(html), imagens e vídeos \cite{mdn_http}.


 O \textbf{Funcionamento do \acrshort{http}}opera no modelo cliente-servidor, onde o cliente inicia uma requisição e o servidor responde com os recursos solicitados ou mensagens de erro, se aplicável. Cada interação consiste em uma mensagem de requisição do cliente e uma mensagem de resposta do servidor. As mensagens \acrshort{http} são compostas por:

\begin{itemize}
    \item \textbf{Linha de início:} Indica o método \acrshort{http} (como \texttt{GET} ou \texttt{POST}) e o caminho do recurso.
    \item \textbf{Cabeçalhos:} Fornecem informações adicionais sobre a requisição ou resposta, como tipo de conteúdo e codificação.
    \item \textbf{Corpo:} Contém os dados enviados ou recebidos, sendo opcional dependendo do método utilizado.
\end{itemize}

\begin{flushright}
    \cite{mdn_http}
\end{flushright}


\textbf{Métodos \acrshort{http}} são operações definidas pelo protocolo que especificam a ação a ser realizada em um recurso. Os métodos mais comuns incluem:

\begin{itemize}
    \item \textbf{GET:} Solicita a representação de um recurso específico. Requisições GET devem ser utilizadas apenas para recuperar dados.
    \item \textbf{POST:} Envia dados ao servidor para processamento, como o envio de formulários.
    \item \textbf{PUT:} Atualiza um recurso existente ou cria um novo se não existir.
    \item \textbf{DELETE:} Remove um recurso específico.
    \item \textbf{HEAD:} Similar ao GET, mas solicita apenas os cabeçalhos da resposta, sem o corpo.
\end{itemize}
Cada método possui uma finalidade específica e deve ser utilizado conforme a necessidade da aplicação \cite{wikipedia_http}.


\textbf{Códigos de Status \acrshort{http}} são códigos de três dígitos que indicam o resultado de uma requisição feita pelo cliente ao servidor. Eles são agrupados em cinco classes principais:

\begin{itemize}
    \item \textbf{1xx (Informativo):} Indica que a requisição foi recebida e o processo continua.
    \item \textbf{2xx (Sucesso):} Indica que a requisição foi bem-sucedida. Exemplo: 200 OK.
    \item \textbf{3xx (Redirecionamento):} Indica que é necessário tomar medidas adicionais para completar a requisição. Exemplo: 301 Moved Permanently.
    \item \textbf{4xx (Erro do Cliente):} Indica que houve um erro na requisição do cliente. Exemplo: 404 Not Found.
    \item \textbf{5xx (Erro do Servidor):} Indica que o servidor falhou ao processar uma requisição válida. Exemplo: 500 Internal Server Error.
\end{itemize}

Esses códigos auxiliam na identificação e resolução de problemas durante a comunicação \acrshort{http} \cite{mdn_http}.


\textbf{Evolução do \acrshort{http}} refere-se às revisões progressivas do protocolo com o objetivo de aprimorar sua eficiência, segurança e desempenho ao longo do tempo. As principais versões são:

\begin{itemize}
    \item \textbf{HTTP/1.0:} Primeira versão oficial do protocolo, em que cada requisição exigia uma nova conexão com o servidor.
    \item \textbf{HTTP/1.1:} Introduziu conexões persistentes, permitindo múltiplas requisições por conexão. Trouxe também melhorias no controle de cache e suporte a novos métodos.
    \item \textbf{HTTP/2:} Implementou multiplexação, compressão de cabeçalhos e priorização de fluxos, resultando em uma transferência de dados mais rápida e eficiente.
    \item \textbf{HTTP/3:} Baseado no protocolo \acrshort{quic}, substitui o \acrshort{tcp} pelo \acrshort{udp}, oferecendo conexões mais rápidas e seguras, com menor latência e melhor desempenho em redes instáveis.
\end{itemize}

Essas atualizações refletem a evolução das necessidades da web e a busca por protocolos mais robustos e otimizados \cite{cloudflare_http}.

\textbf{\acrshort{http} e \acrshort{https}} representam protocolos utilizados para comunicação na web, com a principal distinção centrada na segurança da transmissão dos dados.

O \textbf{\english{\acrfull{https}}} é uma extensão do \acrshort{http} que adiciona uma camada de proteção por meio do protocolo \english{\acrfull{tls}} ou, anteriormente, \english{\acrfull{ssl}}. Essa camada de segurança garante a confidencialidade, integridade e autenticidade dos dados trafegados entre cliente e servidor. 

A criptografia utilizada impede que terceiros acessem ou modifiquem as informações transmitidas, o que é fundamental em transações sensíveis, como cadastros, pagamentos e autenticações. Além disso, o uso de \textit{certificados digitais} garante que o site visitado é realmente aquele que afirma ser, protegendo os usuários contra ataques como o \textit{man-in-the-middle}\footnote{Um ataque \textit{man-in-the-middle} ocorre quando um invasor intercepta e possivelmente altera a comunicação entre duas partes que acreditam estar se comunicando diretamente. Isso pode permitir que o invasor capture informações sensíveis ou injete dados maliciosos na comunicação.\cite{wikipedia_man_in_the_middle}}.

Enquanto o \acrshort{http} tradicional opera normalmente na porta \acrshort{tcp} 80, o \acrshort{https} utiliza, por convenção, a porta 443. Atualmente, o uso do \acrshort{https} é fortemente recomendado — e até exigido por navegadores modernos — como padrão de segurança para qualquer aplicação web, contribuindo para a privacidade e confiança dos usuários \cite{wikipedia_http}.

\subsection{\acrshort{html}, \acrshort{css} e JavaScript}
\label{subsec:html-css-js}


O desenvolvimento frontend, conforme definido por \citeonline{aws_frontend_backend}, refere-se à camada de apresentação de uma aplicação web — a interface gráfica com a qual os usuários interagem diretamente, composta por menus, botões, formulários e outros elementos visuais. Essa camada baseia-se em um conjunto de tecnologias fundamentais que operam em conjunto para fornecer estrutura, estilo e interatividade às páginas: \acrshort{html}, \acrshort{css} e JavaScript. Cada uma dessas linguagens desempenha um papel específico e complementar, sendo essenciais tanto em abordagens tradicionais quanto em técnicas modernas como o \acrshort{csr}.


\textbf{\acrfull{html}} é a linguagem de marcação padrão para a criação da estrutura de páginas web. Através de um conjunto de elementos (ou \textit{tags}), o \acrshort{html} organiza e define o conteúdo exibido ao usuário, como textos, imagens, links, formulários e tabelas. Além de estruturar visualmente o documento, o \acrshort{html} também confere semântica aos elementos, facilitando a indexação por motores de busca e promovendo acessibilidade para leitores de tela. Elementos como \texttt{<header>}, \texttt{<main>}, \texttt{<article>} e \texttt{<footer>} exemplificam essa função semântica~\cite{alura_htmlcssjs}.

\textbf{\acrfull{css}} é a linguagem responsável pela estilização das páginas web. Com o \acrshort{css}, define-se a aparência dos elementos estruturados no \acrshort{html}, controlando propriedades visuais como cores, fontes, espaçamentos, tamanhos e posicionamentos. O \acrshort{css} permite ainda a construção de layouts complexos e responsivos, adaptando o conteúdo para diferentes tamanhos de tela e dispositivos. A separação entre estrutura (\acrshort{html}) e estilo (\acrshort{css}) é um dos pilares das boas práticas em desenvolvimento web, promovendo manutenibilidade, reutilização e modularidade do código.

Entre os recursos modernos do \acrshort{css}, destacam-se os seletores avançados, variáveis \acrshort{css}, pseudo-classes, animações e as funcionalidades de \textit{Flexbox} e \textit{Grid}, que facilitam a criação de interfaces ricas e adaptáveis~\cite{herocode_diferencas}.

\textbf{JavaScript} é uma linguagem de programação interpretada, orientada a objetos e baseada em eventos, amplamente utilizada para adicionar interatividade e dinamismo às páginas web. Por meio da manipulação da \textit{\acrfull{dom}}, permite implementar funcionalidades como respostas a cliques, envio de formulários, movimentações do mouse, digitação, animações, validações e atualizações em tempo real, enriquecendo significativamente a experiência do usuário~\cite{alura_htmlcssjs}. Além disso, possibilita o carregamento assíncrono de dados com a técnica \textit{AJAX} (\textit{Asynchronous JavaScript and XML}), evitando recarregamentos completos da página.

JavaScript é uma das três principais tecnologias da World Wide Web, juntamente com \acrshort{html} e \acrshort{css}, sendo essencial tanto em abordagens tradicionais quanto modernas. Nas aplicações baseadas em \acrshort{csr}, essa linguagem tem papel central, pois a renderização das páginas ocorre diretamente no navegador do usuário. Com a evolução do ecossistema JavaScript, surgiram bibliotecas e frameworks robustos como a biblioteca React e os frameworks Vue.js e Angular que facilitam o desenvolvimento de aplicações complexas com componentes reutilizáveis e gerenciamento eficiente de estado.

Adicionalmente, o JavaScript também pode ser executado no lado do servidor (\acrshort{ssr}) por meio de ambientes como o Node.js — uma plataforma de código aberto e multiplataforma baseada em eventos e não bloqueante, ideal para aplicações escaláveis e em tempo real~\cite{nodejs2025, js2025}. Isso permite o desenvolvimento de aplicações completas utilizando uma única linguagem em ambas as camadas, cliente e servidor.

A interação entre essas três tecnologias pode ser compreendida por meio de uma analogia: o \acrshort{html} representa a estrutura de um corpo (esqueleto), o \acrshort{css} corresponde à sua aparência externa (pele, roupas, estilo), enquanto o JavaScript age como os músculos e o sistema nervoso, controlando os movimentos e respostas interativas da aplicação. A Figura~\ref{fig:html-css-js} ilustra essa relação.

\begin{figure}[H]
  \centering
  \caption{Analogia entre HTML, CSS e JavaScript e os componentes de um corpo humano.}
  \includegraphics[width=0.6\textwidth]{media/html_css_js_analogia.png}
  \legend{Fonte: \cite{herocode_diferencas}.}
  \label{fig:html-css-js}
\end{figure}

Portanto, o domínio dessas três tecnologias é indispensável para qualquer desenvolvedor web. Elas formam o alicerce sobre o qual se constroem interfaces acessíveis, performáticas e envolventes, sendo empregadas tanto em aplicações renderizadas no servidor (\acrshort{ssr}) quanto no cliente (\acrshort{csr}), com adaptações específicas conforme a abordagem escolhida.


\section{\english{Client-Side Rendering} (\acrshort{csr})}
\label{subsec:csr}

A \textbf{\english{\acrfull{csr}}} é uma técnica em que a geração da interface e do conteúdo final ocorre diretamente no navegador do usuário, utilizando JavaScript. Nessa abordagem, o servidor envia um arquivo \english{\acrfull{html}} mínimo, contendo apenas a estrutura básica da página e referências a arquivos de estilo e scripts.{\cite{atori2024}}

Segundo \citeonline{atori2024}, o processo de renderização no cliente segue as seguintes etapas:

\begin{enumerate}
    \item O servidor envia uma página \acrshort{html} em branco contendo apenas links para os arquivos \english{\acrfull{css}} e JavaScript.
    \item O navegador interpreta o \acrshort{html} e constrói a árvore do \english{\acrfull{dom}}
    \item Os arquivos de estilo (\acrshort{css}) e script (JavaScript) são baixados pelo navegador.
    \item A aplicação é renderizada dinamicamente pelo JavaScript, incluindo elementos visuais como texto, imagens e botões.
    \item O conteúdo da página é atualizado de forma interativa conforme o usuário interage com a aplicação.
\end{enumerate}

Esse modelo é comumente utilizado em aplicações \english{\acrfull{spa}}, nas quais o carregamento inicial é seguido por atualizações dinâmicas sem recarregamento da página. Ferramentas como a biblioteca React, e frameworks como Vue.js, Angular e Svelte são amplamente utilizadas para implementar \acrshort{csr}, permitindo o desenvolvimento de interfaces dinâmicas, interativas e responsivas.


A renderização no lado do cliente (\acrshort{csr}) é especialmente vantajosa em aplicações que exigem alta interatividade e atualizações frequentes de conteúdo, como redes sociais, plataformas de streaming e jogos online. No entanto, essa abordagem pode apresentar desvantagens em termos de desempenho inicial e \acrshort{seo}, uma vez que o conteúdo só é exibido após a execução do JavaScript, o que pode impactar negativamente a indexação por motores de busca e a experiência do usuário em conexões lentas \cite{atori2024}.

\begin{figure}[h!]
    \centering
    \caption{Etapas do método de renderização no lado do cliente}
    \includegraphics[width=0.8\textwidth]{media/client_side_rendering.png}
    \legend{Fonte: \cite{atori2024} (adaptado)}
    \label{fig:client_side_rendering}
\end{figure}


A \autoref{fig:client_side_rendering} ilustra visualmente o fluxo completo da renderização no lado do cliente (\acrshort{csr}). O processo é iniciado quando o usuário acessa o site em questão. Em resposta, o servidor envia o arquivo \acrshort{html} básico, contendo apenas links para os arquivos de estilo \acrshort{css} e scripts JavaScript responsáveis por carregar e renderizar o conteúdo da aplicação.

Na sequência, o navegador interpreta esse \acrshort{html} e constrói a estrutura da página por meio da árvore \acrshort{dom}. No entanto, o conteúdo principal ainda não está visível. O navegador então precisa baixar os arquivos de estilo (\acrshort{css}) e os scripts JavaScript referenciados no documento inicial.

Com os scripts carregados, o navegador executa o código JavaScript, que normalmente utiliza bibliotecas ou frameworks como React ou Vue para gerar dinamicamente o conteúdo da aplicação. Somente após essa etapa o conteúdo completo do site é finalmente exibido ao usuário, quando o navegador conclui o processo de renderização e o site é carregado completamente.

\begin{codigo}[H]
  \begin{lstlisting}[language=html]
<!DOCTYPE html>
<html lang="en">
<head>
  <meta charset="utf-8">
  <title>CryptoWebsite</title>
  <base href="/">
  <meta name="viewport" content="width=device-width, initial-scale=1">
  <link rel="icon" type="image/x-icon" href="favicon.ico">
  <style>*,*:before,*:after{margin:0;padding:0;box-sizing:border-box;
    font-family:Inter,sans-serif}html{font-size:62.5%}</style>
  <link rel="stylesheet" href="styles.9d4c7581c7242.css">
</head>
<body>
  <app-root></app-root>
  <script src="runtime.6170988ad52a05db.js" type="module"></script>
  <script src="polyfills.574970d5ec4bdb97.js" type="module"></script>
  <script src="main.202d37bb6740400e.js" type="module"></script>
</body>
</html>
\end{lstlisting}
  \caption{Exemplo de HTML mínimo em aplicação Angular com CSR}
  \label{lst:angular_html}
\end{codigo}

Esse padrão é típico de aplicações \acrshort{spa}, onde todo o conteúdo é inserido dinamicamente a partir da execução dos arquivos JavaScript. O elemento \texttt{<app-root>} funciona como ponto de entrada da aplicação, sendo substituído no navegador pelos componentes definidos no framework Angular. {\cite{atori2024}}


\section{\english{Server-Side Rendering} (\acrshort{ssr})}
\label{subsec:ssr}

A \textbf{\english{\acrfull{ssr}}} é uma abordagem em que a geração do conteúdo e da interface ocorre integralmente no servidor antes de ser enviada ao navegador do cliente. Ou seja, o servidor processa a lógica da aplicação, obtém dados necessários (por exemplo, em bancos de dados ou \emph{APIs}) e retorna ao cliente um arquivo \english{\acrshort{html}} já renderizado. Dessa forma, o navegador exibe imediatamente a página completa, sem precisar executar \emph{scripts} para montar o conteúdo inicial \cite{atori2024}. 

Segundo \citeonline{atori2024}, o processo típico de renderização no lado do servidor pode ser descrito em quatro etapas principais:

\begin{enumerate}
    \item O servidor recebe uma requisição para uma página e recupera os dados necessários para compor seu conteúdo (por exemplo, produtos de uma base de dados ou artigos de um blog).
    \item O servidor insere esses dados em um \emph{template} \acrshort{html}, gerando a estrutura final da página.
    \item Em seguida, o servidor aplica estilos e finaliza a renderização, resultando em um documento \acrshort{html} completamente montado.
    \item Por fim, esse documento \acrshort{html} é enviado ao navegador do usuário, exibindo a página prontamente, sem a necessidade de executar \emph{JavaScript} durante o carregamento inicial.
\end{enumerate}

Nesse modelo, a fase de hydration\footnote{Hydration é uma etapa essencial no \acrshort{ssr}, em que o JavaScript torna interativo o conteúdo HTML previamente renderizado no servidor.} ocorre após o carregamento inicial da página. costuma ocorrer após a entrega do conteúdo estático. Significa que, assim que o arquivo \acrshort{html} é carregado e mostrado ao usuário, o \emph{JavaScript} do lado do cliente assume o controle para tratar as interações e atualizações dinâmicas subsequentes. Dessa forma, o \acrshort{ssr} beneficia tanto o primeiro acesso (tornando o conteúdo visível rapidamente) quanto o \acrshort{seo}, por exibir ao rastreador dos mecanismos de busca um código \acrshort{html} completo. \cite{atori2024}.

\begin{figure}[H]
  \centering
  \caption{Etapas do método de renderização no lado do servidor}
  \includegraphics[width=0.8\textwidth]{media/server_side_rendering.jpeg}
  \legend{Fonte: \cite{atori2024} (adaptado)}
  \label{fig:server_side_rendering}
\end{figure}

A \autoref{fig:server_side_rendering} ilustra o fluxo de uma aplicação \acrshort{ssr}. Ao receber a requisição, o servidor gera a página completa em \acrshort{html} e a envia ao cliente. Essa estratégia costuma ser vantajosa em cenários onde o carregamento inicial rápido e a indexação por motores de busca são prioridades, como em sites de e-commerce e páginas de \emph{landing}, permitindo que o usuário visualize o conteúdo de forma imediata. 

\emph{Meta-frameworks} como Next.js, Nuxt.js, SvelteKit, Angular Universal, Remix, Astro e Qwik são amplamente utilizados para construir aplicações com suporte a \acrshort{ssr}. Esses frameworks operam em um nível superior aos tradicionais (como React, Vue ou Svelte), agregando funcionalidades comuns ao desenvolvimento web, como roteamento, pré-renderização, recuperação de dados e \emph{hydration} podendo oferecer uma estrutura mais completa, opinativa e voltada à escalabilidade.

O \acrshort{ssr} é especialmente útil em aplicações que exigem um carregamento inicial rápido e uma boa indexação por motores de busca, como sites de e-commerce, blogs e páginas de \emph{landing}. Essa abordagem permite que o usuário visualize o conteúdo imediatamente, sem esperar pela execução do JavaScript. Além disso, o \acrshort{ssr} melhora a \acrshort{seo}, pois os mecanismos de busca conseguem indexar o conteúdo completo da página desde o início.

No \autoref{lst:nextjs_html}, pode-se observar que o arquivo \acrshort{html} já contém todo o \emph{markup} necessário para exibir o conteúdo da página. Assim que o navegador recebe esse arquivo, o usuário já visualiza o cabeçalho, o texto e o layout definidos. Posteriormente, o \emph{JavaScript} baixado (por exemplo, \texttt{main.js}) pode entrar em ação para lidar com eventos, rotas adicionais e atualizações dinâmicas, caso o desenvolvedor deseje funcionalidades mais interativas.

Por fim, aplicações \acrshort{ssr} tendem a apresentar melhor performance em termos de \emph{time-to-first-byte}\footnote{O \emph{time-to-first-byte} (TTFB) é uma métrica que mede o tempo decorrido entre o envio de uma solicitação HTTP pelo cliente e o recebimento do primeiro byte da resposta do servidor. Um TTFB menor indica maior rapidez na resposta do servidor, impactando diretamente na velocidade de carregamento da página e na experiência do usuário. \cite{ttfb-craig}} e de \emph{indexabilidade}\footnote{A \emph{indexabilidade} refere-se à capacidade dos motores de busca de rastrear e indexar o conteúdo de uma página web. Aplicações SSR, ao fornecerem conteúdo totalmente renderizado no servidor, facilitam a indexação eficiente pelos motores de busca, melhorando a visibilidade nos resultados de pesquisa. \cite{ttfb-oskay}} por motores de busca, ao mesmo tempo em que podem demandar maior carga de processamento no servidor. A escolha por \acrshort{ssr} ou não, portanto, depende do perfil da aplicação e das prioridades do projeto, considerando fatores como volume de tráfego, necessidade de interatividade e requisitos de otimização de conteúdo.

\begin{codigo}[H]
  \begin{lstlisting}[language=html]
    <!DOCTYPE html>
    <html lang="en">
    <head>
      <meta charset="utf-8">
      <title>My SSR App</title>
      <meta name="viewport" content="width=device-width, initial-scale=1">
      <style>
        /* Exemplo simples de estilo inline */
        body {
          margin: 0;
          font-family: Arial, sans-serif;
          background: #f6f6f6;
        }
        h1 { color: #333; }
      </style>
    </head>
    <body>
<!-- Conteudo ja processado e inserido no servidor -->
      <div id="__next">
        <header>
          <h1>Ola, mundo!</h1>
        </header>
        <main>
          <p>Este conteudo foi renderizado no servidor usando Next.js.</p>
        </main>
      </div>
      <!-- Scripts do Next.js para interacao no cliente -->
      <script src="/_next/static/chunks/main.js" defer></script>
    </body>
    </html>
  \end{lstlisting}
  \caption{Exemplo de HTML mínimo em aplicação Next.js com SSR}
  \label{cod:nextjs_html}
\end{codigo}


\section{Frameworks Web}
\label{sec:frameworks-web}

O uso de bibliotecas e frameworks no desenvolvimento web moderno proporciona ganhos significativos de produtividade, desempenho e organização de código. Eles abstraem operações complexas e oferecem estruturas padronizadas para construção de aplicações escaláveis. A escolha da ferramenta está diretamente relacionada à abordagem de renderização adotada, seja no lado do cliente (\acrshort{csr}) ou do servidor (\acrshort{ssr}).

\subsection{Bibliotecas JavaScript}
\label{subsec:bibliotecas-js}

Bibliotecas JavaScript são conjuntos de funcionalidades reutilizáveis que fornecem recursos específicos, permitindo que o desenvolvedor tenha maior controle sobre o fluxo da aplicação. Elas diferem dos frameworks por não imporem uma estrutura rígida, sendo mais flexíveis em sua aplicação.

\textbf{React} é um exemplo amplamente utilizado de biblioteca voltada para a criação de interfaces de usuário baseadas em componentes reutilizáveis. Desenvolvida pelo Facebook, React se destaca por sua abordagem declarativa, desempenho otimizado com o uso de um \textit{virtual DOM}, e ampla adoção na construção de aplicações com \acrshort{csr}. Apesar de muitas vezes ser chamado de framework, React é tecnicamente uma biblioteca, já que seu foco é exclusivamente a camada de visualização, deixando a cargo do desenvolvedor a escolha de ferramentas para rotas, estados e lógica de aplicação~\cite{react2025}.

\subsection{Frameworks para CSR}
\label{subsec:frameworks-csr}

No modelo \acrshort{csr}, a renderização da interface é realizada diretamente no navegador do usuário, após o carregamento dos arquivos JavaScript. Frameworks como os listados a seguir são amplamente utilizados para implementar essa abordagem:

\begin{itemize}
    \item \textbf{Vue.js}: framework progressivo para construção de interfaces web interativas. Seu foco está na camada de visualização, com curva de aprendizado acessível e estrutura modular~\cite{vue2025}.
    
    \item \textbf{Angular}: framework completo mantido pelo Google, baseado em TypeScript, que oferece arquitetura robusta e recursos integrados como injeção de dependência e roteamento~\cite{angular2025}.
    
    \item \textbf{Svelte}: framework que realiza a compilação dos componentes no momento do build, eliminando a necessidade de um \textit{virtual DOM}, o que reduz o tempo de carregamento e o uso de recursos do navegador~\cite{svelte2025}.
\end{itemize}

Esses frameworks tornam o desenvolvimento com \acrshort{csr} mais eficiente e sustentável, proporcionando experiências ricas ao usuário com foco em interatividade e responsividade.

\subsection{Meta-frameworks para SSR}
\label{subsec:frameworks-ssr}

Para aplicações com foco em renderização no lado do servidor, os \emph{meta-frameworks} oferecem soluções completas, otimizando tanto o desempenho inicial quanto a indexabilidade em mecanismos de busca. Eles operam sobre frameworks tradicionais (como Vue ou Svelte) ou bibliotecas (como React), incorporando funcionalidades essenciais como roteamento, pré-renderização, recuperação de dados e \textit{hydration}.

\begin{itemize}
    \item \textbf{Next.js}: baseado em React, fornece recursos para \acrshort{ssr}, geração de sites estáticos e suporte a APIs integradas~\cite{nextjs2024}.
    
    \item \textbf{Nuxt.js}: extensão do Vue.js que oferece SSR, geração estática e arquitetura modular~\cite{nuxtjs2024}.
    
    \item \textbf{SvelteKit}: baseado em Svelte, permite renderização no servidor e no cliente, com foco em simplicidade e desempenho~\cite{sveltekit2024}.
    
    \item \textbf{Angular Universal}: solução oficial para SSR em Angular, melhora a indexação e o tempo de carregamento inicial~\cite{angularuniversal2024}.
    
    \item \textbf{Remix}: framework full-stack para React que adota um modelo de dados centrado em carregadores e ações~\cite{remix2024}.
    
    \item \textbf{Astro}: framework moderno que carrega apenas o JavaScript necessário, permitindo uso híbrido de componentes React, Vue, Svelte e outros~\cite{astro2024}.
    
    \item \textbf{Qwik}: introduz o conceito de aplicações \textit{resumíveis}, com SSR e carregamento progressivo de interatividade~\cite{qwik2024}.
\end{itemize}

Esses meta-frameworks são especialmente indicados para aplicações que priorizam SEO, acessibilidade e desempenho no primeiro carregamento, como páginas institucionais, lojas virtuais e blogs.

\subsection{Comparativo entre Frameworks CSR e SSR}
\label{subsec:comparativo-frameworks}

\begin{table}[H]
\centering
\caption{Comparação entre frameworks para CSR e SSR}
\label{tab:comparativo-frameworks}
\begin{tabular}{|p{3cm}|p{5.5cm}|p{5.5cm}|}
\hline
\textbf{Critério} & \textbf{Frameworks CSR (Vue, Angular, Svelte)} & \textbf{Meta-frameworks SSR (Next.js, Nuxt, SvelteKit, etc.)} \\
\hline
\textbf{Renderização Inicial} & O conteúdo é montado no navegador após o carregamento do JavaScript & O conteúdo é gerado no servidor e entregue já renderizado ao navegador \\
\hline
\textbf{Tempo de Carregamento} & Maior tempo de carregamento inicial (dependente do JS) & Melhor desempenho no carregamento inicial (TTFB menor) \\
\hline
\textbf{SEO} & Pode ser limitado, pois bots podem não processar JavaScript adequadamente & Excelente, já que o HTML completo está disponível para rastreadores \\
\hline
\textbf{Interatividade} & Alta, com foco em aplicações ricas e dinâmicas & Boa, com necessidade de \textit{hydration} após o carregamento \\
\hline
\textbf{Complexidade de Infraestrutura} & Menor, geralmente servido por CDNs e arquivos estáticos & Maior, exige servidores para processar cada requisição \\
\hline
\textbf{Casos de Uso Ideais} & SPAs, dashboards, aplicações com muitas interações em tempo real & Landing pages, blogs, e-commerces, sites que dependem de SEO \\
\hline
\end{tabular}
\end{table}


% \subsection{Modelos de Arquitetura Web}
% \label{subsec:modelos-arq-web}
% Os modelos arquiteturais variam conforme os requisitos de escalabilidade, manutenção e desempenho:

% \begin{itemize}
%     \item \textbf{Arquitetura Monolítica}: Um único projeto concentra \textit{frontend} e \textit{backend}, frequentemente usando \acrshort{ssr}. Possui inicialização simples, mas pode tornar-se complexo de manter e escalar.
%     \item \textbf{Microserviços}: Divide a aplicação em múltiplos serviços independentes. Cada serviço pode escolher a melhor abordagem de renderização (SSR ou CSR), facilitando a escalabilidade seletiva.
%     \item \textbf{Serverless}: As funções são executadas sob demanda em plataformas de nuvem, onde a renderização pode ocorrer tanto no servidor (funções que retornam HTML) quanto no cliente (ao entregar apenas APIs).
% \end{itemize}

% \subsection{Ferramentas e \textit{Frameworks}}
% \label{subsec:ferramentas-frameworks}
% O ecossistema de desenvolvimento web oferece diversas ferramentas que simplificam \acrshort{ssr} e \acrshort{csr}:

% \begin{itemize}
%     \item \textbf{\acrshort{ssr}}: \textit{Next.js} (React), \textit{Nuxt.js} (Vue), \textit{SvelteKit} (Svelte), entre outros.
%     \item \textbf{\acrshort{csr}}: React, Vue.js, Angular e muitas bibliotecas voltadas para \textit{Single Page Applications} (SPA).
% \end{itemize}


% \section{Infraestrutura de Serviço Web}
% \label{sec:infraestrutura-web}

% A decisão por \acrshort{ssr} ou \acrshort{csr} influencia diretamente a infraestrutura necessária:

% \begin{itemize}
%     \item \textbf{Servidores e Processamento}: Em \acrshort{ssr}, o servidor gera páginas dinamicamente, aumentando a carga de CPU. Já em \acrshort{csr}, o servidor atua mais como um provedor de arquivos estáticos e APIs.
%     \item \textbf{\english{Content Delivery Networks} (CDNs)}: Tanto para SSR quanto para CSR, uma CDN pode melhorar a distribuição de arquivos estáticos (HTML, CSS, JavaScript, imagens) e reduzir a latência.
%     \item \textbf{Escalabilidade}: Aplicações com alto número de requisições precisam de estratégias adequadas para lidar com picos de acesso. Em \acrshort{ssr}, muitas requisições simultâneas podem sobrecarregar o servidor; em \acrshort{csr}, o foco está em serviços de dados e na entrega eficiente de arquivos iniciais.
% \end{itemize}

% \textbf{Segurança} também se faz presente em ambas as abordagens. Boas práticas incluem:
% \begin{itemize}
%     \item Uso de \textbf{HTTPS} para proteger a comunicação.
%     \item Implementação de \textbf{CORS} (Cross-Origin Resource Sharing) quando necessário.
%     \item Tratamento de \textbf{tokens de sessão/autenticação} com cuidado para evitar vazamento de dados.
% \end{itemize}

% ---

\section{Experiência do Usuário}
\label{sec:ux}
A \acrfull{ux} é um aspecto crítico no desenvolvimento de aplicações web, influenciando diretamente a satisfação e a eficácia da interação do usuário com o sistema \cite{atori2023}. Para alcançar uma \acrshort{ux} satisfatória, a escolha entre \acrshort{csr} e \acrshort{ssr} deve considerar fatores como: {\acrshort{seo}}, velocidade de carregamento, interatividade e responsividade.
Conforme \citeonline{atori2024}, a \acrshort{ux} vai além da interface gráfica, englobando toda a jornada do usuário desde a navegação até a conclusão de tarefas. 

\subsection{\english{Search Engine Optimization} (SEO)}
\label{sec:seo}

O \acrfull{seo} consiste em um conjunto integrado de práticas de otimização, tanto no aspecto técnico quanto no de conteúdo, com três objetivos principais: maximizar a visibilidade orgânica nos mecanismos de busca, posicionar estrategicamente páginas-chave e garantir uma experiência de usuário qualificada durante o processo de busca. Essas práticas são essenciais para garantir que o conteúdo de um site seja facilmente encontrado e indexado pelos motores de busca, aumentando a probabilidade de atrair visitantes qualificados. Entre os fatores mais conhecidos, destaca-se a velocidade de carregamento da página, que impacta diretamente a experiência do usuário e a classificação nos resultados de busca \cite{conor2022}.
\begin{figure}[H]
    \centering
    \caption{Tempo de Rastreamento e Posicionamento da Página}
    \includegraphics[width=0.6\textwidth]{media/rank_crawl_and_page_rank.png}
    \legend{Fonte: \cite{webPerformance}(adaptado)}
    \label{fig:rank_crawl_and_page_rank}
\end{figure}

A \autoref{fig:rank_crawl_and_page_rank} ilustra a relação entre o tempo de rastreamento e o posicionamento da página. O tempo de rastreamento refere-se ao tempo que os mecanismos de busca levam para acessar e indexar uma página. Quanto mais rápido o tempo de rastreamento, maior a probabilidade de a página ser indexada rapidamente e, consequentemente, melhor seu posicionamento nos resultados de busca. Isso destaca a importância de otimizar o desempenho do site para garantir uma boa classificação nos motores de busca.



\subsection{\english{Velocidade de carregamento} }
\label{sec:velocidade da página}
A velocidade de carregamento da página refere-se ao tempo que uma página da web leva para carregar completamente, desde a solicitação do usuário até que o conteúdo esteja totalmente visível e interativo no navegador. Esse tempo pode ser impactado por diversos fatores, como o tamanho dos arquivos da página, a complexidade do conteúdo, a qualidade da conexão com a internet e o desempenho do servidor.

A velocidade de carregamento é um dos principais fatores que afetam a \acrshort{ux} e o \acrshort{seo}. Páginas que carregam rapidamente tendem a ter taxas de rejeição mais baixas e melhor desempenho em termos de conversão. O \acrshort{seo} também considera a velocidade de carregamento como um fator importante para determinar a classificação nos resultados de busca \cite{conor2022}.

Segundo o \citeonline{google}, aumentar a velocidade de carregamento é fundamental: quanto mais rápido o site carregar, melhor será a experiência do usuário. Caso contrário, o usuário passa menos tempo na página, o que pode aumentar a taxa de rejeição.
