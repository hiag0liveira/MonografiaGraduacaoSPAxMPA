\chapter{Introdução}
\label{cap:introducao}

\section{Problema e contexto}
O crescimento acelerado da web e o aumento da complexidade das aplicações modernas impuseram novos desafios ao desenvolvimento e à entrega de conteúdos na internet. Com o crescimento exponencial da web, estima-se que aproximadamente 252 mil novos sites sejam desenvolvidos diariamente, demonstrando não apenas a rapidez com que aplicações são criadas, mas também a necessidade crescente de estratégias eficientes para otimização de desempenho e escalabilidade \cite{dataInternetUsage}. A escolha da abordagem de renderização tornou-se um fator determinante para a experiência do usuário e a escalabilidade dos sistemas. Inicialmente, os sites eram compostos por páginas estáticas, cujo conteúdo era carregado diretamente do servidor. Com a evolução das tecnologias frontend, novas abordagens surgiram, destacando-se \english{\acrfull{csr}} e \english{\acrfull{ssr}}. Cada uma dessas técnicas possui características específicas que influenciam diretamente o desempenho e a experiência do usuário.

A arquitetura frontend, desempenha um papel essencial na definição do fluxo de desenvolvimento e na escolha entre \acrshort{csr} e \acrshort{ssr} \cite{frontendGodbolt}. O autor enfatiza a importância de um sistema modular e eficiente, que possa ser mantido e escalado de maneira sustentável. No \acrshort{csr}, frameworks como React e Vue.js permitem que a renderização ocorra no navegador do usuário, reduzindo a carga no servidor, mas exigindo mais processamento no cliente. Já no \acrshort{ssr}, o conteúdo é gerado no servidor antes de ser enviado ao cliente, proporcionando carregamento mais rápido e melhor desempenho em dispositivos menos potentes.

A performance em websites é um fator determinante para o sucesso de qualquer aplicação web. O desempenho, frequentemente medido pelo tempo de carregamento das páginas, desempenha um papel fundamental na experiência do usuário e na taxa de conversão de visitantes \cite{webPerformance}. Uma página que carrega rapidamente proporciona uma navegação mais fluida, reduzindo a taxa de rejeição e aumentando a retenção de usuários. Além disso, o desempenho da página não afeta apenas o usuário, mas também a posição nos mecanismos de busca (\english{\acrfull{seo}}), tornando-se um critério essencial para indexação e ranqueamento em plataformas como o \emph{Google} \cite{google}.

A renderização está diretamente ligada à performance de uma aplicação, e a decisão entre \acrshort{csr} e \acrshort{ssr} deve ser tomada com base em diversos fatores, como tempo de carregamento, complexidade da página e número de requisições HTTP \cite{webPerformance}. Diferentes abordagens impactam não apenas a experiência do usuário, mas também os custos operacionais e a infraestrutura necessária para suportar a aplicação.

Um exemplo notável de desafios enfrentados na escolha da estratégia de renderização ocorreu no \emph{Twitter}. Em 2010, a empresa lançou uma nova versão de sua plataforma, conhecida como New Twitter, que utilizava extensivamente a renderização no lado do cliente (\acrshort{csr}) para aprimorar a interatividade e a experiência do usuário. No entanto, essa abordagem resultou em problemas significativos de desempenho, especialmente para usuários com conexões de internet mais lentas ou dispositivos menos potentes. Além disso, a dependência intensa de JavaScript dificultou a indexação de conteúdo pelos mecanismos de busca, impactando negativamente a otimização para motores de busca (\acrshort{seo}) \cite{twitter}. Reconhecendo essas limitações, o Twitter decidiu retornar à renderização no lado do servidor (\acrshort{ssr}) em 2012, visando melhorar o desempenho e a acessibilidade de sua plataforma.

Diante desse cenário, este trabalho tem como objetivo analisar comparativamente as abordagens de \acrshort{csr} e \acrshort{ssr}, avaliando suas vantagens, desvantagens e aplicações ideais. Para isso, serão considerados aspectos como tempo de carregamento, impacto na experiência do usuário, consumo de recursos e facilidade de manutenção. A pesquisa visa contribuir para a compreensão de qual abordagem é mais adequada em diferentes contextos, auxiliando desenvolvedores e arquitetos de software na tomada de decisão.


\section{Justificativa}


\section{Objetivos}

\subsection{Objetivo Geral}


\subsection{Objetivos Específicos}
\begin{itemize}
\item item 1
\item item 2
\item item 3
\end{itemize}

\section{Metodologia}


\section{Estrutura do Trabalho}

