\chapter{Introdução}
\label{cap:introducao}

\section{Problema e contexto}
O crescimento acelerado da web e o aumento da complexidade das aplicações modernas impuseram novos desafios ao desenvolvimento e à entrega de conteúdos na internet. Com o crescimento exponencial da web, estima-se que aproximadamente 252 mil novos sites sejam desenvolvidos diariamente, demonstrando não apenas a rapidez com que aplicações são criadas, mas também a necessidade crescente de estratégias eficientes para otimização de desempenho e escalabilidade \cite{dataInternetUsage}. A escolha da abordagem de renderização tornou-se um fator determinante para a experiência do usuário e a escalabilidade dos sistemas. Inicialmente, os sites eram compostos por páginas estáticas, cujo conteúdo era carregado diretamente do servidor. Com a evolução das tecnologias frontend, novas abordagens surgiram, destacando-se \english{\acrfull{csr}} e \english{\acrfull{ssr}}. Cada uma dessas técnicas possui características específicas que influenciam diretamente o desempenho e a experiência do usuário.

A performance em websites é um fator determinante para o sucesso de qualquer aplicação web. O desempenho, frequentemente medido pelo tempo de carregamento das páginas, desempenha um papel fundamental na experiência do usuário e na taxa de conversão de visitantes \cite{webPerformance}. Uma página que carrega rapidamente proporciona uma navegação mais fluida, reduzindo a taxa de rejeição e aumentando a retenção de usuários. Além disso, o desempenho da página não se limita a impactar a experiência do usuário, mas também interfere diretamente no \english{\acrfull{seo}}, tornando-se um critério essencial de indexação e ranqueamento em plataformas como o Google \cite{google}.

Um exemplo notável de desafios enfrentados na escolha da estratégia de renderização ocorreu no \emph{Twitter}. Em 2010, a empresa lançou uma nova versão de sua plataforma, conhecida como New Twitter, que utilizava extensivamente a renderização no lado do cliente (\acrshort{csr}) para aprimorar a interatividade e a experiência do usuário. No entanto, essa abordagem resultou em problemas significativos de desempenho, especialmente para usuários com conexões de internet mais lentas ou dispositivos menos potentes. Além disso, a dependência intensa de JavaScript dificultou a indexação de conteúdo pelos mecanismos de busca, impactando negativamente a otimização para motores de busca (\acrshort{seo}) \cite{twitter}. Reconhecendo essas limitações, o Twitter decidiu retornar à renderização no lado do servidor (\acrshort{ssr}) em 2012, visando melhorar o desempenho e a acessibilidade de sua plataforma.

A arquitetura de frontend desempenha papel fundamental ao definir o fluxo de desenvolvimento e a escolha entre \acrshort{csr} e \acrshort{ssr}, sendo indispensável a adoção de um sistema modular e eficiente, capaz de ser mantido e escalado de forma sustentável \cite{frontendGodbolt}. Na abordagem \acrshort{csr}, a renderização ocorre diretamente no navegador do usuário, reduzindo a carga no servidor, mas exigindo mais processamento no cliente; já na \acrshort{ssr}, o conteúdo é gerado no servidor antes de ser enviado ao cliente, o que proporciona carregamento mais rápido e melhor desempenho em dispositivos menos potentes. A decisão entre essas estratégias está diretamente ligada à performance da aplicação e deve considerar fatores como tempo de carregamento, complexidade da página e número de requisições HTTP \cite{webPerformance}, já que diferentes abordagens afetam não apenas a experiência do usuário, mas também os custos operacionais e a infraestrutura necessária para suportar a aplicação.

\section{Justificativa}

Nos últimos anos, observou-se um notável crescimento na adoção das abordagens de renderização no lado do cliente (\acrshort{csr}) e no lado do servidor (\acrshort{ssr}) no desenvolvimento de aplicações web. Contudo, essas técnicas são frequentemente empregadas de maneira inadequada em muitos projetos, seja pela falta de entendimento de suas vantagens e limitações, seja por uma análise superficial das necessidades do produto \cite{atori2024}. Um exemplo ilustrativo dessa realidade pode ser visto na experiência do \emph{Airbnb}, que optou por uma abordagem de \acrshort{ssr} com o intuito de melhorar o desempenho em dispositivos com recursos limitados e, sobretudo, otimizar a indexação de seu vasto catálogo de acomodações em mecanismos de busca \cite{neary2017}. Por outro lado, a equipe do \emph{Instagram} enfrentou desafios ao equilibrar o carregamento dinâmico de conteúdo no cliente com a necessidade de garantir uma experiência fluida aos usuários, levando-os a adotar soluções híbridas que envolvem tanto \acrshort{csr} quanto \acrshort{ssr} em diferentes partes da aplicação \cite{conner2019}.

Paralelamente a esses casos, identifica-se uma carência de estudos de caso reais que analisem de forma aprofundada o impacto da adoção de \acrshort{csr} e \acrshort{ssr}, principalmente no contexto nacional. Enquanto algumas publicações se concentram em apenas uma dessas abordagens, outras fornecem exemplos excessivamente simplificados, limitando a compreensão dos desafios técnicos e de negócios ao combinar essas estratégias em sistemas complexos.

Diante desse cenário, o presente trabalho visa suprir essa lacuna ao oferecer uma análise comparativa detalhada sobre a implementação de casos com \acrshort{csr} e \acrshort{ssr}, avaliando seus efeitos no desempenho, na experiência do usuário, segurança, otimização do \acrshort{seo} e na escalabilidade de aplicações web modernas. Por meio de um estudo de caso abrangente, pretende-se não apenas enriquecer a literatura acadêmica, mas também fornecer insights práticos que possam orientar equipes de desenvolvimento e gestores na seleção e aplicação adequada dessas técnicas, contribuindo para a construção de sistemas mais robustos, eficientes e alinhados às demandas de um mercado em constante evolução.

\section{Objetivos}

\subsection{Objetivo Geral}


\subsection{Objetivos Específicos}
\begin{itemize}
\item item 1
\item item 2
\item item 3
\end{itemize}

\section{Metodologia}


\section{Estrutura do Trabalho}

