\chapter{Trabalhos Relacionados}
\label{cap:trabalhos}
Este capítulo apresenta os trabalhos relacionados ao objeto de pesquisa obtidos utilizando o protocolo descrito na seção a seguir.

\section{Protocolo}
Esta seção apresenta o protocolo de mapeamento sistemático da literatura usado para atingir os objetivos da pesquisa.

\subsection{Questões de pesquisa}
\label{section:questoes_pesquisa}
\begin{enumerate}
    \item[Q1:] De que maneira a escolha entre \acrshort{csr} e \acrshort{ssr} influenciam a experiência do usuário?
    \item[Q2:] Como as abordagens \acrshort{csr} e \acrshort{ssr} afetam métricas de performance, tempo de carregamento e tempo até a interatividade em aplicações web?
    \item[Q3:] Quais são os principais desafios e \english{trade-offs} na implementação de \acrshort{csr} e \acrshort{ssr}?
\end{enumerate}

\subsection{Estratégia de busca}
Esta seção apresenta a estratégia de buscas de artigos científicos e livros relacionados à pesquisa. As ferramentas utilizadas para realizar as buscas são:
\begin{itemize}
    \item \textbf{Periódicos Capes:} É uma ferramenta disponibilizada pelo governo federal para uso de estudantes e pesquisadores. Acessando através da instituição de ensino ou pesquisa, é possível ter acesso completo a uma grande quantidade de artigos científicos publicados em variadas revistas, conferências e universidades. A principal vantagem dessa ferramenta é a possibilidade de ler o conteúdo integral de grande parte das publicações disponíveis. Por outro lado, as expressões de busca atualmente suportadas são bem limitadas.
    \item \textbf{\english{Scopus:}} Trata-se de um ferramenta similar ao Periódicos Capes. No entanto, o \english{Scopus} permite a elaboração de expressões de buscas mais complexas e sofisticadas, servindo para descobrir publicações não detectadas pelas outras plataformas. Além disso, possui um acervo bem mais amplo que o Periódicos Capes. Entretanto, algumas publicações não podem ser vistas na íntegra de forma gratuita.
    \item \textbf{\english{Google Docs:}} Ferramenta desenvolvida pela \english{Google LLC} que permite a criação e edição de documentos de texto. Suas grandes vantagens em relação a ferramentas de outros fornecedores são as avançadas ferramentas de colaboração e a possibilidade de acesso por meio de navegadores \english{web}, sem necessidade de instalação de \english{software} específico.
    \item \textbf{\english{Google Sheets}}: Com as mesmas características e vantagens do \english{Google Docs}, essa ferramenta fornece recursos para elaboração de planilhas de cálculo. É muito útil para realizar análise de dados simples e também visualizar e apresentar dados tabulares.
\end{itemize}

Como grande parte das publicações na área de computação são em inglês, esta pesquisa utiliza esse idioma para fazer buscas nas ferramentas indicadas. Além disso, \acrfull{csr} e \acrfull{ssr} são relativamente recentes, as buscas se limitaram a publicações feitas nos últimos 20 anos.

Os termos-chave para realização das buscas são: Microsserviço, \acrshort{csr} e \acrlong{ssr}. Como a busca é feita em inglês, se usará \english{web performance} nas buscas.

\subsection{Expressão de busca}
\label{section:string_busca}

\begin{quadro}[H]
\centering

\setlength{\tabcolsep}{0.8em} % for the horizontal padding
\renewcommand{\arraystretch}{1.5}% for the vertical padding
\caption{Expressão de busca utilizada}
\begin{tabular}{|p{4.5in}|}

\hline
Expressão de Busca \\ \hline
\english{( ( TITLE-ABS-KEY (CSR OR Client-Side Rendering) AND TITLE-ABS-KEY (SSR OR Server-Side Rendering) AND TITLE-ABS-KEY (web performance)
) )} \\ \hline

\end{tabular}
\label{quad:string_busca}
\fonte{os autores}
\end{quadro}