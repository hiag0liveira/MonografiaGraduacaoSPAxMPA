\chapter{Conclusão}
\label{cap:conclusao}

Este trabalho de conclusão de curso se propôs a realizar uma análise comparativa aprofundada entre as arquiteturas de renderização no lado do cliente (\acrshort{csr}) e no lado do servidor (\acrshort{ssr}), com o intuito de dissecar suas implicações na performance, na experiência do usuário e nos desafios de implementação em aplicações web modernas. A partir de um estudo de caso prático, materializado no desenvolvimento e teste de duas versões da plataforma "WallTech" em um ambiente controlado, foi possível coletar dados empíricos que não apenas validam conceitos teóricos, mas também oferecem um panorama claro sobre os trade-offs inerentes a cada abordagem.

A investigação revelou que a escolha entre \acrshort{csr} e \acrshort{ssr} molda fundamentalmente a percepção e a jornada do usuário. A abordagem \acrshort{ssr} demonstrou um impacto profundamente positivo na fase inicial da interação. Ao entregar ao navegador um documento HTML já processado e com conteúdo, a aplicação obteve métricas de \acrfull{ttfb} e \acrfull{fcp} notavelmente rápidas. Essa agilidade na exibição do primeiro conteúdo visível é um fator psicológico poderoso, que transmite uma sensação de eficiência e robustez, crucial para reter a atenção do usuário nos primeiros segundos críticos. Adicionalmente, a alta estabilidade visual, com um índice de \acrfull{cls} praticamente nulo, garantiu que a experiência de leitura não fosse interrompida por mudanças de layout abruptas, um ponto essencial para a credibilidade de um portal de notícias. A contrapartida, no entanto, manifesta-se na navegação subsequente, que, ao exigir uma nova requisição completa ao servidor para cada página, pode quebrar a sensação de fluidez contínua.

Em contrapartida, o \acrshort{csr} redefiniu a experiência do usuário após o carregamento inicial. Embora tenha apresentado um desafio no tempo de carregamento inicial com métricas de \acrfull{lcp} mais lentas devido à necessidade de baixar e executar o pacote JavaScript antes da renderização, uma vez operacional, a aplicação ofereceu uma interatividade superior. A navegação entre seções e o acesso a detalhes de notícias ocorreram de forma quase instantânea, com um \acrfull{inp} muito baixo, emulando a agilidade de um aplicativo nativo. Essa fluidez é o grande trunfo do \acrshort{csr}. Contudo, o estudo também expôs sua principal vulnerabilidade: uma forte tendência a problemas de estabilidade visual. O processo de hidratação e a renderização dinâmica de componentes levaram a ocorrências de \acrfull{cls} classificadas como "ruins", um atrito significativo que pode frustrar o usuário e comprometer a usabilidade.

Do ponto de vista da implementação, a análise evidenciou que a escolha arquitetônica representa um trade-off fundamental sobre onde a carga computacional deve residir. O \acrshort{ssr} concentra o processamento no servidor, o que implica um maior consumo de CPU e memória para cada requisição, elevando os custos de infraestrutura e a complexidade do escalonamento. O desafio técnico para a equipe de desenvolvimento está em otimizar não apenas a renderização no servidor, mas também o processo de hidratação no cliente, garantindo que a interatividade seja ativada de forma eficiente sem criar gargalos.

O \acrshort{csr}, por outro lado, transfere essa carga para o dispositivo do usuário. Esta abordagem simplifica drasticamente a infraestrutura do servidor, que passa a atuar majoritariamente como um provedor de arquivos estáticos, resultando em custos operacionais significativamente menores. No entanto, a complexidade é movida para o frontend. Os desafios tornam-se gerenciar o tamanho do \textit{bundle} JavaScript, que pode crescer a ponto de comprometer o desempenho em dispositivos menos potentes; implementar um gerenciamento de estado coeso e escalável no cliente; e, crucialmente, contornar as limitações inerentes de \acrshort{seo}, além de aplicar uma disciplina rigorosa de desenvolvimento para prevenir as instabilidades de layout que este estudo identificou.

Os resultados obtidos alinham-se e fornecem validação empírica para as recomendações consolidadas na literatura técnica e acadêmica. A pesquisa confirmou que não existe uma solução universalmente superior, mas sim uma adequação ao contexto. Os achados deste trabalho reforçam a recomendação de priorizar o \acrshort{ssr} para aplicações de conteúdo, como portais de notícias, plataformas de e-commerce e blogs, onde a otimização para mecanismos de busca é vital para a aquisição de tráfego e a velocidade da primeira impressão é um fator determinante para o sucesso do negócio. Em contraste, o \acrshort{csr} se mostra a escolha mais acertada para aplicações web ricas e complexas, como dashboards analíticos, sistemas de gestão interna e plataformas de software como serviço (SaaS), que geralmente são acessadas após autenticação e onde a fluidez e a riqueza da interatividade contínua são mais valorizadas do que a indexabilidade de páginas individuais.

Em suma, este trabalho cumpriu seus objetivos ao demonstrar, com dados concretos, que a decisão entre \acrshort{csr} e \acrshort{ssr} é uma escolha estratégica que transcende a tecnologia e deve ser ponderada à luz dos objetivos do produto, das expectativas do usuário e das capacidades da equipe de desenvolvimento. A principal contribuição desta pesquisa reside na sua metodologia sistemática e reprodutível, que oferece um roteiro prático para a avaliação comparativa de arquiteturas frontend. Como limitação, reconhece-se que o ambiente de teste local, embora essencial para o controle experimental, não captura a variabilidade de redes e dispositivos do mundo real.

Para trabalhos futuros, abre-se um leque de possibilidades, como a análise de arquiteturas híbridas a exemplo de Static Site Generation (SSG) e Incremental Static Regeneration (ISR) que prometem unir o melhor dos dois mundos. Adicionalmente, seria de grande valia investigar o impacto de paradigmas emergentes, como os React Server Components e o Streaming SSR, na performance e na experiência do desenvolvedor. Por fim, a complementação dos dados quantitativos com estudos qualitativos de usabilidade poderia fornecer insights ainda mais profundos sobre como os usuários percebem, de fato, as diferenças entre essas duas poderosas abordagens de renderização na web.