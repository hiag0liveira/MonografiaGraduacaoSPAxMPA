\chapter{Conclusão}
\label{cap:conclusao}

Este trabalho de conclusão de curso realizou uma análise comparativa aprofundada entre as arquiteturas de renderização no lado do cliente (\acrshort{csr}) e no lado do servidor (\acrshort{ssr}), com o intuito de dissecar suas implicações na performance, na experiência do usuário e nos desafios de implementação. A partir do desenvolvimento e teste de duas versões da plataforma "WallTech" em um ambiente controlado, foi possível validar empiricamente os trade-offs inerentes a cada abordagem.

A investigação confirmou que a escolha arquitetônica molda fundamentalmente a jornada do usuário. A abordagem \acrshort{ssr} demonstrou ser superior na fase inicial da interação, entregando um primeiro conteúdo visual (\acrshort{fcp}) de forma notavelmente rápida e garantindo excelente estabilidade visual (\acrshort{cls}). Essa performance inicial, aliada à otimização natural para mecanismos de busca (\acrshort{seo}), a torna ideal para aplicações onde a primeira impressão e a descoberta de conteúdo são críticas.

Em contrapartida, o \acrshort{csr} redefiniu a experiência após o carregamento inicial, oferecendo uma interatividade superior e navegação quase instantânea, refletida em um baixo \acrshort{inp}, que emula a fluidez de um aplicativo nativo. Contudo, essa vantagem vem ao custo de um carregamento inicial mais lento e de uma maior suscetibilidade a instabilidades de layout, exigindo uma disciplina de desenvolvimento rigorosa para mitigar os pontos fracos da abordagem.

Portanto, este trabalho cumpriu seus objetivos ao demonstrar, com dados concretos, que a decisão entre \acrshort{csr} e \acrshort{ssr} é uma escolha estratégica. Ela se resume a um trade-off fundamental sobre onde a carga computacional deve residir e o que se deve priorizar: a velocidade da primeira renderização e a indexabilidade do \acrshort{ssr}, ou a interatividade contínua e a redução de custos de servidor do \acrshort{csr}.

A principal contribuição desta pesquisa reside na sua metodologia sistemática e reprodutível, que oferece um roteiro prático para a avaliação comparativa de arquiteturas frontend. Como limitação, reconhece-se que o ambiente de teste local, embora essencial para o controle experimental, não captura a variabilidade de redes e dispositivos do mundo real.

Para trabalhos futuros, sugere-se a análise de arquiteturas híbridas, como \textit{Static Site Generation} (SSG) e \textit{Incremental Static Regeneration} (ISR), que prometem unir o melhor dos dois mundos. Adicionalmente, seria de grande valia investigar o impacto de paradigmas emergentes, como os \textit{React Server Components} e o \textit{Streaming SSR}, e complementar os dados quantitativos com estudos qualitativos de usabilidade, para aprofundar a compreensão sobre a percepção real do usuário.