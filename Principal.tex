% !TeX root = Principal.tex


% Desenvolvido por: Prof. Dr. David Buzatto
% Adaptado por: Prof. Dr. Fernando Carvalho
%
% Baseado na documentação do abntex2 e nos modelos em
% Microsoft Word propostos pela Profa. Dra. Rosana F. L. Rodrigues
% e pela bibliotecária M.Sc. Maria Carolina Gonçalves do câmpus
% São João da Boa Vista do IFSP.
%
% Versão 1.5
% Data: 03/09/2018


\documentclass[
	% -- opções da classe memoir --
	12pt,				% tamanho da fonte
	openany,			% capítulos começam em pág ímpar openright (insere página vazia caso preciso)
	oneside,			% para impressão em verso e anverso (twoside). Oposto a oneside 
	a4paper,			% tamanho do papel. 
	normalfigtabnum,
	pnumromarab,
	% -- opções da classe abntex2 --
	chapter=Title,		% títulos de capítulos convertidos em letras maiúsculas
	section=Title,		% títulos de seções convertidos em letras maiúsculas
	%subsection=TITLE,	% títulos de subseções convertidos em letras maiúsculas
	%subsubsection=TITLE,% títulos de subsubseções convertidos em letras maiúsculas
	% -- opções do pacote babel --
	english,			% idioma adicional para hifenização
	french,				% idioma adicional para hifenização
	spanish,			% idioma adicional para hifenização
	brazil,				% o último idioma é o principal do documento
]{abntex2}

% \usepackage{blindtext, showframe}

% \usepackage[a4paper,showframe,
% top    = 2.75cm,
% bottom = 2.50cm,
% left   = 3.00cm,
% right  = 2.50cm]{geometry}


%%%%%%%%%%%%%%%%%%%%%%%%%%%%%%%%%%%%%%%%%%%%%%%%%%%%%%%%%%%%


\newenvironment{conditions}[1][onde:]
  {\noindent
	#1
	
	\indent
	\begin{tabular}[t]{>{$}l<{$} @{${} \> \> {}$} l}}
	{\end{tabular}\\[\belowdisplayskip]}


% --- Controla a geração de listas de siglas, símbolos e glossário
% \RequirePackage[style=long]{glossaries}
% ,long,numberlist


\usepackage[acronyms,symbols,nonumberlist]{glossaries}
\setacronymstyle{long-short}

\makenoidxglossaries


%Definição das listas de glossários e parâmertos para compilação.
%-------------------------------------------------------
\newglossary{verbetes}{verbetes}{vrl}{verbetes}
\newglossary{siglas}{siglas}{sgl}{siglas}
\newglossary{simbolos}{simbolos}{sbl}{simbolos}


%Definição dos comandos para exibição das listas individuais.
%-----------------------------------------------------------
\newcommand{\listarsiglas}{\printglossary[type=siglas, title=\listadesiglasname, nonumberlist=true]}
\newcommand{\listarsimbolos}{\printglossary[type=simbolos, title=\listadesimbolosname, nonumberlist=true]}
\newcommand{\listarverbetes}{\printglossary[type=verbetes]}

%Definição dos Comandos para inclusão no texto das siglas.

\newcommand{\verbete}[3]{
	\newglossaryentry{glossary}{name=#1, description={#2}, plural={#3}, type=verbetes}
	\gls{#1}
}

\newcommand{\sigla}[2]{
	\newacronym[type=siglas]{#1}{#1}{#2}
	\gls{#1}
}

\newcommand{\simbolo}[3]{
	\newglossaryentry{#1}{type=simbolos, name=#3, text=#3, description=#2,symbol=#3, sort=def}
	\gls{#1}
}

%%%%%%%%%%%%%%%%%%%%%%%%%%%%%%%%%%%%%%%%%%%%%%%%%%%%%%%%%%%%

\RequirePackage{bookmark}   			



% ---------------------------------------------------------------------------------
%                                   PACOTES
% ---------------------------------------------------------------------------------


% ---
% Pacotes básicos 
% ---
\usepackage{cmap}
\usepackage{lmodern}

\usepackage{amsmath}
\usepackage{latexsym}
\usepackage{amsfonts}
\usepackage[normalem]{ulem}
\usepackage{array}
\usepackage{amssymb}
\usepackage{graphicx}

% \usepackage[backend=bibtex,
% style=authoryear,
% natbib=true,
% sorting=none,
% isbn=false,
% doi=false,
% url=false,

\usepackage{subfig}
\usepackage{wrapfig}
\usepackage{wasysym}
\usepackage{enumitem}
\usepackage{adjustbox}
\usepackage{ragged2e}
\usepackage[svgnames,table]{xcolor}
\usepackage{tikz}
\usetikzlibrary{intersections}
\usetikzlibrary{shapes,arrows,chains}
\tikzstyle{line}=[draw] % here
\usepackage{longtable}
\usepackage{changepage}
\usepackage{setspace}
\usepackage{hhline}
\usepackage{multicol}
\usepackage{tabto}
\usepackage{multirow}
\usepackage{makecell}
\usepackage{fancyhdr}

\usepackage{lmodern}			% Usa a fonte Latin Modern			
\usepackage[T1]{fontenc}		% Selecao de codigos de fonte.
\usepackage[utf8]{inputenc}		% Codificacao do documento (conversão automática dos acentos)
\usepackage{lastpage}			% Usado pela Ficha catalográfica
\usepackage{indentfirst}		% Indenta o primeiro parágrafo de cada seção.
\usepackage{xcolor,colortbl}	% Controle das cores
\usepackage{graphicx}			% Inclusão de gráficos
\usepackage{pgfplots}
\pgfplotsset{compat=1.17}
\usepackage{microtype} 			% para melhorias de justificação
\usepackage{hyperref}
\usepackage{subfig}
\usepackage{epigraph}
\usepackage{url}
\usepackage{placeins}
\usepackage{multirow}
\usepackage[figuresright]{rotating}
\usepackage{chemfig}
\usepackage{amsmath}
\usepackage{amssymb}
\usepackage{enumitem}
\usepackage{bigints}
\usepackage{listings}
\usepackage{etoolbox}
\usepackage[final]{pdfpages}
\usepackage{bigstrut}
\usepackage{makeidx}
\usepackage{float}

% ---



% ---
% Pacotes adicionais, usados apenas no âmbito do Modelo Canônico do abnteX2
% ---
\usepackage{lipsum}				% para geração de dummy text
% ---

% ---
% Pacotes de citações
% ---
\usepackage[brazilian,hyperpageref]{backref}	 % Paginas com as citações na bibl
% \usepackage[alf,abnt-emphasize=bf]{abntex2cite}  % Citações padrão ABNT
\usepackage[alf,abnt-etal-cite=2]{abntex2cite}  % Citações padrão ABNT

% ---------------------------------------------------------------------------------
%                          CONFIGURAÇÕES DOS PACOTES
% ---------------------------------------------------------------------------------

% ---
% Configurações do pacote backref
%
% Para desativar, tire o comentário de \begin{comment} e \end{comment} 
% das próximas linhas e comente a linha \usepackage[brazilian,hyperpageref]{backref}
% acima.
% ---

% \newcommand{\imprimirdata}{\data}


%\begin{comment}
% ---
% Configurações do pacote backref
% Usado sem a opção hyperpageref de backref
\renewcommand{\backrefpagesname}{Citado na(s) página(s):~}
% Texto padrão antes do número das páginas
\renewcommand{\backref}{}
% Define os textos da citação
\renewcommand*{\backrefalt}[4]{
	\ifcase #1 %
	Nenhuma citação no texto.%
	\or
	Citado na página #2.%
	\else
	Citado #1 vezes nas páginas #2.%
	\fi}%
% ---
%\end{comment}


% listagens
\definecolor{corComentario}{RGB}{150,150,150}
\definecolor{corString}{RGB}{206,123,0}
\definecolor{corPalavraChave}{RGB}{0,0,230}

\lstset{
	numbers=left,
	stepnumber=1,
	firstnumber=1,
	numberstyle=\footnotesize,
	extendedchars=true,
	breaklines=true,
	lineskip=0pt,
	frame=tb,
	basicstyle=\ttfamily\footnotesize,
	showstringspaces=false,
	stringstyle=\color{corString},
	commentstyle=\color{corComentario},
	keywordstyle=\color{corPalavraChave}
}

\newcommand{\graficoname}{Gráfico}
\newcommand{\graficorefname}{Gráfico}
\newcommand{\listofgraficosname}{Lista de gráficos}





% \newcolumntype{Y}{>{\centering\arraybackslash}X}

\newcommand{\ano}[1]{\def \oano {#1}}
\newcommand{\imprimirano}{\oano}

\newcommand{\mes}[1]{\def \omes {#1}}
\newcommand{\imprimirmes}{\omes}

\newcommand{\dia}[1]{\def \odia {#1}}
\newcommand{\imprimirdia}{\odia}

% \newcommand{\subtitulo}[1]{\def \osubtitulo {#1}}
% \newcommand{\imprimirsubtitulo}{\osubtitulo}

\newcommand{\area}[1]{\def \aarea {#1}}
\newcommand{\imprimirarea}{\aarea}

\newcommand{\disciplina}[1]{\def \adisciplina {#1}}
\newcommand{\imprimirdisciplina}{\adisciplina}

% \renewcommand{\coorientador}[1]{\def \ocoorientador {#1}}
% \renewcommand{\imprimircoorientador}{\ocoorientador}

\newcommand{\grau}[1]{\def \ograu {#1}}
\newcommand{\imprimirgrau}{\ograu}

\newcommand{\curso}[1]{\def \ocurso {#1}}
\newcommand{\imprimircurso}{\ocurso}

\newcommand{\campus}[1]{\def \ocampus {#1}}
\newcommand{\imprimircampus}{\ocampus}


% Área de Concentração: \imprimirarea}
% ---


% ---
% Configurações de aparência do PDF final
% ---

% alterando o aspecto da cor azul
\definecolor{blue}{RGB}{41,5,195}

% informações do PDF
\makeatletter
\hypersetup{
	%pagebackref=true,
	pdftitle={\@title}, 
	pdfauthor={\@author},
	pdfsubject={\imprimirpreambulo},
	pdfcreator={\@author},
	pdfkeywords={Palavra chave 1}{Palavra chave 2}{Palavra chave 3}{Palavra chave n}, 
	colorlinks=true,       		% false: boxed links; true: colored links
	linkcolor=blue,          	% color of internal links
	citecolor=blue,       		% color of links to bibliography
	filecolor=black,      		% color of file links
	urlcolor=blue,
	bookmarksdepth=4
}
\makeatother
% --- 


% ---
% Comandos do autor
% ---

% comando para inserir autor e ano
\newcommand{\citeauthorandyear}[1]{\citeauthoronline{#1} (\citeyear{#1})}

% --- 
% Espaçamentos entre linhas e parágrafos 
% --- 

% O tamanho do parágrafo é dado por:
\setlength{\parindent}{1.3cm}

% Controle do espaçamento entre um parágrafo e outro:
\setlength{\parskip}{0.2cm}  % tente também \onelineskip

\glsaddall

% ---
% compila os glossários e listas
% \makeglossaries

% ---
% compila o indice
% ---
\makeindex
% ---
%---


%%%%%%%%%%%%%%%%%%%%%%%%%%%%%%%%%%%%%%%%%%%%

% para o pacote "Lst Listing"

\newcommand{\source}[1]{\caption*{Fonte: {#1}}}

% Altera o nome padrão do rótulo usado no comando \autoref{}
\renewcommand{\lstlistingname}{Código}

% Altera o rótulo a ser usando no elemento pré-textual "Lista de código"
\renewcommand{\lstlistlistingname}{Lista de códigos}

%% Configuração para o ambiente de código (algorítmos)
\lstset{
%   numbers=left,
%   inputencoding=latin1,
%   basicstyle=\footnotesize\ttfamily,
%   keywordstyle=\color{blue},         
%   breaklines=true, 
%   showtabs=false,
%   showstringspaces=false,
%   numberstyle=\tiny\color{mygray}
   basicstyle=\fontsize{9}{11}\selectfont\ttfamily
}
%% 
%% exemplo da página https://latex.org/forum/viewtopic.php?t=24038
%%

% comandos para excluir items da table of contents (Sumário)
\let\oldaddcontentsline\addcontentsline
\newcommand{\hidefromtoc}{\renewcommand{\addcontentsline}[3]{}}
\newcommand{\writetotoc}{\let\addcontentsline\oldaddcontentsline}
\newacronym{csr}{CSR}{Client-Side Rendering}
\newacronym{ssr}{SSR}{Server-Side Rendering}
\newacronym{seo}{SEO}{Search Engine Optimization}
\newacronym{html}{HTML}{HyperText Markup Language}
\newacronym{css}{CSS}{Cascading Style Sheets}
\newacronym{spa}{SPA}{Single Page Application}
\newacronym{mpa}{MPA}{Multi Page Application}
\newacronym{dom}{DOM}{Document Object Model}
\newacronym{ux}{UX}{User Experience}
\newacronym{http}{HTTP}{Hypertext Transfer Protocol}
\newacronym{https}{HTTPS}{Hypertext Transfer Protocol Secure}
\newacronym{tls}{TLS}{Transport Layer Security}
\newacronym{ssl}{SSL}{Secure Sockets Layer}
\newacronym{quic}{QUIC}{Quick UDP Internet Connections}
\newacronym{tcp}{TCP}{Transmission Control Protocol}
\newacronym{udp}{UDP}{User Datagram Protocol}
\newacronym{isr}{ISR}{Incremental Static Regeneration}
\newacronym{ssg}{SSG}{Static Site Generation}
\newacronym{dsg}{DSG}{Deferred Static Generation}
\newacronym{rum}{RUM}{Real User Monitoring}
\newacronym{ttfb}{TTFB}{Time to First Byte}
\newacronym{fcp}{FCP}{First Contentful Paint}
\newacronym{lcp}{LCP}{Largest Contentful Paint}
\newacronym{inp}{INP}{Interaction to Next Paint}
\newacronym{cls}{CLS}{Cumulative Layout Shift}
\newacronym{fid}{FID}{First Input Delay}



% ---------------------------------------------------------------------
% Informações de dados para CAPA, FOLHA DE ROSTO e FOLHA DE ASSINATURAS
% ---------------------------------------------------------------------

% ---------------------------------------------------------------------
% Alterar os dados abaixo com os seus dados
%
% obs: Os nomes e instituições dos membros da banca deverão ser
%       alterados no arquivo 03-assinaturas.tex 
% ---------------------------------------------------------------------

\curso{Bacharelado em Sistemas de Informação}
\grau{Bacharel em Sistemas de Informação} 
\titulo{Uma Análise Comparativa entre \english{Client-Side Rendering} e \english{Server-Side Rendering} em Aplicações Web}
\tipotrabalho{Trabalho de Conclusão}
\area{Tecnologia da Informação}

\autor{Hiago de Oliveira Mendes e Lucas Sales Salvo Petruci}
\orientador{Prof. D.Sc. Ronaldo Amaral Santos}

% caso não haja coorientador, comente a linha abaixo
% \coorientador{Profa. M.Sc. Nome da Professora Coorientadora}

\local{Campos dos Goytacazes-RJ}
\dia{31}
\mes{Setembro}
\ano{2025}
\data{Setembro de \imprimirano}

\instituicao{Instituto Federal de Educação, Ciência e Tecnologia Fluminense}
\campus{Campos-Centro}

\preambulo{\imprimirtipotrabalho\ apresentado ao curso \imprimircurso~ do \imprimirinstituicao, como parte dos requisitos para a obtenção do título de \imprimirgrau.}

% ---------------------------------------------------------------------------------
%                                   INÍCIO DO DOCUMENTO
% ---------------------------------------------------------------------------------

\begin{document}

\pretextual

\begin{capa}
\imprimircapa 
\end{capa}


\input{pre/02-rosto}


% ---
% Inserir folha de aprovação
% ---

% Isto é um exemplo de Folha de aprovação, elemento obrigatório da NBR
% 14724/2011 (seção 4.2.1.3). Você pode utilizar este modelo até a aprovação
% do trabalho. Após isso, substitua todo o conteúdo deste arquivo por uma
% imagem da página assinada pela banca com o comando abaixo:
%
% \begin{folhadeaprovacao}
% 	\includepdf{pre/FolhaAprovacaoAssinada}
% \end{folhadeaprovacao}

\begin{folhadeaprovacao}

\setlength{\ABNTEXsignwidth}{14cm}

    \begin{center}
    {\ABNTEXchapterfont\large\imprimirautor}

    \vspace*{\fill}\vspace*{\fill}
    {\ABNTEXchapterfont\bfseries\Large\imprimirtitulo}
    \vspace*{\fill}
    
    \hspace{.45\textwidth}
    \begin{minipage}{.5\textwidth}
        \imprimirpreambulo
    \end{minipage}%
    \vspace*{\fill}
   
   \end{center}
 
 
   \begin{center}
    \imprimirlocal, \imprimirdia ~de \imprimirmes ~de \imprimirano.
   \end{center}
   
   % Instituto Federal Fluminense (IFF)
   \assinatura{\textbf{\imprimirorientador \, (orientador)} \\ Instituto Federal Fluminense (IFF)}
   \assinatura{\textbf{Prof. D.Sc. banca 1} \\ Instituto Federal Fluminense (IFF)}
   \assinatura{\textbf{Prof. D.Sc. banca 2} \\ Instituto Federal Fluminense (IFF)}

   \begin{center}
    \vspace*{0.5cm}
    {\ABNTEXchapterfont\large\imprimirlocal}
    \par
    {\ABNTEXchapterfont\large\imprimirdata}
    \vspace*{1cm}
  \end{center}
  
\end{folhadeaprovacao}


\begin{agradecimentos}

Em primeiro lugar, agradecemos a Deus, por nos dar força, fé e sabedoria para superar todos os desafios e dificuldades ao longo de nossa jornada acadêmica.

Agradecemos de maneira profunda e especial às nossas famílias, nossos alicerces. Em especial às nossas mães, que sempre nos apoiaram com amor incondicional em cada passo desta jornada. Somos imensamente gratos por todo o sacrifício, carinho e dedicação. Em particular, um de nós presta uma homenagem carinhosa à sua bisavó, cuja memória e legado serviram como uma silenciosa e constante fonte de inspiração. Se hoje celebramos esta conquista, é porque tivemos o privilégio de contar com o suporte de vocês.

Ao nosso orientador, Prof. Dr. Ronaldo Amaral Santos, expressamos nossa mais profunda gratidão. Seu apoio, orientação precisa e ensinamentos foram fundamentais não apenas para a condução deste trabalho, mas para nosso crescimento como pesquisadores. Agradecemos por sua paciência, dedicação e pelo conhecimento compartilhado.

Gostaríamos de registrar um agradecimento especial ao Prof. Dr. Rogerio Atem. A oportunidade de participar dos projetos de bolsa sob sua mentoria no polo de inovação nos proporcionou nossas primeiras experiências práticas, sendo um passo fundamental para a aplicação dos conhecimentos teóricos e para o despertar de nossa paixão pela tecnologia. Sua confiança foi essencial em nosso início de carreira.

Agradecemos a todos os professores do Instituto Federal Fluminense que, ao longo do curso, compartilharam seus conhecimentos e nos inspiraram, bem como aos nossos colegas e amigos, que tornaram a jornada mais leve com momentos de companheirismo e apoio mútuo.

Por fim, esta conquista reforça nossa crença de que a educação é a chave não apenas para atingir objetivos pessoais, mas para transformar a sociedade em um lugar mais próspero, solidário e justo.

\end{agradecimentos}

\begin{resumo}

% O comando lipsum abaixo é um gerador automático de texto.
% Substitua-o pelo texto do seu resumo.
% Lembre-se: Um resumo deve ser um parágrafo único que apresente os seguintes tópicos:

% Contexto;
% Problema;
% Objetivo;
% Justificativa;
% Metodologia;
% Resultado;
% Conclusão.

% O crescimento exponencial da web, com milhares de sites surgindo diariamente, tem elevado a complexidade do desenvolvimento de aplicações web e evidenciado a importância da escolha adequada da estratégia de renderização de conteúdo. Dentre as principais abordagens estão o \textit{Client-Side Rendering} (CSR) e o \textit{Server-Side Rendering} (SSR), cada uma com características distintas que impactam diretamente a performance, a experiência do usuário (UX), a escalabilidade e a otimização para motores de busca (SEO). Este trabalho tem como objetivo principal realizar uma análise comparativa entre CSR e SSR, destacando seus efeitos técnicos e funcionais em aplicações web modernas. A justificativa fundamenta-se na lacuna de estudos de caso práticos e aprofundados, especialmente no contexto nacional, que analisem criticamente os impactos reais da adoção dessas abordagens. A metodologia adotada envolveu fundamentação teórica, mapeamento sistemático da literatura com uso da técnica \textit{PICOC}, e a implementação de um estudo de caso realista, com coleta de métricas de desempenho, tempo de carregamento, impacto em SEO e UX. Os resultados apontaram que, embora o CSR seja mais vantajoso em aplicações altamente interativas, o SSR apresenta melhor desempenho inicial e maior indexabilidade. Conclui-se que a escolha entre CSR e SSR deve considerar o perfil do sistema, a infraestrutura disponível e os objetivos estratégicos do projeto, sendo muitas vezes recomendável a adoção de soluções híbridas.

% \textbf{Palavras-chave: } Renderização do lado do cliente, CSR, Renderização do lado do servidor, SSR, Renderização Web, Desempenho, SEO, UX.

% \end{resumo}


% \begin{resumo}
Com a crescente complexidade das aplicações web e a busca por experiências de usuário ricas e performáticas, a escolha da estratégia de renderização de conteúdo tornou-se uma decisão arquitetural fundamental. Nesse cenário, o \textit{Client-Side Rendering} (CSR) e o \textit{Server-Side Rendering} (SSR) emergem como as principais abordagens, cada uma com \textit{trade-offs} significativos: o CSR favorece a interatividade contínua, enquanto o SSR otimiza o carregamento inicial e a indexação por motores de busca (SEO). A literatura, no entanto, carece de estudos de caso práticos que comparem seus impactos de forma direta e controlada.

Visando preencher essa lacuna, este trabalho apresenta uma análise comparativa detalhada, fundamentada em um estudo de caso realista: o desenvolvimento de uma plataforma de notícias em duas versões funcionalmente idênticas, uma com React (CSR) e outra com Next.js (SSR). Por meio da coleta de métricas de \textit{Core Web Vitals} em um ambiente controlado com contêineres Docker, a pesquisa avalia empiricamente os efeitos de cada arquitetura no desempenho e na experiência do usuário. O estudo busca, assim, oferecer uma contribuição clara para a compreensão dos cenários de aplicação de cada abordagem, auxiliando equipes de desenvolvimento a tomarem decisões mais informadas e estratégicas.

\textbf{Palavras-chave:} Renderização do lado do cliente, CSR, Renderização do lado do servidor, SSR, Renderização Web, Desempenho, SEO, UX.
\end{resumo}


\begin{resumo}[Abstract]
\begin{otherlanguage*}{english}
With the growing complexity of web applications and the demand for rich, high-performance user experiences, the choice of a content rendering strategy has become a fundamental architectural decision. In this context, \textit{Client-Side Rendering} (CSR) and \textit{Server-Side Rendering} (SSR) emerge as the main approaches, each with significant \textit{trade-offs}: CSR favors continuous interactivity, while SSR optimizes initial load times and search engine optimization (SEO). The literature, however, lacks practical case studies that directly and controllably compare their impacts.

To address this gap, this work presents a detailed comparative analysis based on a realistic case study: the development of a news platform in two functionally identical versions, one with React (CSR) and the other with Next.js (SSR). Through the collection of \textit{Core Web Vitals} metrics in a controlled environment using Docker containers, the research empirically evaluates the effects of each architecture on performance and user experience.

The empirical results revealed a clear trade-off: the SSR architecture demonstrated superiority in initial loading (TTFB and FCP) and SEO, while the CSR approach excelled in fluid interactivity (INP) and reduced server cost, although it exhibited greater visual instability (CLS). It is concluded that the choice is strategic and depends on the project's objectives: SSR is recommended for applications focused on content and discovery, such as portals and e-commerces, while CSR is suited for interaction-rich systems, such as dashboards and logged-in platforms. As a limitation, the study was conducted in a local environment, not encompassing the variability of real-world networks and devices.

\textbf{Keywords:} Client-side rendering, CSR, Server-side rendering, SSR, Web Rendering, Performance, SEO, UX.
\end{otherlanguage*}
\end{resumo}

\input{pre/07-figuras}

% não é necessário alterar este arquivo


\hidefromtoc
\listofquadros
\writetotoc
\clearpage

% \input{pre/09-tabelas}

% não é necessário alterar este arquivo

\listofcodigos*




% \printglossaries
\printglossary

\setglossarypreamble[acronym]{%
	\glsresetentrycounter
}
\setglossarystyle{long3col}
\printnoidxglossary[type=acronym]

\clearpage


\input{pre/11-simbolos}

 %%%%%%%%%%%%  This Produces Table Of Contents %%%%%%%%%%%%%%

\vspace{\baselineskip}
\setlength{\parskip}{0.0pt}

\begin{Center}
\end{Center}\par


\vspace{\baselineskip}
\setlength{\parskip}{9.96pt}


\tableofcontents*

\clearpage

 %%%%%%%%%%%%  Starting New Page here %%%%%%%%%%%%%%


\textual

\chapter{Introdução}
\label{cap:introducao}

\section{Problema e contexto}
O crescimento acelerado da web e o aumento da complexidade das aplicações modernas impuseram novos desafios ao desenvolvimento e à entrega de conteúdos na internet. Com o crescimento exponencial da web, estima-se que aproximadamente 252 mil novos sites sejam desenvolvidos diariamente, demonstrando não apenas a rapidez com que aplicações são criadas, mas também a necessidade crescente de estratégias eficientes para otimização de desempenho e escalabilidade \cite{dataInternetUsage}. A escolha da abordagem de renderização tornou-se um fator determinante para a experiência do usuário e a escalabilidade dos sistemas. Inicialmente, os sites eram compostos por páginas estáticas, cujo conteúdo era carregado diretamente do servidor. Com a evolução das tecnologias frontend, novas abordagens surgiram, destacando-se \english{\acrfull{csr}} e \english{\acrfull{ssr}}. Cada uma dessas técnicas possui características específicas que influenciam diretamente o desempenho e a experiência do usuário.

A performance em websites é um fator determinante para o sucesso de qualquer aplicação web. O desempenho, frequentemente medido pelo tempo de carregamento das páginas, desempenha um papel fundamental na experiência do usuário e na taxa de conversão de visitantes \cite{webPerformance}. Uma página que carrega rapidamente proporciona uma navegação mais fluida, reduzindo a taxa de rejeição e aumentando a retenção de usuários. Além disso, o desempenho da página não se limita a impactar a experiência do usuário, mas também interfere diretamente no \english{\acrfull{seo}}, tornando-se um critério essencial de indexação e ranqueamento em plataformas como o Google \cite{google}.

Um exemplo notável de desafios enfrentados na escolha da estratégia de renderização ocorreu no \emph{Twitter}. Em 2010, a empresa lançou uma nova versão de sua plataforma, conhecida como New Twitter, que utilizava extensivamente a renderização no lado do cliente (\acrshort{csr}) para aprimorar a interatividade e a experiência do usuário. No entanto, essa abordagem resultou em problemas significativos de desempenho, especialmente para usuários com conexões de internet mais lentas ou dispositivos menos potentes. Além disso, a dependência intensa de JavaScript dificultou a indexação de conteúdo pelos mecanismos de busca, impactando negativamente a otimização para motores de busca (\acrshort{seo}) \cite{twitter}. Reconhecendo essas limitações, o Twitter decidiu retornar à renderização no lado do servidor (\acrshort{ssr}) em 2012, visando melhorar o desempenho e a acessibilidade de sua plataforma.

A arquitetura de frontend desempenha papel fundamental ao definir o fluxo de desenvolvimento e a escolha entre \acrshort{csr} e \acrshort{ssr}, sendo indispensável a adoção de um sistema modular e eficiente, capaz de ser mantido e escalado de forma sustentável \cite{frontendGodbolt}. Na abordagem \acrshort{csr}, a renderização ocorre diretamente no navegador do usuário, reduzindo a carga no servidor, mas exigindo mais processamento no cliente; já na \acrshort{ssr}, o conteúdo é gerado no servidor antes de ser enviado ao cliente, o que proporciona carregamento mais rápido e melhor desempenho em dispositivos menos potentes. A decisão entre essas estratégias está diretamente ligada à performance da aplicação e deve considerar fatores como tempo de carregamento, complexidade da página e número de requisições HTTP \cite{webPerformance}, já que diferentes abordagens afetam não apenas a experiência do usuário, mas também os custos operacionais e a infraestrutura necessária para suportar a aplicação.

\section{Justificativa}

Nos últimos anos, observou-se um notável crescimento na adoção das abordagens de renderização no lado do cliente (\acrshort{csr}) e no lado do servidor (\acrshort{ssr}) no desenvolvimento de aplicações web. Contudo, essas técnicas são frequentemente empregadas de maneira inadequada em muitos projetos, seja pela falta de entendimento de suas vantagens e limitações, seja por uma análise superficial das necessidades do produto \cite{atori2024}. Um exemplo ilustrativo dessa realidade pode ser visto na experiência do \emph{Airbnb}, que optou por uma abordagem de \acrshort{ssr} com o intuito de melhorar o desempenho em dispositivos com recursos limitados e, sobretudo, otimizar a indexação de seu vasto catálogo de acomodações em mecanismos de busca \cite{neary2017}. Por outro lado, a equipe do \emph{Instagram} enfrentou desafios ao equilibrar o carregamento dinâmico de conteúdo no cliente com a necessidade de garantir uma experiência fluida aos usuários, levando-os a adotar soluções híbridas que envolvem tanto \acrshort{csr} quanto \acrshort{ssr} em diferentes partes da aplicação \cite{conner2019}.

Paralelamente a esses casos, identifica-se uma carência de estudos de caso reais que analisem de forma aprofundada o impacto da adoção de \acrshort{csr} e \acrshort{ssr}, principalmente no contexto nacional. Enquanto algumas publicações se concentram em apenas uma dessas abordagens, outras fornecem exemplos excessivamente simplificados, limitando a compreensão dos desafios técnicos e de negócios ao combinar essas estratégias em sistemas complexos.

Diante desse cenário, o presente trabalho visa suprir essa lacuna ao oferecer uma análise comparativa detalhada sobre a implementação de casos com \acrshort{csr} e \acrshort{ssr}, avaliando seus efeitos no desempenho, na experiência do usuário, segurança, otimização do \acrshort{seo} e na escalabilidade de aplicações web modernas. Por meio de um estudo de caso abrangente, pretende-se não apenas enriquecer a literatura acadêmica, mas também fornecer insights práticos que possam orientar equipes de desenvolvimento e gestores na seleção e aplicação adequada dessas técnicas, contribuindo para a construção de sistemas mais robustos, eficientes e alinhados às demandas de um mercado em constante evolução.

\section{Objetivos}

\subsection{Objetivo Geral}

O objetivo geral deste trabalho de conclusão de curso é apresentar uma análise comparativa detalhada sobre a implementação de casos com \acrshort{csr} e \acrshort{ssr}, avaliando seus efeitos no desempenho, na experiência do usuário, segurança, otimização do \acrshort{seo} e na escalabilidade de aplicações web modernas. 

\subsection{Objetivos Específicos}
\begin{itemize}
\item Apresentar estratégias de escolhas entre \acrshort{csr} e \acrshort{ssr}, analisando métricas de desempenho, tempo de resposta, experiência do usuário e carga no servidor.
\item Identificar as principais limitações e desafios enfrentados na escolha entre \acrshort{csr} e \acrshort{ssr}, considerando otimização de \acrshort{seo}, escalabilidade e requisitos de infraestrutura.
\item Apresentar recomendações práticas para desenvolvedores e gestores, auxiliando na tomada de decisão sobre qual abordagem utilizar com base nos objetivos do projeto e nas demandas do mercado.
\end{itemize}

\section{Metodologia}
Pode-se observar na \autoref{fig:metodo_recurso} as etapas de execução dessa pesquisa. Inicialmente, o escopo é definido e o primeiro capítulo é elaborado. Nesse momento, busca-se apresentar o contexto do estudo, as justificativas e os objetivos a serem atingidos. Em seguida, a fundamentação teórica com os conveitos-chaves é construída, visando fornecer uma base sólida para o desenvolvimento do estudo de caso.
Posteriormente, é realizado um mapeamento da literatura buscando trabalhos similares. Esse mapeamento é essencial para identificar lacunas no conhecimento e definir o escopo do estudo de caso. Além disso, ele permite a identificação de desafios, práticas e padrões comuns na utilização de \acrshort{csr} e \acrshort{ssr}.
Por fim, um estudo de caso realista é desenvolvido, explorando a implementação prática de ambas as abordagens. Nesse estudo, são apresentados os requisitos, métodos e organização do sistema, além de trechos de código relevantes para demonstrar a aplicação das técnicas. Além disso, são realizados testes de desempenho para avaliar o impacto de \acrshort{csr} e \acrshort{ssr} em termos de tempo de resposta, carga no servidor, experiência do usuário e otimização para \acrshort{seo}.
Finalmente, um relatório final é elaborado, contendo análises quantitativas e qualitativas dos resultados obtidos, além de uma discussão sobre trabalhos futuros, desafios e benefícios do uso de \acrshort{csr} e \acrshort{ssr} em aplicações web modernas.

\begin{figure}[h!]
    \centering
    \caption{Etapas de desenvolvimento da pesquisa}
    \includegraphics[width=0.9\textwidth]{media/bpmn_metodo_recurso.png}
    \legend{Fonte: os autores}
    \label{fig:metodo_recurso}
\end{figure}

\section{Estrutura do Trabalho}
Este trabalho está dividido em sete capítulos. O \autoref{cap:introducao} expõe o contexto do estudo, as justificativas desta pesquisa e os objetivos a serem atingidos. O \autoref{cap:fundamentacao} apresenta conceitos fundamentais sobre \acrshort{csr} e \acrshort{ssr}, abordando suas principais características, vantagens e desafios. O \autoref{cap:trabalhos} expõe o protocolo e o resultado do mapeamento da literatura, analisando estudos relacionados e identificando lacunas no conhecimento sobre a adoção dessas abordagens. Da mesma forma, o \autoref{cap:estudo_caso1} descreve os requisitos, métodos e organização do estudo de caso. Em seguida, o \autoref{cap:estudo_caso2} apresenta o estudo de caso desenvolvido, incluindo o \english{design} do sistema e trechos de código chave da implementação. Posteriormente, o \autoref{cap:resultados} apresenta os resultados obtidos com a execução dos testes de desempenho, analisando métricas como tempo de resposta, consumo de recursos, impacto no \acrshort{seo} e experiência do usuário. Por fim, o \autoref{cap:conclusão} apresenta as conclusões obtidas com o desenvolvimento deste trabalho, destacando os principais achados, desafios e recomendações para a escolha entre \acrshort{csr} e \acrshort{ssr} em aplicações web modernas.

\chapter{Fundamentação Teórica}
\label{cap:fundamentacao}


\section{\english{S1}} 

\subsection{\textit{Aggregate}}
\label{s1}

\section{S2} 
\label{s2}




\chapter{Metodologia}
\label{cap:metodologia}

\section{Levantamento Teórico}
\label{sec:levantamento-teorico}

O levantamento teórico consistiu na revisão e sistematização dos principais conceitos, tecnologias e práticas relacionadas à renderização de conteúdo em aplicações web, com foco nas abordagens de \english{\acrfull{csr}} e \english{\acrfull{ssr}}. Essa etapa teve como objetivo fornecer o embasamento necessário para o desenvolvimento do estudo de caso e da análise comparativa proposta neste trabalho.

A fundamentação iniciou-se com a exploração dos princípios do desenvolvimento web moderno, incluindo a arquitetura cliente-servidor, o funcionamento do protocolo \acrshort{http} e as tecnologias essenciais do \textit{frontend}: \acrshort{html}, \acrshort{css} e JavasSript. Estes elementos formam a base para compreender como as estratégias de renderização operam, tanto no lado do cliente quanto no lado do servidor.

Em seguida, foram estudadas em detalhes as abordagens \acrshort{csr} e \acrshort{ssr}. A renderização no lado do cliente (\acrshort{csr}) foi analisada quanto ao seu funcionamento típico em aplicações \english{\acrfull{spa}}, caracterizadas por uma única página carregada inicialmente, com atualizações dinâmicas de conteúdo via JavaScript. Essa abordagem oferece vantagens como maior interatividade e fluidez na navegação, além de transições rápidas entre páginas internas. No entanto, apresenta desvantagens como maior tempo de carregamento inicial e limitações de indexação por mecanismos de busca.

Por outro lado, a renderização no lado do servidor (\acrshort{ssr}) foi abordada sob a ótica de desempenho inicial otimizado e maior compatibilidade com \acrshort{seo}, pois o conteúdo é entregue já renderizado ao navegador. Essa abordagem é comumente utilizada em aplicações do tipo \english{\acrfull{mpa}}, que possuem múltiplas páginas distintas e se beneficiam da pré-renderização para melhorar a performance inicial, a acessibilidade e a visibilidade em mecanismos de busca. Como contraponto, o SSR demanda maior processamento no servidor e pode aumentar a complexidade da infraestrutura.

Além dessas duas abordagens principais, o estudo também incluiu métodos híbridos como o \english{\acrfull{ssg}}, \english{\acrfull{isr}} e {\acrfull{dsg}}, que visam equilibrar performance, escalabilidade e atualizações de conteúdo dinâmico, especialmente em aplicações que exigem alta eficiência e atualizações frequentes.

O levantamento teórico foi complementado por uma análise dos principais \textit{frameworks} e bibliotecas utilizados no desenvolvimento web atual, como \textit{React}, \textit{Vue.js}, \textit{Angular}, \textit{Next.js}, \textit{Nuxt} e \textit{SvelteKit}, e por uma discussão sobre os impactos de cada abordagem na experiência do usuário (\acrshort{ux}), incluindo aspectos como \acrshort{seo}, acessibilidade, tempo de carregamento e interatividade.

Esse base teórica serviu para as próximas etapas da pesquisa, em especial o mapeamento sistemático da literatura e a condução do estudo de caso prático.

\section{Mapeamento Sistemático da Literatura}
\label{sec:mapemento-sistematico-da-literatura}

O mapeamento sistemático da literatura teve como objetivo identificar, selecionar e analisar estudos acadêmicos relevantes que abordassem comparações entre as abordagens de renderização \english{\acrshort{csr}} e \english{\acrshort{ssr}} no contexto do desenvolvimento de aplicações web. Essa etapa foi essencial para compreender o estado da arte, bem como identificar lacunas e oportunidades para a realização do estudo de caso proposto neste trabalho.

A estratégia de busca foi estruturada com o apoio da metodologia \textit{PICOC}, que define os elementos População, Intervenção, Comparação, Resultado e Contexto, com o intuito de guiar a construção das expressões de busca e garantir abrangência e precisão nos resultados. As principais bases de dados utilizadas incluíram Periódicos Capes e Scopus, por oferecerem amplo acervo e suporte a pesquisas refinadas.

Foram utilizadas expressões booleanas combinando termos como \textit{Client-Side Rendering}, \textit{Server-Side Rendering}, \textit{Web Performance}, \textit{SEO}, \textit{UX} e \textit{Frontend Architecture}. Após a aplicação dos critérios de inclusão e exclusão, os artigos resultantes foram classificados e analisados de acordo com sua relevância, tipo de abordagem estudada, metodologias utilizadas e principais conclusões.

A seleção final contemplou trabalhos que abordavam métricas de desempenho, tempo de carregamento, interatividade, \acrshort{seo} e experiência do usuário. Além disso, foram considerados estudos que analisavam o uso de frameworks modernos como \textit{React}, \textit{Next.js}, \textit{Nuxt.js} e \textit{Angular Universal}, além de pesquisas aplicadas em contextos reais de produção.

Como resultado, foi possível consolidar uma visão abrangente sobre os desafios, vantagens e limitações de cada abordagem, fornecendo subsídios importantes para a execução do estudo de caso prático apresentado nos capítulos seguintes. O mapeamento sistemático também evidenciou a escassez de estudos nacionais aplicados ao tema, reforçando a relevância deste trabalho no cenário acadêmico e profissional brasileiro.


\section{Estudo de Caso Prático}
\label{sec:estudo-de-caso-pratico}

O escopo da aplicação desenvolvida visa avaliar os impactos das abordagens de renderização \english{\acrshort{csr}} e \english{\acrshort{ssr}} no desenvolvimento de aplicações web. Para isso, considera-se o contexto de uma empresa fictícia cujo time de gestores precisa decidir qual modelo de renderização adotar para a criação de sua nova aplicação web. Desenvolvem-se duas versões de uma mesma aplicação: uma utilizando \acrshort{csr} e outra baseada em \acrshort{ssr}. Ambas as implementações possuem as mesmas funcionalidades, aparência visual e estrutura de dados, de modo a permitir uma análise equitativa quanto ao desempenho, à experiência do usuário e à otimização para mecanismos de busca.

A aplicação simula um catálogo de produtos, com navegação por páginas, carregamento de dados via \textit{API} e exibição de informações detalhadas. Esse modelo é escolhido por representar um cenário comum na web moderna, abrangendo interações usuais como carregamento dinâmico de conteúdo, roteamento entre páginas e exibição de listas e detalhes.

Cada versão é implementada conforme os princípios da respectiva abordagem de renderização: a versão \acrshort{csr} tem a interface renderizada predominantemente no navegador do usuário, enquanto a versão \acrshort{ssr} conta com o conteúdo renderizado no servidor e enviado ao cliente já montado.

Durante o desenvolvimento, seguem-se boas práticas de acessibilidade, responsividade e otimização para \acrshort{seo}, garantindo que ambas as versões possam ser avaliadas com base em critérios equivalentes. As métricas analisadas incluem tempo de carregamento, tempo até a interatividade, consumo de recursos, desempenho percebido, qualidade do código e compatibilidade com ferramentas de análise de \acrshort{seo}.

A coleta de dados ocorre por meio de ferramentas como Google Lighthouse, PageSpeed Insights e WebPageTest, além de testes manuais com usuários, a fim de observar qualitativamente a experiência de uso. Esses dados fundamentam a análise comparativa e a discussão dos resultados, que são apresentados nas seções seguintes.

\section{Coleta de Dados e Testes}
\label{sec:coleta-de-dados-e-testes}

Esta seção descreve o procedimento de coleta e análise dos dados obtidos a partir da comparação entre as aplicações desenvolvidas com \english{\acrshort{csr}} e \english{\acrshort{ssr}}. O objetivo é mensurar o desempenho, a eficiência e a experiência do usuário proporcionada por cada aplicação sob condições controladas.

\subsection{Definição das Métricas}

As métricas utilizadas para a avaliação foram selecionadas com base nas recomendações do Google e nos indicadores mais relevantes para mensurar a performance e a experiência do usuário em aplicações web modernas. São elas:

\begin{itemize}
    \item \textbf{Time to First Byte (TTFB)}: Tempo até o primeiro byte da resposta ser recebido;
    \item \textbf{First Contentful Paint (FCP)}: Tempo até a exibição do primeiro conteúdo visível;
    \item \textbf{Largest Contentful Paint (LCP)}: Tempo até o carregamento do maior bloco de conteúdo;
    \item \textbf{Cumulative Layout Shift (CLS)}: Medida de estabilidade visual da interface;
    \item \textbf{Interaction to Next Paint (INP)}: Latência de interação (Core Web Vital atual para responsividade);
\end{itemize}

\noindent
Adicionalmente, foram observados: \textbf{número de requisições HTTP} e \textbf{uso de cache} (no cliente), bem como \textbf{consumo de recursos} do contêiner (CPU e memória) durante os testes. O \textbf{Time to Interactive (TTI)} foi empregado apenas como métrica de \textit{laboratório} via Lighthouse, não integrando o conjunto atual de Core Web Vitals. O tratamento estatístico detalhado das métricas será apresentado posteriormente.

\subsection{Ferramentas de Teste}

Foram utilizadas ferramentas complementares, com ênfase na coleta contínua no navegador (\textit{field}) e apoio de auditoria em \textit{laboratório}:

\begin{itemize}
    \item \textbf{Web Vitals (\acrfull{rum})}: Instrumentação no cliente para TTFB, FCP, LCP, CLS e INP, com envio via \texttt{navigator.sendBeacon} a um endpoint interno e persistência em formato NDJSON;
    \item \textbf{Google Lighthouse} (auxiliar): Auditoria em \textit{laboratório} para Performance (incluindo FCP, LCP, CLS, Speed Index e TBT), executada localmente sobre os serviços em Docker;
    \item \textbf{Chrome DevTools}: Inspeção de rede, \textit{waterfall}, cache e verificação dos \textit{POSTs} de métricas;
    \item \textbf{docker stats}: Acompanhamento do uso de CPU e memória dos contêineres durante a execução dos testes.
\end{itemize}

\subsection{Ambiente de Testes}

Os testes foram realizados em \textbf{ambiente local controlado} com contêineres \textit{Docker} para as duas versões (\acrshort{csr} e \acrshort{ssr}). Essa decisão decorre de limitações práticas da \textbf{NewsAPI} em produção ( políticas de CORS/uso do plano), e visa garantir controle experimental e reprodutibilidade. Ambos os serviços foram executados com \textbf{paridade de recursos}:

\begin{itemize}
    \item \textbf{CPU}: \texttt{--cpus="1.0"} e \texttt{--cpuset-cpus="0"};
    \item \textbf{Memória}: \texttt{--memory="1g"} e \texttt{--memory-swap="1g"};
    \item \textbf{Sistema de arquivos}: \texttt{--read-only} com \texttt{--tmpfs /tmp};
    \item \textbf{Persistência de métricas}: volume em \texttt{/data} com \texttt{METRICS\_PATH=/data/webvitals.ndjson}.
\end{itemize}

\noindent
A aplicação \acrshort{ssr} (Next.js) foi empacotada em modo \textit{standalone}. A aplicação \acrshort{csr} (React + Vite) foi servida por um processo \texttt{Node.js} simples que também expõe o endpoint de métricas. Os serviços públicos locais utilizados nos testes foram:
\begin{itemize}
    \item \textbf{SSR/Next.js}: \texttt{http://localhost:3001}
    \item \textbf{CSR/React}: \texttt{http://localhost:3002}
\end{itemize}

\subsection{Execução dos Testes}

A execução foi conduzida da seguinte forma:
\begin{enumerate}
    \item \textbf{Aquecimento}: duas visitas iniciais à mesma rota em cada aplicação, para estabilização de caches e recursos;
    \item \textbf{Coleta principal (Web Vitals)}: navegação real em janela anônima, registrando TTFB, FCP, LCP, CLS e INP por meio do endpoint interno e persistindo em arquivo NDJSON;
    \item \textbf{Coleta auxiliar (Lighthouse)}: auditorias repetidas em modo desktop, com parâmetros de \textit{throttling} consistentes entre cenários, para fornecer referência de \textit{laboratório};
    \item \textbf{Observabilidade do contêiner}: monitoramento pontual com \texttt{docker stats} para CPU e memória durante as execuções.
\end{enumerate}

\noindent
Para reduzir variabilidade, cada teste foi repetido \textbf{no mínimo cinco vezes} por cenário. A quantidade total de repetições e o método de agregação (mediana/p50 e p95) são detalhados em seção específica de resultados.

\subsection{Registro e Organização dos Dados}

Os dados coletados pela instrumentação (\acrshort{rum}) foram armazenados em arquivos \textbf{NDJSON} separados por abordagem, contendo os campos de identificação da métrica, valor e carimbo temporal. As auditorias do Lighthouse foram salvas em arquivos JSON/HTML para futura referência. Em seguida, os dados foram consolidados em planilhas, organizados por data, métrica e abordagem (CSR/SSR), para posterior análise comparativa.

\subsection{Método de Análise}

Os valores foram analisados por meio de \textbf{estatísticas descritivas}, com foco em \textbf{mediana (p50)} e \textbf{p95}, além de médias e desvios padrão quando apropriado. As comparações entre \acrshort{csr} e \acrshort{ssr} foram conduzidas métrica a métrica (TTFB, FCP, LCP, CLS e INP), considerando os \textit{trade-offs} de cada abordagem. Gráficos e tabelas de síntese são apresentados na seção de Resultados e Discussões.


\chapter{Trabalhos Relacionados}
\label{cap:trabalhos}
Este capítulo apresenta os trabalhos relacionados ao objeto de pesquisa, obtidos por meio de um mapeamento sistemático da literatura. O objetivo desse mapeamento foi identificar, analisar e sintetizar estudos acadêmicos que abordam comparações entre as abordagens de renderização \acrshort{csr} (Client-Side Rendering) e \acrshort{ssr} (Server-Side Rendering) no contexto do desenvolvimento de aplicações web. Buscou-se compreender como essas estratégias impactam aspectos como desempenho, tempo de carregamento, \acrshort{seo}, experiência do usuário e escalabilidade. O protocolo adotado, descrito nas seções seguintes, foi elaborado para garantir a abrangência, a precisão e a relevância dos resultados encontrados.


\section{Questões de pesquisa}
\label{section:questoes_pesquisa}
\begin{enumerate}
    \item[Q1:] De que maneira a escolha entre \acrshort{csr} e \acrshort{ssr} influenciam a experiência do usuário?
    \item[Q2:] Como as abordagens \acrshort{csr} e \acrshort{ssr} afetam métricas de performance, tempo de carregamento e tempo até a interatividade em aplicações web?
    \item[Q3:] Quais são os principais desafios e \english{trade-offs} na implementação de \acrshort{csr} e \acrshort{ssr}?
    \item[Q4:] Quais trabalhos relacionados existem na literatura que abordam recomendações sobre quando usar o \acrshort{csr} ou \acrshort{ssr}?

\end{enumerate}

\section{Estratégia de busca}
Esta seção apresenta a estratégia de buscas de artigos científicos e livros relacionados à pesquisa. As ferramentas utilizadas para realizar as buscas são:
\begin{itemize}
    \item \textbf{Periódicos Capes:} É uma ferramenta disponibilizada pelo governo federal para uso de estudantes e pesquisadores. Acessando através da instituição de ensino ou pesquisa, é possível ter acesso completo a uma grande quantidade de artigos científicos publicados em variadas revistas, conferências e universidades. A principal vantagem dessa ferramenta é a possibilidade de ler o conteúdo integral de grande parte das publicações disponíveis. Por outro lado, as expressões de busca atualmente suportadas são bem limitadas.
    \item \textbf{\english{Scopus:}} Trata-se de um ferramenta similar ao Periódicos Capes. No entanto, o \english{Scopus} permite a elaboração de expressões de buscas mais complexas e sofisticadas, servindo para descobrir publicações não detectadas pelas outras plataformas. Além disso, possui um acervo bem mais amplo que o Periódicos Capes. Entretanto, algumas publicações não podem ser vistas na íntegra de forma gratuita.
    \item \textbf{\english{PICOC}}: A técnica \english{PICOC} foi utilizada para estruturar e refinar a estratégia de busca. Essa abordagem consiste em definir cinco elementos principais que auxiliam na formulação da expressão booleana para a pesquisa:
\begin{itemize}
    \item \textbf{P (População/Problema):} Define os estudos ou o grupo de interesse, ou seja, o problema ou a população que se deseja investigar. Por exemplo, “artigos que tratem da integração de tecnologias digitais na educação.”
    \item \textbf{I (Intervenção/Interesse):} Refere-se à intervenção, prática ou fenômeno que está sendo analisado. Neste caso, pode ser a “inserção de tecnologias digitais nos processos de ensino e aprendizagem.”
    \item \textbf{C (Comparação):} Descreve o(s) elemento(s) com os quais a intervenção ou situação é comparada, como “ensino tradicional” ou a comparação entre diferentes estratégias digitais, quando aplicável.
    \item \textbf{O (Outcome/Desfecho):} Indica os resultados ou efeitos esperados da intervenção. Por exemplo, “melhora do desempenho acadêmico” ou “maior engajamento dos alunos.”
    \item \textbf{C (Contexto):} Considera o ambiente ou cenário onde a intervenção ocorre, como “instituições de ensino, universidades” ou “publicações indexadas em bases internacionais.”
\end{itemize}
A partir da definição desses elementos, foi possível construir uma expressão booleana que unisse os principais termos de interesse para a pesquisa. Esse método colaborou para refinar os resultados, tornando a busca mais precisa e abrangente, conforme exemplificado no \autoref{quad:quadro_picoc}.
\end{itemize}


% Como grande parte das publicações na área de computação são em inglês, esta pesquisa utiliza esse idioma para fazer buscas nas ferramentas indicadas. Além disso, \acrfull{csr} e \acrfull{ssr} são relativamente recentes, as buscas se limitaram a publicações feitas nos últimos 20 anos.

% Os termos-chave para realização das buscas são: Microsserviço, \acrshort{csr} e \acrlong{ssr}. Como a busca é feita em inglês, se usará \english{web performance} nas buscas.

\section{Quadro PICOC}
\label{section:quadro_picoc}

\begin{quadro}[H]
\centering

\setlength{\tabcolsep}{0.8em} % espaçamento horizontal
\renewcommand{\arraystretch}{1.5} % espaçamento vertical
\caption{Estrutura PICOC aplicada à pesquisa}
\begin{tabular}{|p{1.5in}|p{4.2in}|}
\hline
\textbf{Elemento} & \textbf{Descrição} \\ \hline

\textbf{P (População/Problema)} & 
Equipes de desenvolvimento web, arquitetos de software e gestores de TI que precisam escolher estratégias de renderização (\acrshort{csr} ou \acrshort{ssr}) para aplicações web modernas, visando otimizar desempenho, \acrshort{seo} e experiência do usuário. 
\\ \hline

\textbf{I (Intervenção)} & 
Adoção de técnicas de \textbf{\acrshort{csr}} (Client-Side Rendering): todo (ou quase todo) o conteúdo gerado no lado do cliente, utilizando frameworks/libraries como React, Vue, Angular etc.
\\ \hline

\textbf{C (Comparação)} & 
Implementação de \textbf{\acrshort{ssr}} (Server-Side Rendering): conteúdo pré-renderizado no servidor antes de ser enviado ao cliente, usando \emph{meta-frameworks} como Next.js, Nuxt.js, SvelteKit, Angular Universal, entre outros.
\\ \hline

\textbf{O (Outcome / Resultado)} & 
\begin{itemize}
  \item Métricas de desempenho (tempo de carregamento, \textit{time-to-first-byte}, \textit{largest contentful paint}, etc.)
  \item Impacto no \textbf{\acrshort{seo}} (indexabilidade, posicionamento em buscadores)
  \item Experiência do usuário e usabilidade
  \item Escalabilidade do sistema (uso de recursos de servidor/cliente)
\end{itemize}
\\ \hline

\textbf{C (Contexto)} & 
Aplicações web modernas que buscam equilibrar interatividade, rapidez de carregamento, otimização para motores de busca e redução de custos operacionais. O estudo pode ser aplicado a sistemas de e-commerce, portais de conteúdo, \emph{landing pages}, etc.
\\ \hline

\end{tabular}
\label{quad:quadro_picoc}
\fonte{os autores}
\end{quadro}


\subsection{Expressão de busca}
\label{section:string_busca}

\begin{quadro}[H]
\centering

\setlength{\tabcolsep}{0.8em} % for the horizontal padding
\renewcommand{\arraystretch}{1.5}% for the vertical padding
\caption{Expressão de busca utilizada}
\begin{tabular}{|p{4.5in}|}

\hline
Expressão de Busca \\ \hline
\english{(TITLE-ABS-KEY("Client-Side Rendering" OR "CSR" OR "Server-Side Rendering" OR "SSR")) AND (TITLE-ABS-KEY("web performance" OR "page speed" OR "web optimization" OR "SEO" OR "search engine optimization" OR "user experience" OR "UX" OR "usability"))} \\ \hline

\end{tabular}
\label{quad:string_busca}
\fonte{os autores}
\end{quadro}

\section{Estratégia de seleção}
\label{section:estrategia_selecao}
A estratégia de seleção dos artigos foi baseada em critérios de inclusão e exclusão, conforme descrito a seguir:
\begin{itemize}
    \item \textbf{Critérios de inclusão:}
    \begin{itemize}
        \item Artigos publicados entre 2020 e 2025.
        \item Artigos que abordem o impacto de \acrshort{csr} e \acrshort{ssr} em aplicações web.
        \item Artigos que apresentem resultados de experimentos ou estudos de caso relacionados a \acrshort{csr} e \acrshort{ssr}.
    \end{itemize}
    \item \textbf{Critérios de exclusão:}
    \begin{itemize}
        \item Artigos que não estejam disponíveis na íntegra.
        \item Artigos que não abordem diretamente o tema da pesquisa.
        \item Artigos que sejam duplicados ou muito semelhantes a outros já selecionados.
    \end{itemize}
\end{itemize}

\section{Caracterização de pesquisa}
\label{section:caracterizacao_pesquisa}
Na \autoref{fig:docs_by_year}, é apresentado o número de publicações identificadas por ano, no intervalo entre 2020 e 2025. Observa-se um crescimento progressivo de 2020 a 2023, culminando em um pico em 2023, com 15 documentos publicados. A partir desse ponto, nota-se uma queda significativa: em 2024, o número de publicações cai para 9, e em 2025 esse número se reduz ainda mais, atingindo apenas 3 documentos. Essa redução pode ser parcialmente atribuída ao fato de que a coleta foi realizada no mês de abril de 2025, o que possivelmente não contempla todas as publicações previstas para o ano. No total, foram encontrados 50 documentos relevantes, distribuídos de forma desigual, evidenciando uma tendência crescente de interesse pelo tema até 2023, seguido de uma possível estabilização ou atraso na indexação dos dados mais recentes.

\begin{figure}[H]
    \centering
    \caption{Número de publicações ao longo dos anos}
    \includegraphics[width=0.9\textwidth]{media/docs_by_year.png}
    \legend{Fonte: Scopus}
    \label{fig:docs_by_year}
\end{figure}

Na \autoref{fig:docs_by_country} são apresentados os países com maior número de publicações relacionadas ao tema desta pesquisa. Observa-se que a China lidera com 10 documentos, seguida pela Alemanha (7) e pelos Estados Unidos (6). Em seguida, aparecem Índia, Reino Unido e Suíça, cada um com 5 documentos publicados. A Turquia também se destaca com 4 publicações, enquanto Croácia, Estônia e França completam a lista com 2 documentos cada. Esses dados evidenciam uma concentração relevante de estudos em países com infraestrutura tecnológica consolidada, especialmente na Ásia, Europa Ocidental e América do Norte, o que reforça o caráter global do interesse em torno da comparação entre abordagens de renderização no desenvolvimento de aplicações web.

\begin{figure}[H]
    \centering
    \caption{Distribuição das publicações por país}
    \includegraphics[width=0.9\textwidth]{media/docs_by_country.png}
    \legend{Fonte: Scopus}
    \label{fig:docs_by_country}
\end{figure}


\section{Artigos selecionados}
\label{section:artigos_selecionados}

Após realizadas as leituras preliminares, apenas 13 publicações mostraram-se relevantes para responder às questões de pesquisa e/ou apoiar na elaboração do estudo de caso. Esta seleção considerou critérios de alinhamento temático, profundidade técnica e aplicabilidade ao escopo da pesquisa. A listagem completa dos artigos selecionados pode ser consultada no Quadro~\ref{quad:publicacoes_desenvolvimento_web}.

\begin{quadro}[H]
    \centering
    
    \setlength{\tabcolsep}{0.8em} % espaçamento horizontal
    \renewcommand{\arraystretch}{1.5}% espaçamento vertical
    \begin{tabular}{p{4in}|p{0.5in}}
    \hline
    
    \multicolumn{1}{|p{4in}}{\textbf{Título}} & 
    \multicolumn{1}{|p{0.5in}|}{\textbf{Ano}} \\
    \hhline{--}
    
    \multicolumn{1}{|p{4in}}{\english{Progressive Server-Side Rendering with Suspendable Web Templates}} & 
    \multicolumn{1}{|p{0.5in}|}{\citeyear{Carvalho2025458}} \\
    \hhline{--}
    
    \multicolumn{1}{|p{4in}}{\english{Requirements for the Development of a Website Builder with Adaptive Design}} & 
    \multicolumn{1}{|p{0.5in}|}{\citeyear{Bekmanova2024265}} \\
    \hhline{--}
    
    \multicolumn{1}{|p{4in}}{\english{Enhancing SEO in Single-Page Web Applications in Contrast With Multi-Page Applications}} & 
    \multicolumn{1}{|p{0.5in}|}{\citeyear{Kowalczyk202411597}} \\
    \hhline{--}
    
    \multicolumn{1}{|p{4in}}{\english{Web Development Using ReactJS}} & 
    \multicolumn{1}{|p{0.5in}|}{\citeyear{Keshari20231571}} \\
    \hhline{--}
    
    \multicolumn{1}{|p{4in}}{\english{Improving Universal Rendering Performance on NuxtJS-based Web Application}} & 
    \multicolumn{1}{|p{0.5in}|}{\citeyear{Angkasa2023}} \\
    \hhline{--}
    
    \multicolumn{1}{|p{4in}}{\english{Comparison between client-side and server-side rendering in the web development}} & 
    \multicolumn{1}{|p{0.5in}|}{\citeyear{FadhilahIskandar2020}} \\
    \hhline{--}
    
    \multicolumn{1}{|p{4in}}{\english{Methods of Improving and Optimizing React Web-applications}} & 
    \multicolumn{1}{|p{0.5in}|}{\citeyear{Pavic20211753}} \\
    \hhline{--}
    
    \multicolumn{1}{|p{4in}}{\english{Improving ruby on rails-based web application performance}} & 
    \multicolumn{1}{|p{0.5in}|}{\citeyear{Klochkov2021}} \\
    \hhline{--}
    
    \multicolumn{1}{|p{4in}}{\english{An integrated framework of user experience-oriented smart service requirement analysis for smart product service system development}} & 
    \multicolumn{1}{|p{0.5in}|}{\citeyear{Zhou2022}} \\
    \hhline{--}
    
    \multicolumn{1}{|p{4in}}{\english{A Research Framework for B2B Green Marketing Innovation: the Design of Sustainable Websites}} & 
    \multicolumn{1}{|p{0.5in}|}{\citeyear{Lacom2022}} \\
    \hhline{--}
    
    \multicolumn{1}{|p{4in}}{\english{Corporate Social Responsibility: Hiring Requisition in Media Companies?}} & 
    \multicolumn{1}{|p{0.5in}|}{\citeyear{Boehncke202375}} \\
    \hhline{--}
    
    \multicolumn{1}{|p{4in}}{\english{Single page optimization techniques using react}} & 
    \multicolumn{1}{|p{0.5in}|}{\citeyear{Pokhriyal2024338}} \\
    \hhline{--}

    \multicolumn{1}{|p{4in}}{\english{Proceedings of the 19th International Conference on Web Information Systems and Technologies, WEBIST 2023}} & 
    \multicolumn{1}{|p{0.5in}|}{\citeyear{2023}} \\
    \hhline{--}
    
    \end{tabular}
    \caption{Artigos selecionados sobre estratégias de renderização e desempenho em aplicações web}
    \label{quad:publicacoes_desenvolvimento_web}
\end{quadro}
    

\section{Resultados e Discussão}
\label{section:resultados_discussao}

A análise dos 13 artigos selecionados permitiu identificar padrões, lacunas e contribuições relevantes no debate sobre as abordagens de renderização \acrshort{csr} e \acrshort{ssr} no desenvolvimento de aplicações web. Os resultados foram organizados de acordo com as questões de pesquisa previamente definidas (\autoref{section:questoes_pesquisa}).

\subsection{Q1: De que maneira a escolha entre \acrshort{csr} e \acrshort{ssr} influencia a experiência do usuário?}

Diversos estudos destacaram o impacto direto da estratégia de renderização na experiência percebida pelo usuário. Por exemplo, o trabalho de \cite{Zhou2022} propõe um framework voltado à experiência do usuário em sistemas inteligentes, indicando que tempos de resposta mais rápidos — geralmente proporcionados por renderização no servidor — influenciam positivamente a satisfação dos usuários finais. Embora o estudo não se concentre diretamente em aplicações web tradicionais, os princípios aplicados ao tempo de resposta e fluidez de interação podem ser extrapolados para cenários SSR, destacando o papel do tempo até a interatividade como um fator crítico. Essa abordagem abre espaço para futuras investigações que apliquem o framework em contextos específicos de CSR e SSR.

De forma complementar, \cite{Lacom2022} reforça a importância do carregamento eficiente para sites sustentáveis e voltados ao marketing digital, mencionando que usuários tendem a abandonar páginas lentas, o que afeta indicadores de engajamento e conversão.

Além disso, \cite{Pokhriyal2024338} e \cite{Keshari20231571} apontam que aplicações construídas com \acrshort{csr}, embora mais interativas após o carregamento inicial, tendem a apresentar maior latência no primeiro carregamento, o que pode prejudicar a primeira impressão do usuário e sua propensão a permanecer na página.

\subsection{Q2: Como as abordagens \acrshort{csr} e \acrshort{ssr} afetam métricas de performance, tempo de carregamento e tempo até a interatividade em aplicações web?}

Os estudos de \cite{FadhilahIskandar2020} e \cite{Angkasa2023} apresentam comparações diretas entre \acrshort{csr} e \acrshort{ssr} em relação a métricas como \textit{time-to-first-byte (TTFB)}, \textit{first contentful paint (FCP)} e \textit{largest contentful paint (LCP)}. Os resultados mostram que a abordagem \acrshort{ssr}, especialmente quando combinada com estratégias híbridas (como \textit{static site generation} ou \textit{incremental rendering}), apresenta desempenho superior nos momentos iniciais de carregamento da página.

No entanto, o trabalho de \cite{Pavic20211753} chama atenção para a possibilidade de otimização no lado do cliente, sugerindo que ajustes finos em bibliotecas como React podem mitigar parte das desvantagens de \acrshort{csr} em termos de tempo de carregamento.

Já \cite{Bekmanova2024265} e \cite{Klochkov2021} demonstram que a performance final da aplicação pode depender mais da arquitetura geral e das boas práticas de desenvolvimento do que apenas da escolha entre SSR ou CSR. Isso evidencia que decisões arquiteturais e técnicas de otimização, como pré-carregamento de recursos e minimização de dependências, exercem papel crucial na performance percebida pelo usuário.

\subsection{Q3: Quais são os principais desafios e \textit{trade-offs} na implementação de \acrshort{csr} e \acrshort{ssr}?}

A literatura aponta diversos desafios na escolha entre as duas abordagens, incluindo complexidade de implementação, compatibilidade com \acrshort{seo}, custo computacional e escalabilidade.

O artigo de \cite{Carvalho2025458} destaca que o uso de SSR com \textit{templates} suspensíveis pode melhorar a performance, mas introduz complexidade no código, exigindo mais esforços de manutenção e testes. Por outro lado, \cite{Kowalczyk202411597} aponta que aplicações baseadas em CSR, especialmente SPAs, enfrentam desafios consideráveis em relação à indexabilidade por mecanismos de busca, o que limita seu uso em sites orientados a conteúdo.

Além disso, a adoção de SSR pode impactar negativamente a escalabilidade do sistema, como discutido por \cite{Angkasa2023}, uma vez que o servidor passa a ter maior carga de processamento por requisição. Em contextos de alto tráfego, isso pode implicar em aumento de custos com infraestrutura e complexidade na distribuição de carga.

\subsection{Q4: Quais trabalhos relacionados existem na literatura que abordam recomendações sobre quando usar o \acrshort{csr} ou o \acrshort{ssr}?}

Embora a maioria dos artigos foque em aspectos técnicos, alguns oferecem orientações práticas sobre o uso apropriado de cada abordagem. Por exemplo, \cite{FadhilahIskandar2020} sugere que aplicações voltadas a conteúdo estático ou com forte dependência de \acrshort{seo} devem optar por SSR ou SSG. Já aplicações altamente interativas, como dashboards, sistemas internos ou PWAs, se beneficiam mais da flexibilidade e dinamismo do CSR.

O estudo publicado da WEBIST 2023 \cite{2023} também apresenta uma síntese interessante sobre abordagens mistas, propondo que a adoção de técnicas como \textit{server-side hydration} e \textit{isomorphic rendering} pode combinar o melhor dos dois mundos, oferecendo equilíbrio entre tempo de carregamento e interatividade.

Apesar das recomendações, nota-se ainda uma carência de diretrizes sistematizadas que ajudem desenvolvedores a escolherem a abordagem ideal com base em variáveis como tipo de aplicação, volume de tráfego, perfil dos usuários e metas de negócio. Essa lacuna representa uma oportunidade relevante para pesquisas futuras que proponham modelos de decisão mais robustos para contextos reais de desenvolvimento web.


% https://www-scopus-com.ez135.periodicos.capes.gov.br/results/results.uri?sort=plf-f&src=s&sid=c5e620a132b74a852dccc08c5c330e9d&sot=a&sdt=cl&sl=247&s=%28TITLE-ABS-KEY%28%22Client-Side+Rendering%22+OR+%22CSR%22+OR+%22Server-Side+Rendering%22+OR+%22SSR%22%29%29+AND+%28TITLE-ABS-KEY%28%22web+performance%22+OR+%22page+speed%22+OR+%22web+optimization%22+OR+%22SEO%22+OR+%22search+engine+optimization%22+OR+%22user+experience%22+OR+%22UX%22+OR+%22usability%22%29%29&origin=resultslist&editSaveSearch=&txGid=584773b92f904689225d6d8bd0143429&sessionSearchId=c5e620a132b74a852dccc08c5c330e9d&limit=10&yearFrom=2020&yearTo=2025


\input{textual/t5-cronograma}

\chapter{Estudo de Caso }
% : Requisitos, Organização, Métodos, Design, Implementação e Testes
\label{cap:estudo_caso1}
Este capítulo apresenta o contexto, os requisitos, a organização e os métodos utilizados no desenvolvimento do estudo de caso.

\section{Contexto}
\label{section:contexto}
O estudo de caso é realizado em uma empresa fictícia.

\section{Ciclo de Vida do Desenvolvimento de Software}



\section{Requisitos}
\label{section:requisitos}

\section{Design do Sistema}
\label{cap:design}

O design do sistema foi orientado para refletir as diferenças estruturais entre as abordagens \acrshort{spa} e \acrshort{mpa}, levando em consideração os requisitos funcionais da plataforma \textit{WallTech}. Esta vitrine digital exibe notícias de tecnologia obtidas por meio da \textit{NewsAPI}, com recursos de busca, filtragem e destaque de conteúdo segmentado por país. Não há backend próprio, sendo todas as chamadas feitas diretamente para a API externa, o que simplifica a arquitetura e acentua o papel do frontend na renderização de conteúdo.

Para modelagem arquitetural e comportamental, foram utilizados diagramas da UML, incluindo:
\begin{itemize}
  \item \textbf{Diagrama de Caso de Uso}, para representar as principais funcionalidades acessadas pelos usuários visitantes.
  \item \textbf{Diagrama de Sequência}, a fim de ilustrar o fluxo de interação entre navegador e a \textit{NewsAPI} durante operações como busca e carregamento de notícias.
  \item \textbf{Diagrama de Componentes}, para representar os módulos da aplicação, como a interface, o serviço de requisição à API, e os componentes de renderização.
\end{itemize}

As decisões de design foram fundamentadas em boas práticas para renderização web discutidas por \cite{osmani2025}, bem como nas diretrizes da literatura especializada em arquitetura de frontend, como apresentado pela \cite{atori2024}.

Segundo \cite{osmani2025}, a escolha entre renderização no cliente ou no servidor deve considerar o contexto da aplicação, os requisitos de desempenho e os objetivos de SEO. Já o artigo da \cite{atori2024} destaca que SPAs tendem a oferecer maior fluidez e interatividade, enquanto MPAs são mais eficazes em aplicações que dependem de indexação e acessibilidade.

\subsection{Arquitetura SPA}

A versão \acrshort{spa} da aplicação foi projetada com foco em responsividade e fluidez de interação. Utilizando bibliotecas como React, a renderização do conteúdo é feita inteiramente no navegador, após o carregamento inicial de um documento \texttt{HTML} mínimo. Todas as requisições são feitas dinamicamente via \texttt{fetch} para a \textit{NewsAPI}, permitindo uma experiência de uso imersiva sem recarregamentos completos de página.

Essa abordagem reduz a carga no servidor, melhora a experiência de usuário em conexões rápidas e facilita a manutenção do código por meio de componentes reutilizáveis. Por outro lado, apresenta desafios quanto à indexação por mecanismos de busca, dado que o conteúdo depende da execução de JavaScript, conforme destacado por \cite{osmani2025}. Essa limitação pode ser parcialmente mitigada com técnicas como \textit{pre-rendering} e \textit{hydration} progressiva.

O diagrama de componentes da SPA evidencia a centralização da lógica no navegador, enquanto o diagrama de sequência demonstra o ciclo de vida de requisições e renderização client-side.

\subsection{Arquitetura MPA}

A versão \acrshort{mpa}, por sua vez, simula a renderização \acrshort{ssr}, em que cada rota representa uma nova página entregue já montada. Neste modelo, cada requisição à aplicação resulta na obtenção de uma nova instância de HTML contendo os dados pré-processados. Frameworks como Next.js permitem implementar essa estratégia, mesmo quando os dados provêm de APIs externas como a \textit{NewsAPI}.

Esse tipo de renderização favorece o desempenho inicial (menor \acrshort{fcp}) e melhora a indexação do conteúdo pelos mecanismos de busca, como discutido pela \cite{atori2024}. Na aplicação, isso se traduz em páginas com conteúdo estático inicial pronto para rastreamento, mesmo que a interatividade seja ativada posteriormente via JavaScript (\textit{hydration}).

O diagrama de componentes da MPA demonstra a responsabilidade do servidor em montar o HTML completo, enquanto o diagrama de sequência mostra o papel da API externa na geração do conteúdo antes da entrega ao navegador.



\section{Implementação}
\label{sec:implementacao}

Esta seção descreve o conjunto de tecnologias, bibliotecas e ferramentas que são empregadas para a construção das duas versões do sistema de prova de conceito, detalhando a fundamentação para a escolha de cada componente do ecossistema de desenvolvimento. O gerenciamento do código-fonte e do ciclo de vida do projeto é realizado com o sistema de controle de versão \textbf{Git} e a plataforma de hospedagem \textbf{GitHub}, conforme as práticas descritas na Seção~\ref{sec:git-github}.

Ambas as implementações consomem dados da mesma fonte externa, a \textbf{News API}, uma \acrshort{api} RESTful que fornece o conteúdo jornalístico para a aplicação, como detalhado na Seção~\ref{sec:news-api}. Para garantir a consistência visual e a qualidade da interface entre as duas arquiteturas, utiliza-se a biblioteca de componentes \textbf{shadcn/ui}, que oferece um conjunto de componentes acessíveis e personalizáveis, conforme apresentado na Seção~\ref{sec:ferramentas-modernas}.

\subsection{Implementação da Aplicação SPA}
\label{ssec:implementacao_spa}

A implementação da \acrfull{spa} é desenvolvida utilizando a biblioteca \textbf{React} na sua versão 18. O React é uma biblioteca JavaScript declarativa, mantida pela Meta, focada na construção de interfaces de usuário a partir de componentes reutilizáveis. Sua adoção neste projeto se dá por sua vasta popularidade no mercado e ao seu paradigma de componentização, que facilita a criação de UIs modulares e de fácil manutenção \cite{react2025}. A eficiência da renderização é otimizada pelo uso de um DOM Virtual, um conceito central da biblioteca que minimiza as manipulações diretas no navegador.

Para a estruturação inicial do projeto e o gerenciamento do ambiente de desenvolvimento, utiliza-se a ferramenta de \textit{build} \textbf{Vite}. O Vite é um ecossistema de desenvolvimento frontend moderno que oferece um servidor de desenvolvimento com recarregamento rápido (\textit{Hot Module Replacement}) e um processo de compilação (\textit{build}) otimizado, que resulta em pacotes de produção menores e mais eficientes \cite{vite_docs}.

O roteamento no lado do cliente, uma característica fundamental da arquitetura SPA, implementa-se com a biblioteca \textbf{React Router}. Trata-se da solução padrão para navegação em aplicações React, que possibilita a criação de uma experiência de usuário fluida e sem recarregamentos de página ao manipular a \acrshort{api} de Histórico do navegador \cite{react_router_docs}. A comunicação com a News API é realizada por meio da \acrshort{api} \texttt{fetch}, nativa dos navegadores modernos.

\subsection{Implementação da Aplicação MPA}
\label{ssec:implementacao_mpa}

A implementação da \acrfull{mpa}, com foco em \acrfull{ssr}, é desenvolvida com o \emph{framework} \textbf{Next.js} na versão 14. O Next.js é um meta-framework baseado em React, mantido pela Vercel, que se posiciona como uma solução completa para a construção de aplicações web de produção. Sua escolha para este estudo de caso justifica-se por ser a principal referência de mercado para a implementação de \acrshort{ssr} no ecossistema React, oferecendo uma estrutura robusta e opinativa \cite{nextjs2024}.

O \emph{framework} opera sobre um ambiente \textbf{Node.js}, o que permite a execução de código JavaScript no lado do servidor \cite{nodejs2025}. Essa capacidade é a base da renderização no servidor, onde o Next.js utiliza funções específicas, como a \texttt{getServerSideProps}, para buscar dados de fontes externas e pré-renderizar o HTML completo de uma página antes de enviá-la ao navegador. Além disso, o Next.js implementa um sistema de roteamento baseado no sistema de arquivos, onde a estrutura de diretórios da pasta \texttt{pages} define automaticamente as rotas da aplicação, simplificando a configuração e a manutenção do projeto.

\chapter{Resultados e Discussões}
\label{cap:resultados}

Este capítulo apresenta os achados empíricos do estudo, com foco na experiência do usuário medida por \textit{Core Web Vitals} (coleta em campo, no cliente) e, de forma complementar, em diagnósticos laboratoriais do Lighthouse. Como contexto operacional, reporta-se também o consumo de CPU, memória e PIDs dos contêineres durante as execuções. A metodologia, o ambiente controlado e os procedimentos de coleta já foram descritos no Cap.~\ref{cap:estudo_caso_dev}; aqui concentramos a atenção em resultados e implicações.

\section{Resultados}
Os testes cobriram as duas variantes da WallTech (\acrshort{ssr}/\acrshort{mpa} em Next.js e \acrshort{csr}/\acrshort{spa} em React) nas páginas: inicial, resultados de busca e detalhes da notícia. Para cada cenário, foram realizados 10--15 carregamentos após \textit{warm-up}, e as leituras são analisadas por mediana (p50) e, quando pertinente, p95.

\subsection{Fontes de dados}
\begin{itemize}
    \item \textbf{Web Vitals (núcleo)}: \acrshort{ttfb}, \acrshort{fcp}, \acrshort{lcp}, \acrshort{cls}, \acrshort{inp} registrados no cliente e persistidos em NDJSON.
    \item \textbf{Lighthouse (apoio)}: TBT e principais oportunidades/auditorias de desempenho, além de verificações de Acessibilidade e SEO.
    \item \textbf{Recursos do servidor}: CPU/memória/PIDs via \texttt{docker stats}, como contexto de custo de execução.
\end{itemize}

\subsection{Análise e Visualização dos Dados}
Os dados coletados, incluindo as métricas de \textit{Web Vitals} e as medições de recursos do servidor, foram exportados e analisados com o auxílio do Power BI. A ferramenta foi utilizada para gerar gráficos e visualizações interativas que facilitam a interpretação dos resultados e a comparação entre os diferentes cenários (\acrshort{ssr} e \acrshort{csr}). O uso do Power BI permitiu a criação de dashboards dinâmicos que ilustram claramente as tendências das métricas ao longo das execuções e facilitam a análise de variações nas diferentes páginas testadas.

\subsection{Tratamento dos Dados}
Os dados do Web Vitals foram coletados no formato NDJSON (Newline Delimited JSON) e importados como arquivos de texto no Power BI. Cada linha do arquivo correspondia a um objeto JSON individual representando uma métrica registrada no navegador.

Para estruturar essas informações, foi utilizada a funcionalidade de transformação de dados do Power BI. Cada linha foi analisada como um objeto JSON, e, em seguida, esses objetos foram expandidos em colunas, convertendo os campos internos de cada métrica em atributos tabulares.

Essa expansão resultou nas seguintes colunas principais:

\begin{itemize}
    \item \textbf{id}: identificador da interação registrada.
    \item \textbf{name}: nome da métrica (por exemplo, \texttt{INP}, \texttt{CLS}, \texttt{LCP}).
    \item \textbf{value}: valor numérico observado para a métrica.
    \item \textbf{rating}: classificação qualitativa (\textit{good}, \textit{needs improvement}, \textit{poor}).
    \item \textbf{navigationType}: tipo de navegação associada à métrica (ex: reload, back/forward).
    \item \textbf{attribution}: detalhes adicionais sobre o elemento ou evento associado à métrica.
    \item \textbf{entries}: informações brutas sobre a entrada original da medição.
    \item \textbf{ts}: timestamp da coleta.
\end{itemize}

Com os dados estruturados dessa forma, foi possível aplicar filtros, agregações e análises comparativas entre SSR e CSR, utilizando os recursos visuais e analíticos do Power BI de maneira eficiente.

\subsection{Metas de referência}
Adotaram-se os seguintes alvos para interpretação:
\begin{itemize}
    \item \textbf{LCP} $\leq 2{,}5$\,s;\quad \textbf{CLS} $\leq 0{,}10$;\quad \textbf{INP} $\leq 200$\,ms;\quad \textbf{TTFB} $\leq 800$\,ms.
    \item \textbf{TBT (Lighthouse)} $<200$\,ms;\quad \textbf{Acessibilidade (LH)} $\geq 90$;\quad \textbf{SEO (LH)} $\geq 90$.
\end{itemize}

\subsection{Resultados para SSR}
\label{subsec:resultados-ssr}

\subsubsection{Recursos do servidor}
Ao longo dos cenários, o contêiner \textit{Next.js} manteve \textbf{CPU} baixa e estável (aprox.\ 8--13\%), sem picos relevantes; a queda ao final da série coincide com o término do experimento (\autoref{fig:ssr-cpu}). A \textbf{memória} permaneceu em patamar quase constante, com variação estreita e redução apenas no encerramento das execuções (\autoref{fig:ssr-mem}). Esse comportamento é compatível com o processamento por requisição do SSR e não sinaliza saturação.

\begin{figure}[H]
    \centering
    \caption{Média de uso de CPU por tempo (\acrshort{ssr})}
    \includegraphics[width=0.9\textwidth]{media/uso_cpu_ssr.jpeg}
    \legend{Fonte: os autores.}
    \label{fig:ssr-cpu}
\end{figure}

\begin{figure}[H]
    \centering
    \caption{Média de uso de memória por tempo (\acrshort{ssr})}
    \includegraphics[width=0.9\textwidth]{media/uso_memoria_ssr.jpeg}
    \legend{Fonte: os autores.}
    \label{fig:ssr-mem}
\end{figure}

\subsubsection{Web Vitals (cliente)}
Conforme a \autoref{fig:ssr-webvitals}, as leituras de campo foram, em sua maioria, classificadas como \textit{good}. Os poucos casos de \textit{needs-improvement} concentraram-se em \acrshort{lcp} e \acrshort{inp}, e houve uma fração reduzida de \acrshort{fcp} em \textit{poor} efeito compatível com custos de \textit{hydration} e com elementos visuais de grande porte na dobra (imagens \emph{hero}). Em relação às metas adotadas, os resultados situam-se, em geral, próximos ou dentro dos limiares de referência: \acrshort{lcp} $\leq 2{,}5$\,s; \acrshort{cls} $\leq 0{,}10$; \acrshort{inp} $\leq 200$\,ms; \acrshort{ttfb} $\leq 800$\,ms.

\begin{figure}[H]
    \centering
    \caption{Classificação das métricas de desempenho (\acrshort{ssr}) com Web Vitals}
    \includegraphics[width=0.9\textwidth]{media/metricas_ssr_web_vitals.jpeg}
    \legend{Fonte: os autores.}
    \label{fig:ssr-webvitals}
\end{figure}

\subsubsection{Lighthouse (apoio diagnóstico)}
Para explicar variações residuais e priorizar melhorias, auditamos as rotas \emph{home}, \emph{busca} e \emph{detalhe} no modo \emph{Navigation} do Chrome DevTools, considerando três indicadores: TBT, Acessibilidade e SEO.
A Tabela~\ref{tab:lh-ssr} consolida as medianas observadas nos relatórios HTML anexados (\texttt{ssrMobile.html}, \texttt{ssrMobileBusca.html}, \texttt{ssrtestePage.html}).

\begin{table}[H]
\centering
\caption{Lighthouse (SSR) — mediana por rota}
\label{tab:lh-ssr}
\begin{tabular}{|l|c|c|c|}
\hline
\textbf{Rota} & \textbf{TBT (ms)} & \textbf{Acessibilidade (\%)} & \textbf{SEO (\%)} \\
\hline
Home    & 12 & 95  & 100 \\
Busca   & 10 & 99  & 100 \\
Detalhe & 0--7\footnotemark[1] & 100 & 100 \\
\hline
\end{tabular}
\end{table}
\footnotetext[1]{Variações muito baixas entre execuções; mediana $\approx$ 5\,ms.}

\noindent \textit{Leitura.}
(i) \textbf{TBT} muito abaixo da meta interna ($<200$\,ms) em todas as rotas, corroborando \acrshort{inp} em \textit{good};
(ii) \textbf{Acessibilidade} alta (95--100\%), com ajustes residuais (hierarquia de headings, foco visível, rótulos) de fácil endereçamento;
(iii) \textbf{SEO} consistente (100\%), indicando sinalização adequada (título, \emph{meta description}, \texttt{canonical}, \texttt{robots}, \texttt{viewport}) aspecto particularmente relevante no fluxo SSR.

\subsection{Síntese (SSR)}
Servidor com CPU/RAM contidos; \emph{Web Vitals} majoritariamente \textit{good}, com atenção pontual a \acrshort{lcp}/\acrshort{fcp} em páginas com imagens destacadas e pós-hidratação; \emph{Lighthouse} confirma TBT reduzido e altos escores de Acessibilidade e SEO, sustentando o SSR para páginas públicas sensíveis à descoberta.

\subsection{Resultados para CSR}
\label{subsec:resultados-csr}

\subsubsection{Recursos do servidor}
O contêiner da \emph{SPA} apresentou \textbf{CPU} muito baixa e estável, com oscilações discretas e picos inferiores a 5\% (\autoref{fig:csr-cpu}). A \textbf{memória} permaneceu praticamente constante e em patamar reduzido durante todo o ensaio, coerente com a entrega estática e a execução da lógica no cliente (\autoref{fig:csr-mem}). Não se observaram sinais de saturação.

\begin{figure}[H]
\centering
\caption{Média de uso de CPU por tempo (\acrshort{csr})}
\includegraphics[width=0.9\textwidth]{media/uso_cpu_csr.jpeg}
\legend{Fonte: os autores.}
\label{fig:csr-cpu}
\end{figure}

\begin{figure}[H]
\centering
\caption{Média de uso de memória por tempo (\acrshort{csr})}
\includegraphics[width=0.9\textwidth]{media/uso_memoria_csr.jpeg}
\legend{Fonte: os autores.}
\label{fig:csr-mem}
\end{figure}

\subsubsection{Web Vitals (cliente)}
As leituras de campo indicam predomínio de \textit{good} em \acrshort{ttfb}, \acrshort{fcp}, \acrshort{lcp} e \acrshort{inp} (\autoref{fig:csr-webvitals}). O ponto fora da curva é a \textbf{\acrshort{cls}}, com concentrações em \textit{poor}. O padrão é típico de SPAs quando ocorrem \emph{layout shifts} durante a hidratação ou quando imagens/slots não reservam espaço antes do carregamento (dimensões ausentes, fontes sem \emph{preload}, inserções acima da dobra). Em relação às metas, as leituras ficaram, em geral, dentro ou próximas dos limiares de referência (\acrshort{lcp} $\leq 2{,}5$\,s; \acrshort{cls} $\leq 0{,}10$; \acrshort{inp} $\leq 200$\,ms; \acrshort{ttfb} $\leq 800$\,ms), com a ressalva da \acrshort{cls}.

\begin{figure}[H]
\centering
\caption{Classificação das métricas de desempenho (\acrshort{csr}) com Web Vitals}
\includegraphics[width=0.9\textwidth]{media/metricas_csr_web_vitals.jpeg}
\legend{Fonte: os autores.}
\label{fig:csr-webvitals}
\end{figure}

\subsubsection{Lighthouse (apoio diagnóstico)}
Para explicar oscilações residuais e orientar priorizações, auditamos as rotas críticas no \emph{Chrome DevTools} (modo \emph{Navigation}), com três indicadores: TBT, Acessibilidade e SEO.

\noindent \textit{Achados.}
\begin{itemize}
    \item \textbf{TBT}: manteve-se em patamar \textbf{muito baixo} (dezenas de milissegundos) nas páginas inspecionadas, consistente com \acrshort{inp} \textit{good}.
    \item \textbf{Acessibilidade}: escores altos na \emph{home} (93--94\,\%), com ajustes residuais (hierarquia de \emph{headings}, foco visível, rótulos/\textit{alt}) valores observados nos relatórios “Home (Web)” e “Home (Mobile)”.
    \item \textbf{SEO}: a \emph{home} registrou \textasciitilde{}83\,\% (Web e Mobile), sugerindo oportunidades pontuais (por exemplo, \emph{meta description}/\texttt{canonical}/\texttt{robots}/\texttt{hreflang}, quando aplicável).
\end{itemize}

\noindent \textit{Leitura.} Em conjunto, os relatórios apontam que o gargalo do CSR não está em bloqueio de \emph{main thread} (TBT baixo), mas em estabilidade visual e em alguns sinais de SEO. Na prática, os seguintes ajustes tendem a capturar os ganhos: (i) reservar dimensões de imagens e \emph{cards} acima da dobra (\texttt{width}/\texttt{height} ou \texttt{aspect-ratio}); (ii) \emph{preload} de fontes e imagens \emph{hero} (com \texttt{font-display: swap}); (iii) evitar inserções DOM acima da dobra durante a hidratação; (iv) completar metadados de SEO (título/descrição/\texttt{canonical}/\texttt{robots}) e revisar indexabilidade.

\subsubsection{Síntese (CSR)}
Servidor praticamente ocioso (CPU e RAM baixos); \emph{Web Vitals} em boa forma para \acrshort{ttfb}/\acrshort{fcp}/\acrshort{lcp}/\acrshort{inp}, com atenção especial à \acrshort{cls}. O \emph{Lighthouse} confirmou TBT reduzido, Acessibilidade alta e SEO com melhorias de rápida implementação, coerentes com o perfil de SPA e suas responsabilidades no cliente.

\subsection{Comparação entre SSR e CSR}
\label{subsec:comparacao-ssr-csr}

Esta seção compara, de forma integrada, experiência do usuário (\emph{Web Vitals} em campo), diagnóstico laboratorial (Lighthouse: TBT, Acessibilidade e SEO) e custo operacional (CPU/Memória nos contêineres). A análise revela que a escolha arquitetônica representa um trade-off fundamental sobre onde a carga computacional deve residir: no servidor ou no cliente. A leitura está ancorada nas medianas (p50) e, quando pertinente, no p95 dos cenários home, busca e detalhe.

\begin{table}[H]
\centering
\caption{Síntese comparativa dos resultados (SSR $\times$ CSR) neste estudo}
\label{tab:comparativo-ssr-csr}
\begin{tabular}{|p{4.2cm}|p{5.2cm}|p{5.2cm}|}
\hline
\textbf{Aspecto} & \textbf{SSR (MPA/Next.js)} & \textbf{CSR (SPA/React)} \\
\hline
\textbf{CPU no servidor} & Baixa e estável (\textit{$\sim$8--13\%}), sem picos relevantes. & Muito baixa; oscilações discretas com picos $<5\%$. \\
\hline
\textbf{Memória no servidor} & Estável, variação estreita; queda apenas ao término dos testes. & Muito baixa e praticamente constante. \\
\hline
\textbf{Web Vitals (campo)} & Predominância de \textit{good}; pequenos trechos \textit{needs-improvement} em \textbf{LCP}/\textbf{INP} e fração \textit{poor} em \textbf{FCP} (páginas com imagens \emph{hero} e pós-hidratação). & \textbf{TTFB}/\textbf{FCP}/\textbf{LCP}/\textbf{INP} majoritariamente \textit{good}; \textbf{CLS} com ocorrências \textit{poor} (shifts na montagem/hidratação e elementos sem reserva de espaço). \\
\hline
\textbf{Lighthouse: TBT} & \textbf{Muito baixo} (home $\approx$12\,ms; busca $\approx$10\,ms; detalhe $\approx$5\,ms). & \textbf{Muito baixo} (ordem de dezenas de ms na home). \\
\hline
\textbf{Lighthouse: Acessibilidade} & \textbf{Alta} (95--100\%). & \textbf{Alta} (93--94\% na home). \\
\hline
\textbf{Lighthouse: SEO} & \textbf{100\%} nas rotas auditadas. & \textbf{$\sim$83\%} na home; indica metadados/sinalizações a completar. \\
\hline
\textbf{Leitura operacional} & Parte da renderização no servidor melhora \textbf{TTFB}/\emph{first paint} e favorece \textbf{SEO}; cuidado com \emph{hydration} e imagens de grande porte. & Renderização no cliente alivia o servidor e mantém \textbf{TBT/INP} baixos; requer disciplina de layout para evitar \textbf{CLS} e ajustes de \textbf{SEO}. \\
\hline
\end{tabular}
\end{table}

\subsubsection{Pontos fortes do SSR.}
(i) Tempo até primeira resposta/primeiro conteúdo. O HTML pré-renderizado acelera a exibição inicial, refletindo-se em \textit{TTFB}/\textit{FCP} consistentes e TBT residual. Essa agilidade é um fator psicológico poderoso, transmitindo eficiência e sendo crucial para reter a atenção do usuário nos primeiros segundos.
(ii) Descoberta e rastreabilidade. SEO com 100\% nas rotas testadas, reforçando a adequação do SSR para páginas públicas e conteúdo editorial.
(iii)Estabilidade Visual. O índice de \acrshort{cls} praticamente nulo garantiu que a experiência não fosse interrompida por mudanças de layout abruptas.
(iv) Previsibilidade de performance. Em cenários com rede/CPU do cliente mais limitados, a renderização no servidor reduz o risco de longas tarefas no \emph{main thread} do navegador.

\subsubsection{Pontos fracos do SSR.}
(i) \textbf{Custo de hidratação e navegação.} Após o HTML inicial, a ativação dos componentes pode degradar \textbf{LCP}/\textbf{FCP} em páginas com muito JS. A navegação subsequente, ao exigir uma nova requisição completa ao servidor, pode quebrar a sensação de fluidez contínua.
(ii) \textbf{Custo no servidor (ainda que baixo).} O processamento por requisição eleva discretamente \textbf{CPU}/\textbf{memória} no host, implicando maiores custos de infraestrutura e complexidade de escalonamento.
(iii) \textbf{Complexidade de build/deploy.} Estratégias como \emph{streaming} e \emph{Server Components} elevam a complexidade de arquitetura.

\subsubsection{Pontos fortes do CSR.}
(i) \textbf{Interatividade contínua.} Após o carregamento inicial, a navegação entre seções ocorre de forma quase instantânea, com \textbf{TBT/INP} muito baixos, emulando a agilidade de um aplicativo nativo.
(ii) \textbf{Custo operacional.} CPU/memória do servidor permanecem muito baixos, já que o servidor atua majoritariamente como um provedor de arquivos estáticos, resultando em custos operacionais menores.
(iii) \textbf{Escalabilidade horizontal simples.} Conteúdo estático facilita o uso de cache e CDNs, reduzindo a pressão sobre o backend.

\subsubsection{Pontos fracos do CSR.} 
(i) \textbf{Estabilidade visual.} O CLS apresentou ocorrências \textit{poor}. O processo de hidratação e a renderização dinâmica podem causar mudanças de layout que frustram o usuário e comprometem a usabilidade.
(ii) \textbf{Sinais de SEO.} Escores de \textasciitilde83\% no Lighthouse indicam a necessidade de completar metadados e sinalizações essenciais para a indexabilidade.
(iii) \textbf{Dependência de JS no cliente.} O tempo de carregamento inicial (LCP) é mais lento, pois depende do download e execução do \textit{bundle} JavaScript. Em dispositivos modestos, isso pode comprometer a performance se o \textit{bundle} não for estritamente otimizado.

\subsubsection{Observação sobre a Ausência do FID em CSR.}
No \acrshort{csr}, o \acrfull{fid} não é capturado da maneira tradicional, pois a página precisa ser 'hidratada' pelo JavaScript antes de se tornar completamente interativa. Esse processo de hidratação impede que o \acrshort{fid} meça corretamente a primeira interação do usuário, tornando-o menos relevante em ambientes \acrshort{csr}. Por isso, o \acrfull{inp} é utilizado como métrica alternativa, já que ele captura o tempo de resposta entre a interação do usuário e a atualização visual da página, oferecendo uma avaliação mais precisa da interatividade contínua, o que é especialmente importante em páginas altamente dinâmicas como as renderizadas no cliente.

\subsection{Discussão}
\label{subsec:discussao-comparativa}
Os resultados alinham-se e fornecem validação empírica para as recomendações consolidadas na literatura, confirmando que não existe uma solução universalmente superior. A escolha é uma decisão estratégica que transcende a tecnologia.

\textbf{(i) Recursos do servidor.} O \acrshort{ssr} consumiu levemente mais CPU/memória (ainda baixos), resultado do processamento por requisição. O \acrshort{csr}, por outro lado, opera como \emph{static hosting}, com custo mínimo no host, transferindo a carga computacional para o dispositivo do usuário.

\textbf{(ii) Experiência do usuário.} Em \acrshort{ssr}, a entrega inicial é perceptivelmente rápida e os escores de SEO e Acessibilidade são superiores. Em \acrshort{csr}, a interatividade após o carregamento é o grande trunfo, mas o principal cuidado é a CLS, que pode degradar a experiência.

\textbf{(iii) Coerência entre campo e laboratório.} Os \textit{Web Vitals} em campo e o TBT do Lighthouse convergiram, indicando baixo bloqueio de \emph{main thread} em ambos os modelos. As divergências em CLS (CSR) e SEO(CSR $<$ SSR) são áreas classicamente sensíveis à estratégia de renderização.

\subsection{Implicações práticas}
A decisão entre as arquiteturas deve ser ponderada à luz dos objetivos do produto, das expectativas do usuário e das capacidades da equipe de desenvolvimento.
\begin{itemize}
    \item \textbf{Priorizar SSR} para aplicações de conteúdo, como portais de notícias, plataformas de e-commerce e blogs, onde a otimização para mecanismos de busca (SEO) é vital, e a velocidade da primeira impressão (TTFB e FCP) é um fator determinante para o sucesso do negócio.
    \item \textbf{Priorizar CSR} para aplicações web ricas e complexas, como dashboards analíticos, sistemas de gestão interna (ERPs) e plataformas de software como serviço (SaaS), que geralmente são acessadas após autenticação e onde a fluidez da interatividade contínua (INP) é mais valorizada do que a indexabilidade de páginas individuais.
\end{itemize}

\subsection{Ajustes recomendados}
\textbf{SSR (mitigar LCP/INP).} Otimizar e dimensionar imagens (usar \texttt{priority} para \emph{hero} e \emph{lazy} abaixo da dobra), considerar \emph{streaming}/\emph{partial hydration}/\emph{Server Components}, reduzir JS não crítico e declarar \texttt{preload}/\texttt{dns-prefetch} para recursos essenciais. \\
\textbf{CSR (mitigar CLS e elevar SEO).} Reservar dimensões de mídia (\texttt{width}/\texttt{height} ou \texttt{aspect-ratio}); evitar inserções acima da dobra durante a hidratação; \texttt{font-display: swap} com \emph{preload} de fontes e imagens \emph{hero}; placeholders do tamanho final; completar metadados e sinalizações (\emph{title}/\emph{description}/\texttt{canonical}/\texttt{robots}/\texttt{hreflang} quando aplicável).

\subsection{Síntese}
No cenário testado, \textbf{SSR} e \textbf{CSR} entregaram boa experiência de uso com baixa pressão de recursos. O SSR destacou-se por SEO e exibição inicial mais previsível, sendo ideal para aplicações de conteúdo. O CSR reduziu o custo no servidor e garantiu interatividade fluida, com a contrapartida de maior propensão a CLS e desafios de SEO, mostrando-se adequado para aplicações ricas e logadas. A decisão final deve considerar o perfil de tráfego, a exigência de descoberta e o tipo de interação, aplicando as otimizações para mitigar os pontos fracos identificados em cada abordagem.

\chapter{Conclusão}
\label{cap:conclusao}

Este trabalho de conclusão de curso realizou uma análise comparativa aprofundada entre as arquiteturas de renderização no lado do cliente (\acrfull{csr}) e no lado do servidor (\acrfull{ssr}), com o intuito de dissecar suas implicações na performance, na experiência do usuário e nos desafios de implementação. A partir do desenvolvimento e teste de duas versões da plataforma "WallTech" em um ambiente controlado, foi possível validar empiricamente os trade-offs inerentes a cada abordagem.

A investigação confirmou que a escolha arquitetônica molda fundamentalmente a jornada do usuário. A abordagem \acrshort{ssr} demonstrou ser superior na fase inicial da interação, entregando um primeiro conteúdo visual (\acrfull{fcp}) de forma notavelmente rápida e garantindo excelente estabilidade visual (\acrfull{cls}). Essa performance inicial, aliada à otimização natural para mecanismos de busca (\acrfull{seo}), a torna ideal para aplicações onde a primeira impressão e a descoberta de conteúdo são críticas.

Em contrapartida, o \acrshort{csr} redefiniu a experiência após o carregamento inicial, oferecendo uma interatividade superior e navegação quase instantânea, refletida em um baixo \acrfull{inp}, que emula a fluidez de um aplicativo nativo. Contudo, essa vantagem vem ao custo de um carregamento inicial mais lento e de uma maior suscetibilidade a instabilidades de layout, exigindo uma disciplina de desenvolvimento rigorosa para mitigar os pontos fracos da abordagem.

Portanto, este trabalho cumpriu seus objetivos ao demonstrar, com dados concretos, que a decisão entre \acrshort{csr} e \acrshort{ssr} é uma escolha estratégica. Ela se resume a um trade-off fundamental sobre onde a carga computacional deve residir e o que se deve priorizar: a velocidade da primeira renderização e a indexabilidade do \acrshort{ssr}, ou a interatividade contínua e a redução de custos de servidor do \acrshort{csr}.

A principal contribuição desta pesquisa reside na sua metodologia sistemática e reprodutível, que oferece um roteiro prático para a avaliação comparativa de arquiteturas frontend. Como limitação, reconhece-se que o ambiente de teste local, embora essencial para o controle experimental, não captura a variabilidade de redes e dispositivos do mundo real.

Para trabalhos futuros, sugere-se a análise de arquiteturas híbridas, como \acrfull{ssg} e \acrfull{isr}, que prometem unir o melhor dos dois mundos. Adicionalmente, seria de grande valia investigar o impacto de paradigmas emergentes, como os \textit{React Server Components} e o \textit{Streaming SSR}, e complementar os dados quantitativos com estudos qualitativos de usabilidade, para aprofundar a compreensão sobre a percepção real do usuário.

\postextual

\bibliography{Referencias}

% Imprime uma página indicando o início dos apêndices
% \partapendices
%\part*{Apêndices}

%\begin{apendicesenv}

%
\chapter{Questionário de Avaliação do Protótipo}









%\end{apendicesenv}

%\begin{anexosenv}

%\input{pos/pos-anexos}

%\end{anexosenv}

\end{document}
