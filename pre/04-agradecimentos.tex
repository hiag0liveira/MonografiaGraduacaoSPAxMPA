\begin{agradecimentos}

Em primeiro lugar, agradecemos a Deus, por nos dar força, fé e sabedoria para superar todos os desafios e dificuldades ao longo de nossa jornada acadêmica.

Agradecemos de maneira profunda e especial às nossas famílias, nossos alicerces. Em especial às nossas mães, que sempre nos apoiaram com amor incondicional em cada passo desta jornada. Somos imensamente gratos por todo o sacrifício, carinho e dedicação. Em particular, um de nós presta uma homenagem carinhosa à sua bisavó, cuja memória e legado serviram como uma silenciosa e constante fonte de inspiração. Se hoje celebramos esta conquista, é porque tivemos o privilégio de contar com o suporte de vocês.

Ao nosso orientador, Prof. Dr. Ronaldo Amaral Santos, expressamos nossa mais profunda gratidão. Seu apoio, orientação precisa e ensinamentos foram fundamentais não apenas para a condução deste trabalho, mas para nosso crescimento como pesquisadores. Agradecemos por sua paciência, dedicação e pelo conhecimento compartilhado.

Gostaríamos de registrar um agradecimento especial ao Prof. Dr. Rogerio Atem. A oportunidade de participar dos projetos de bolsa sob sua mentoria no polo de inovação nos proporcionou nossas primeiras experiências práticas, sendo um passo fundamental para a aplicação dos conhecimentos teóricos e para o despertar de nossa paixão pela tecnologia. Sua confiança foi essencial em nosso início de carreira.

Agradecemos a todos os professores do Instituto Federal Fluminense que, ao longo do curso, compartilharam seus conhecimentos e nos inspiraram, bem como aos nossos colegas e amigos, que tornaram a jornada mais leve com momentos de companheirismo e apoio mútuo.

Por fim, esta conquista reforça nossa crença de que a educação é a chave não apenas para atingir objetivos pessoais, mas para transformar a sociedade em um lugar mais próspero, solidário e justo.

\end{agradecimentos}