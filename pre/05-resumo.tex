\begin{resumo}

% O comando lipsum abaixo é um gerador automático de texto.
% Substitua-o pelo texto do seu resumo.
% Lembre-se: Um resumo deve ser um parágrafo único que apresente os seguintes tópicos:

% Contexto;
% Problema;
% Objetivo;
% Justificativa;
% Metodologia;
% Resultado;
% Conclusão.

% O crescimento exponencial da web, com milhares de sites surgindo diariamente, tem elevado a complexidade do desenvolvimento de aplicações web e evidenciado a importância da escolha adequada da estratégia de renderização de conteúdo. Dentre as principais abordagens estão o \textit{Client-Side Rendering} (CSR) e o \textit{Server-Side Rendering} (SSR), cada uma com características distintas que impactam diretamente a performance, a experiência do usuário (UX), a escalabilidade e a otimização para motores de busca (SEO). Este trabalho tem como objetivo principal realizar uma análise comparativa entre CSR e SSR, destacando seus efeitos técnicos e funcionais em aplicações web modernas. A justificativa fundamenta-se na lacuna de estudos de caso práticos e aprofundados, especialmente no contexto nacional, que analisem criticamente os impactos reais da adoção dessas abordagens. A metodologia adotada envolveu fundamentação teórica, mapeamento sistemático da literatura com uso da técnica \textit{PICOC}, e a implementação de um estudo de caso realista, com coleta de métricas de desempenho, tempo de carregamento, impacto em SEO e UX. Os resultados apontaram que, embora o CSR seja mais vantajoso em aplicações altamente interativas, o SSR apresenta melhor desempenho inicial e maior indexabilidade. Conclui-se que a escolha entre CSR e SSR deve considerar o perfil do sistema, a infraestrutura disponível e os objetivos estratégicos do projeto, sendo muitas vezes recomendável a adoção de soluções híbridas.

% \textbf{Palavras-chave: } Renderização do lado do cliente, CSR, Renderização do lado do servidor, SSR, Renderização Web, Desempenho, SEO, UX.

% \end{resumo}


% \begin{resumo}
Com a crescente complexidade das aplicações web e a busca por experiências de usuário ricas e performáticas, a escolha da estratégia de renderização de conteúdo tornou-se uma decisão arquitetural fundamental. Nesse cenário, o \textit{Client-Side Rendering} (CSR) e o \textit{Server-Side Rendering} (SSR) emergem como as principais abordagens, cada uma com \textit{trade-offs} significativos: o CSR favorece a interatividade contínua, enquanto o SSR otimiza o carregamento inicial e a indexação por motores de busca (SEO). A literatura, no entanto, carece de estudos de caso práticos que comparem seus impactos de forma direta e controlada.

Visando preencher essa lacuna, este trabalho apresenta uma análise comparativa detalhada, fundamentada em um estudo de caso realista: o desenvolvimento de uma plataforma de notícias em duas versões funcionalmente idênticas, uma com React (CSR) e outra com Next.js (SSR). Por meio da coleta de métricas de \textit{Core Web Vitals} em um ambiente controlado com contêineres Docker, a pesquisa avalia empiricamente os efeitos de cada arquitetura no desempenho e na experiência do usuário. O estudo busca, assim, oferecer uma contribuição clara para a compreensão dos cenários de aplicação de cada abordagem, auxiliando equipes de desenvolvimento a tomarem decisões mais informadas e estratégicas.

\textbf{Palavras-chave:} Renderização do lado do cliente, CSR, Renderização do lado do servidor, SSR, Renderização Web, Desempenho, SEO, UX.
\end{resumo}
