\begin{resumo}[Abstract]
\begin{otherlanguage*}{english}
With the growing complexity of web applications and the demand for rich, high-performance user experiences, the choice of a content rendering strategy has become a fundamental architectural decision. In this context, \textit{Client-Side Rendering} (CSR) and \textit{Server-Side Rendering} (SSR) emerge as the main approaches, each with significant \textit{trade-offs}: CSR favors continuous interactivity, while SSR optimizes initial load times and search engine optimization (SEO). The literature, however, lacks practical case studies that directly and controllably compare their impacts.

To address this gap, this work presents a detailed comparative analysis based on a realistic case study: the development of a news platform in two functionally identical versions, one with React (CSR) and the other with Next.js (SSR). Through the collection of \textit{Core Web Vitals} metrics in a controlled environment using Docker containers, the research empirically evaluates the effects of each architecture on performance and user experience.

The empirical results revealed a clear trade-off: the SSR architecture demonstrated superiority in initial loading (TTFB and FCP) and SEO, while the CSR approach excelled in fluid interactivity (INP) and reduced server cost, although it exhibited greater visual instability (CLS). It is concluded that the choice is strategic and depends on the project's objectives: SSR is recommended for applications focused on content and discovery, such as portals and e-commerces, while CSR is suited for interaction-rich systems, such as dashboards and logged-in platforms. As a limitation, the study was conducted in a local environment, not encompassing the variability of real-world networks and devices.

\textbf{Keywords:} Client-side rendering, CSR, Server-side rendering, SSR, Web Rendering, Performance, SEO, UX.
\end{otherlanguage*}
\end{resumo}