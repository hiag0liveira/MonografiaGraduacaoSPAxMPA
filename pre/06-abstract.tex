\begin{resumo}[Abstract]
 \begin{otherlanguage*}{english}

% O comando lipsum abaixo é um gerador automático de texto, e deve ser substituído pelo seu abstract.

The exponential growth of the web, with thousands of new websites emerging daily, has increased the complexity of web application development and highlighted the importance of choosing the appropriate content rendering strategy. Among the main approaches are Client-Side Rendering (CSR) and Server-Side Rendering (SSR), each with distinct characteristics that directly affect performance, user experience (UX), scalability, and search engine optimization (SEO). This work aims to conduct a comparative analysis between CSR and SSR, emphasizing their technical and functional effects on modern web applications. The justification lies in the lack of practical and in-depth case studies, especially in the Brazilian context, that critically examine the real impact of adopting these strategies. The methodology included a theoretical foundation, a systematic literature review using the PICOC technique, and the implementation of a realistic case study, collecting performance metrics, load times, SEO impact, and UX data. The results show that while CSR is more advantageous for highly interactive applications, SSR provides better initial performance and greater indexability. It is concluded that the choice between CSR and SSR should consider the system profile, infrastructure, and strategic goals, with hybrid solutions often being the most appropriate.

\textbf{Keywords: } Client-side rendering, CSR, Server-side rendering, SSR, Web Rendering, Performance, SEO, UX.

\end{otherlanguage*}
\end{resumo}
