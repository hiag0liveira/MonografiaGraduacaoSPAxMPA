\begin{resumo}[Abstract]
\begin{otherlanguage*}{english}
    
With the growing complexity of web applications and the demand for rich, high-performance user experiences, the choice of a content rendering strategy has become a fundamental architectural decision. In this context, \textit{Client-Side Rendering} (CSR) and \textit{Server-Side Rendering} (SSR) emerge as the main approaches, each with significant \textit{trade-offs}: CSR favors continuous interactivity, while SSR optimizes initial load times and search engine optimization (SEO). The literature, however, lacks practical case studies that directly and controllably compare their impacts.

Aiming to fill this gap, this work presents a detailed comparative analysis based on a realistic case study: the development of a news platform in two functionally identical versions, one with React (CSR) and the other with Next.js (SSR). Through the collection of \textit{Core Web Vitals} metrics in a controlled environment using Docker containers, the research empirically evaluates the effects of each architecture on performance and user experience. The study thus seeks to offer a clear contribution to understanding the application scenarios for each approach, helping development teams make more informed and strategic decisions.

\textbf{Keywords:} Client-side rendering, CSR, Server-side rendering, SSR, Web Rendering, Performance, SEO, UX.
\end{otherlanguage*}
\end{resumo}